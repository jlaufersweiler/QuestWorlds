% Options for packages loaded elsewhere
\PassOptionsToPackage{unicode}{hyperref}
\PassOptionsToPackage{hyphens}{url}
%
\documentclass[
  11pt,
]{article}
% options for title colors
\usepackage{titlesec}
\titleformat*{\section}{\Large\bfseries\sffamily\color{red}}
\titleformat*{\subsection}{\large\bfseries\sffamily\color{red}}
\titleformat*{\subsubsection}{\large\bfseries\sffamily\color{red}}
\titleformat{\paragraph}{\normalsize\bfseries\sffamily\color{red}}{\theparagraph}{1em}{}
\titlespacing*{\paragraph}{0pt}{3.25ex plus 1ex minus .2ex}{0.5em}
%
% Turn off hyphenation
\usepackage[none]{hyphenat}
%
\usepackage{lmodern}
\usepackage{amssymb,amsmath}
\usepackage{ifxetex,ifluatex}
\ifnum 0\ifxetex 1\fi\ifluatex 1\fi=0 % if pdftex
  \usepackage[T1]{fontenc}
  \usepackage[utf8]{inputenc}
  \usepackage{textcomp} % provide euro and other symbols
\else % if luatex or xetex
  \usepackage{unicode-math}
  \defaultfontfeatures{Scale=MatchLowercase}
  \defaultfontfeatures[\rmfamily]{Ligatures=TeX,Scale=1}
  \setmainfont[]{Garamond}
\fi
% Use upquote if available, for straight quotes in verbatim environments
\IfFileExists{upquote.sty}{\usepackage{upquote}}{}
\IfFileExists{microtype.sty}{% use microtype if available
  \usepackage[]{microtype}
  \UseMicrotypeSet[protrusion]{basicmath} % disable protrusion for tt fonts
}{}
\makeatletter
\@ifundefined{KOMAClassName}{% if non-KOMA class
  \IfFileExists{parskip.sty}{%
    \usepackage{parskip}
  }{% else
    \setlength{\parindent}{0pt}
    \setlength{\parskip}{6pt plus 2pt minus 1pt}}
}{% if KOMA class
  \KOMAoptions{parskip=half}}
\makeatother
\usepackage{xcolor}
\IfFileExists{xurl.sty}{\usepackage{xurl}}{} % add URL line breaks if available
\IfFileExists{bookmark.sty}{\usepackage{bookmark}}{\usepackage{hyperref}}
\hypersetup{
  hidelinks,
  pdfcreator={LaTeX via pandoc}}
\urlstyle{same} % disable monospaced font for URLs
\usepackage[margin=1.25in]{geometry}
\usepackage{longtable,booktabs}
% Correct order of tables after \paragraph or \subparagraph
\usepackage{etoolbox}
\makeatletter
\patchcmd\longtable{\par}{\if@noskipsec\mbox{}\fi\par}{}{}
\makeatother
% Allow footnotes in longtable head/foot
\IfFileExists{footnotehyper.sty}{\usepackage{footnotehyper}}{\usepackage{footnote}}
\makesavenoteenv{longtable}
\usepackage{graphicx}
\makeatletter
\def\maxwidth{\ifdim\Gin@nat@width>\linewidth\linewidth\else\Gin@nat@width\fi}
\def\maxheight{\ifdim\Gin@nat@height>\textheight\textheight\else\Gin@nat@height\fi}
\makeatother
% Scale images if necessary, so that they will not overflow the page
% margins by default, and it is still possible to overwrite the defaults
% using explicit options in \includegraphics[width, height, ...]{}
\setkeys{Gin}{width=\maxwidth,height=\maxheight,keepaspectratio}
% Set default figure placement to htbp
\makeatletter
\def\fps@figure{htbp}
\makeatother
\setlength{\emergencystretch}{3em} % prevent overfull lines
\providecommand{\tightlist}{%
  \setlength{\itemsep}{0pt}\setlength{\parskip}{0pt}}
\setcounter{secnumdepth}{-\maxdimen} % remove section numbering
\usepackage{fancyhdr}
\pagestyle{fancy}
\fancyfoot[C]{QuestWorlds System Reference Document}
\fancyfoot[R]{\thepage}
\renewcommand{\headrulewidth}{0.4pt}
\renewcommand{\footrulewidth}{0.4pt}

\author{}
\date{}

\begin{document}

{
\setcounter{tocdepth}{3}
\tableofcontents
}
\hypertarget{credits-legal-information}{%
\section{0.0 Credits \& Legal
Information}\label{credits-legal-information}}

\hypertarget{legal-information}{%
\section{0.1 Legal Information}\label{legal-information}}

The \emph{QuestWorlds} System Reference Document 0.97 (``QWSRD0.97'')
describes the rules of \emph{QuestWorlds}. You may incorporate the rules
as they appear in QWSRD0.97, wholly or in part, into a derivative work,
through the use of the ORC License. You should read and understand the
terms of that License before creating a derivative work from QWSRD0.97.

\hypertarget{using-this-license}{%
\subsubsection{0.1.1 Using This License}\label{using-this-license}}

You should note that this is version of 0.97 of the \emph{QuestWorlds}
System Reference Document. We expect to release revised versions of this
SRD, especially after development of Chaosium's upcoming
\emph{QuestWorlds Core Book}. When we release the \emph{QuestWorlds Core
Book} we will update the version designation to 1.0, indicating that the
SRD reflects the text published in that book. If you are developing
materials for \emph{QuestWorlds} projects you may want to bear this in
mind. We will track any changes to the SRD at
\emph{https://github.com/ChaosiumInc/QuestWorlds}.

Once we release SRD version 1.0 we expect that to be stable for some
time.

If you have questions about this license, please reach out to Moon
Design at licensing@chaosium.com.

\hypertarget{questworlds-orc-license-statement}{%
\subsubsection{\texorpdfstring{0.1.2 \emph{QuestWorlds} ORC License
Statement}{0.1.2 QuestWorlds ORC License Statement}}\label{questworlds-orc-license-statement}}

This product is licensed under the ORC License held in the Library of
Congress at TX 9-307-067 and available online at various locations
including www.chaosium.com/orclicense, www.azoralaw.com/orclicense,
www.gencon.com/orclicense and others. All warranties are disclaimed as
set forth therein.

This product is the original work of Moon Design Publications.

If you use our ORC Content, please also credit us as follows:

QuestWorlds © copyright 2019--2023 Moon Design Publications LLC

Along with our logo

\includegraphics[width=1in,height=1in]{Logos/QW-Stamp-Red.png}
\includegraphics[width=1in,height=1in]{Logos/QW-Stamp-Black.png}

Reserved Material elements in this product include but may not be
limited to all artwork, illustrations, and graphic design, and trade
dress, and all trademarks, including \emph{Call of Cthulhu},
\emph{Chaosium}, \emph{Future-World}, \emph{Magic World},
\emph{Pendragon}, \emph{RuneQuest}, \emph{Superworld}, and \emph{Worlds
of Wonder}, and all other elements designated as Reserved Material under
the ORC license.

QuestWorlds SRD with annotations for individual contributions can be
found at GitHub: https://github.com/ChaosiumInc/QuestWorlds/

\hypertarget{credits}{%
\subsection{0.2 Credits}\label{credits}}

Original Rules: Robin D. Laws

Lead System Development: Ian Cooper

Editing -- Assistant Development -- Copyediting: Susan O'Brien

Assistant Development: Shawn Carpenter, Jonathan Laufersweiler, James
Lowder, Michael O'Brien, Jeff Richard, David Scott

Further Development: Greg Stafford, David Dunham, Mark Galeotti, Neil
Robinson, Adam RKitch, Roderick Robinson, Lawrence Whitaker,

A modified version of Chained Contests and Plot Edits from \emph{Mythic
Russia} © copyright 2006, 2010 Mark Galeotti; developed by Graham
Robinson (for ``Chained Contests'') appears in this game and is added as
Licensed Material here with permission.

Material in \emph{Section 1, Introduction} and \emph{Section 2, Basic
Mechanics} © copyright 2018 Jonathan Laufersweiler and added as Licensed
Material here with permission.

Material in \emph{Section 2, Basic Mechanics} © copyright 2020 Shawn
Carpenter and added as Licensed Material here with permission.

QuestWorlds SRD with annotations for individual contributions can be
found at GitHub: \url{https://github.com/ChaosiumInc/QuestWorlds/pulls}.

\hypertarget{introduction}{%
\section{1.0 Introduction}\label{introduction}}

\emph{QuestWorlds} is a roleplaying rules engine suitable for you to
play in any genre.

It is a traditional roleplaying game in that there is a GM and players.
The players play characters, each guided by the internal thoughts of
their character as to what decisions they make, and the GM plays the
world, including non-player characters (NPCs) and abstract threats.

It features an abstract, conflict-based, resolution method and scalable,
customizable, character descriptions. Designed to emulate the way
characters in fiction face and overcome challenges, it is suitable for a
wide variety of genres and play styles. It is particularly suited to
pulp genres (including their descendants comic books) and cinematic,
larger-than-life, action.

It is a rules-light system that facilitates beginning play easily, and
resolving conflicts in play quickly.

We refer to a rules-light but traditional roleplaying game as a
storytelling game, after Greg Stafford's definition in \emph{Prince
Valiant}.

\hypertarget{why-questworlds}{%
\subsection{1.1 Why QuestWorlds?}\label{why-questworlds}}

\emph{QuestWorlds} is meant to facilitate your creativity, and then to
get out of your way.

It is well suited to a collaborative, friendly group with a high degree
of trust in each other's creativity. Characters in \emph{QuestWorlds}
are described more in terms of their place in your imagination and the
game setting than by game mechanics.

If your group are often at odds and rely on their chosen rules kit as an
arbiter between competing visions of how the game ought to develop, or
use mechanical options to decide ``what action to take,''
\emph{QuestWorlds} is not a rules set that provides that structure. Make
sure to discuss with your group whether you are collectively on board
with trying a new play style dynamic, or if you would rather stick to
more structured systems.

\hypertarget{version}{%
\subsection{1.2 Version}\label{version}}

The first version of these rules \emph{Hero Wars} was published in 2000
(ISBN 978-1-929052-01-1)

The second version \emph{HeroQuest} was published in 2003 (ISBN
978-1-929052-12-7). We refer to this as \emph{HeroQuest} 1e to
disambiguate.

The third version \emph{HeroQuest}: Core Rules was published in 2009
(ISBN 978-0-977785-32-2). We refer to this as \emph{HeroQuest} 2e.

\emph{HeroQuest Glorantha} was published in 2015 (ISBN
978-1-943223-01-5). It is the version of the rules in \emph{HeroQuest}
2e, presented for playing in Glorantha. We refer to this as
\emph{HeroQuest} 2.1e.

\emph{QuestWorlds} was published as a System Reference Document (SRD)
(this document) in 2020. The version of the rules here is slightly
updated, mainly to clarify ambiguities, from the version presented in
\emph{HeroQuest} 2e and \emph{HeroQuest} 2.1e. This makes this ruleset
\emph{HeroQuest} 2.2e, despite the name change. However, to simplify we
identify this version as \emph{QuestWorlds} 1e.

An Appendix lists changes in this version. As the SRD is updated we will
continue to track version changes there.

\hypertarget{who-is-this-document-for}{%
\subsection{1.3 Who Is This Document
For}\label{who-is-this-document-for}}

The primary audience for this document is game-designers who wish to
utilize the \emph{QuestWorlds} rules framework to implement their own
game.

We also recognize that some people will use this document to learn about
the \emph{QuestWorlds} system before purchasing it, and some players in
games where the GM has a rule book, may use this as a reference to help
understand the rules.

For that latter reason, we address the rules here to a player.

However, this remains a technical document with few examples, advice, or
other non-rules text to help you play your game, as such are beyond the
scope of this System Reference Document.

It is expected that the designers of games you play based on these rules
will include such guidance and context as is relevant to their game's
particular genre or setting, presented in a format better suited for
learning how to play.

\hypertarget{numbering}{%
\subsection{1.4 Numbering}\label{numbering}}

Sections within this document are numbered. This is for the benefit of
game designers and reviewers.

This does not imply that game designers need number the rules in their
own games.

Numbering however makes it easy to refer to rules in this document when
page numbers may vary by presentation format for the purposes of error
trapping or tracking changes. If you need to give us feedback about this
document, that will assist us.

\hypertarget{participants}{%
\subsection{1.5 Participants}\label{participants}}

\hypertarget{players}{%
\subsubsection{1.5.1 Players}\label{players}}

You and your fellow players each create a Player Character (PC) to be
the ``avatar'' or ``persona'' whose role you will play in the game. The
PCs pursue various goals in an imaginary world, using their
\textbf{abilities}, motivations, connections, and more to solve problems
and overcome \textbf{story obstacles} that stand in their way.

When we say `you' in this document we may mean either the player or
their PC. Which of these we're addressing should be clear from the
context or explicitly noted.

\hypertarget{game-master}{%
\subsubsection{1.5.2 Game Master}\label{game-master}}

Your Game Master (GM) is the interface between your imagination and the
game-world in which the PCs have their adventures; describing the
people, places, creatures, objects, and events therein. Your GM also
plays the role of any Non-Player Characters (NPCs) with whom your PC
interacts in the course of your adventures.

We generally refer to the GM as `your GM' in this document's
player-facing language. However, if you are the GM for a given game,
this naturally refers to you.

\hypertarget{mechanics}{%
\section{2.0 Mechanics}\label{mechanics}}

In a \emph{QuestWorlds} game, stories develop dynamically as you and
your GM work together to role-play the dramatic conflict between your
group's PCs in pursuit of their goals and the challenges, or threats
that your GM presents to stand in your way. Stories advance by two
methods: conflict, where your PC is prevented from achieving their goals
because there is something that must be overcome, a \textbf{story
obstacle}, to gain a desired person, thing, or even status: the
\textbf{prize}; or there is something that must be understood, a
\textbf{story question}, to learn a secret, the past, or comprehend: a
different \textbf{prize}.

Over the course of play, your GM will present various \textbf{story
obstacles} and \textbf{revelations} as conflicts to the PCs, resulting
in either \textbf{victory} or \textbf{defeat} for your character, which
determines whether or not you gain the \textbf{prize} you sought. These
conflicts can represent any sort of challenge you might face: fighting,
a trial or debate, survival in a harsh environment, out-wooing rival
suitors, and so on.

Rather than mechanically addressing the individual tasks that make up
these conflicts, \emph{QuestWorlds} usually assesses your overall
\textbf{victory} or \textbf{defeat} in a single \textbf{contest} where
you and your GM make an opposed roll pitting your characters
\textbf{ability} vs the \textbf{resistance} the \textbf{story obstacle}
or \textbf{story question} presents to you achieving the \textbf{prize}.

Whenever the GM presents a \textbf{story obstacle} or \textbf{story
question} for you to overcome, you should \textbf{frame the contest} by
describing what you are trying to accomplish, the \textbf{prize}, and
which of your \textbf{abilities} (see below) you want to use to achieve
that \textbf{prize}, and how.

Based on that \textbf{framing} and other factors, your GM will assess
what \textbf{resistance} the characters face.

You roll a twenty-sided die (D20) against your PC's \textbf{ability},
and your GM rolls a D20 against the \textbf{resistance}. Your GM will
assess your overall \textbf{victory} or \textbf{defeat} in the contest
based on the \textbf{success} or \textbf{failure} of both rolls, and
narrates the results of your attempt to overcome the \textbf{story
obstacle} or answer the \textbf{story question} and gain the
\textbf{prize} accordingly. The direction of the story changes, in
either a big or small way, depending on whether you gain the
\textbf{prize} or not.

We encourage your GM to work with your suggestions when narrating the
\textbf{victory} or \textbf{defeat}, but the final decision rests with
them.

\hypertarget{abilities}{%
\subsection{2.1 Abilities}\label{abilities}}

Characters in \emph{QuestWorlds} are defined by the \textbf{abilities}
they use to face the challenges that arise in the course of their story.
Rather than having a standard list of attributes, skills, powers, etc.
for all characters, anything that you can apply overcome a \textbf{story
obstacle} or answer a \textbf{story question} could be one of your
\textbf{abilities}. While your GM may provide some example
\textbf{abilities} to choose from that connect your PC to a particular
story or game world (whether created by your GM or by the designer of a
particular game), you get to make up and describe most or all of your
\textbf{abilities}.

Some \textbf{abilities} might be broad descriptions of your background
or expertise, like ``Dwarf of the Chalk Hills'' or ``Private Detective''
- implying a variety of related capabilities. Others might represent
specific capabilities or assets such as ``Lore of the Ancients,''
``Captain of the Fencing Team,'' or ``The Jade Eye Medallion.''

Ultimately, \textbf{abilities} are names for the interesting things your
character can do.

\hypertarget{keywords}{%
\subsubsection{2.1.1 Keywords}\label{keywords}}

A \textbf{keyword} is a broad \textbf{ability} that often represents an
occupation, a heritage, a belief system or participation in a community.
The simplest way to use a \textbf{keyword} is as an \textbf{ability}
that represents the competencies of an occupation, knowledge and
attitudes passed down via a heritage, values of a belief system, or
relationships from a community. In the kinds of fiction that
\emph{QuestWorlds} emulates it is usually enough to know that someone
has a particular occupation or heritage to know what they can do.

A \textbf{keyword} reduces the number of \textbf{abilities} your PC
needs to track, and is a simplification to help you easily create
archetypical characters for a genre without excessive bookkeeping.

Your GM's genre pack should have a text description, which hints at the
credible uses of a \textbf{keyword}.

Any broad concept you come up with for an \textbf{ability}, may be
better presented as a \textbf{keyword}.

\hypertarget{breakouts}{%
\paragraph{2.1.1.1 Breakouts}\label{breakouts}}

A \textbf{breakout} allows you to track a competency within the
\textbf{keyword} individually, as an \textbf{ability}.

If your character is renowned for a skill that credibly falls under one
of your \textbf{keywords}, you create a \textbf{breakout ability} under
the \textbf{keyword} at a \textbf{bonus} from the \textbf{rating} of the
\textbf{keyword}. You write these specialized \textbf{breakout
abilities} under the \textbf{keyword}, along with how much they've
improved from the \textbf{keyword}:

Detective 15

\begin{itemize}
\tightlist
\item
  Deduction +5
\item
  Hard Drinking +5
\end{itemize}

In this example, whilst the \textbf{rating} for most \textbf{contests}
in which Detective was an appropriate \textbf{tactic} would be 15, for
contests involving Deduction it would be 20.

A \textbf{breakout} begins at +5 when purchased during character
creation or improvement. A \textbf{distinguishing characteristic} (see
§3.0) under a \textbf{keyword} begins at +10.

The list of \textbf{breakout abilities} in a \textbf{keyword} is
open-ended, but your GM's genre pack description of the keyword will
provide inspiration, and might include a list of suggested
\textbf{abilities} for unfamiliar settings, where it is less clear what
a \textbf{keyword} encompasses.

Even if there is a suggested list, the potential uses of the
\textbf{keyword} are always open-ended, provided they are credible. As
the genre pack description tends to be assumed in the \textbf{keyword},
which is already treated as an \textbf{ability}, it can be more
interesting to have them \emph{differentiate} your PC from the
archetype.

\hypertarget{flaws}{%
\subsubsection{2.1.2 Flaws}\label{flaws}}

Your character may have one or more \textbf{flaws}. Unlike an
\textbf{ability}, you do not use a \textbf{flaw} to accomplish
something; instead the GM uses your \textbf{flaw} to hinder you from
accomplishing something, or invokes your \textbf{flaw} to force you to
act a certain way. \textbf{Flaws} are used to enrich your character and
provide story obstacles to be overcome.

\textbf{Flaws} may be psychological challenges such as ``Addict'', fears
or compulsive behaviors such as ``Afraid of Snakes'' or ``Needs Lucky
Rabbit's Foot'', physical challenges such as ``One-Eyed,''
``Wheelchair-Bound'' or ``Asthmatic.'' A \textbf{flaw} might also be a
philosophy such as ``Code Against Killing,'' ``Pacifist,'' or ``Radical
Candor'' that limit your freedom of action. A \textbf{flaw} might be a
relationship that creates obligations such as a ``Frail Aunt,'' ``Single
Dad,'' or ``Blackmailed''.

Many \textbf{flaws} describe attributes that can be viewed positively.
By making it a \textbf{flaw} and not an \textbf{ability} you are
inviting your GM to use it to make your life more difficult, not easier.

You should not use your \textbf{flaw} to accomplish something; if you
feel that is likely, use an \textbf{ability} and flag to your GM when
you want them to treat it as a \textbf{flaw} at an appropriate moment.

Ultimately, in \emph{QuestWorlds} a \textbf{flaw} is simply something
that you invite the GM to use to hinder or prevent your character doing
something. In return for the GM exercising the \textbf{flaw} you gain
\textbf{experience points} (see §8.1).

In play your PC may work to overcome a \textbf{flaw} and you may reach
the point that you agree with your GM that story events mean that it is
no longer relevant. You can then drop that \textbf{flaw} from your
character sheet when you receive an advance (see §8.2).

You can add a new \textbf{flaw} if play suggests one might emerge, with
discussion with the GM, when you receive an \textbf{advance} (see §8.2).

\hypertarget{ratings-and-masteries}{%
\subsubsection{2.1.3 Ratings and
Masteries}\label{ratings-and-masteries}}

\emph{QuestWorlds} \textbf{abilities} are scored on a \textbf{rating} of
1--20, representing the \textbf{target number (TN)} you need to roll or
less to succeed in a \textbf{contest} (see §2.3 for more details).

Once your \textbf{ability} passes 20, you would always be able to roll
under it on a D20. So to allow abilities to scale over 20 we use
\textbf{tiers} of capability we refer to as a \textbf{masteries}.

We denote a \textbf{rating} with a \textbf{mastery} as \textbf{TN} + M,
for example 7M represents a \textbf{TN} of 7 and one \textbf{mastery}.
We represent abilities above 20M as \textbf{TN} + M2, for example 4M2
represents a \textbf{TN} of 4 and two \textbf{masteries}.

The progression of \textbf{ability} \textbf{ratings} is: 1-20, 1M-20M,
1M2-20M2, 1M3-20M3 and so on.

Specific \emph{QuestWorlds} games or genre packs may use other symbols
relevant to their setting or genre to denote \textbf{mastery} instead of
M. If so, this should be clearly noted by their designers.

For how \textbf{masteries} work in play, see §2.3.6.

\hypertarget{no-relevant-ability}{%
\subsubsection{\texorpdfstring{2.1.4 No Relevant
\textbf{Ability}}{2.1.4 No Relevant Ability}}\label{no-relevant-ability}}

You may sometimes be faced with a \textbf{story obstacle} or
\textbf{story question} for which you have no relevant \textbf{ability}
whatsoever. In such cases, you may still enter into conflict with the
\textbf{story obstacle} using a \textbf{stretch} or a \textbf{rating} of
5 for your \textbf{contest} roll.

\hypertarget{understanding-ratings}{%
\subsubsection{2.1.5 Understanding
Ratings}\label{understanding-ratings}}

\emph{QuestWorlds} treats \textbf{ratings} as a measure of how effective
you are at solving problems with the \textbf{ability}, and does not
limit what you can do with that \textbf{ability}, provided your actions
are credible in genre.

\hypertarget{possessions-and-equipment}{%
\subsection{2.2 Possessions and
Equipment}\label{possessions-and-equipment}}

Your character will generally be considered to have whatever equipment
is reasonably implied by your abilities. Having an ``Athenian Hoplite''
\textbf{ability} will mean that your character possesses bronze armor, a
shield, a spear, and a short-sword; while a ``Country Doctor'' would be
expected to have a well-stocked medical-bag and possibly a horse \&
buggy in the right setting.

However, if you wish your character to possess something that is
particularly special, interesting, or unusual, you may also enumerate it
as a rated \textbf{ability} in its own right, just like any other
\textbf{ability} your character might use to solve a problem.

In play, the degree to which you can overcome \textbf{story obstacles}
with your possessions depends not on any qualities inherent to the
objects themselves, but to the \textbf{rating} of your relevant
\textbf{ability}. However the significance of various sorts of gear lies
in the types of actions you can credibly propose, and what their impact
might reasonably be. An ``Invisibility Cloak'' \textbf{ability} implies
very different fictional capabilities than ``Souped-up Muscle Car''
does.

Conversely, if in the course of play you find your character in a
situation without equipment essential to utilize an ability effectively,
or where your character's gear is poorly suited to the task at hand,
your GM may take into account in assessing credibility-based
\textbf{situational modifiers} (see §2.5).

\hypertarget{wealth}{%
\subsubsection{2.2.1 Wealth}\label{wealth}}

In \emph{QuestWorlds}, wealth is treated as just another way to overcome
\textbf{story obstacles}. Many characters may not even have an explicit
wealth \textbf{ability}, with their wealth or assets instead implied by
\textbf{abilities} representing their background, profession, or status.
Whether explicit or implied, the relevant \textbf{rating} is not an
objective measure of the size of your fortune, but instead indicates how
well you solve problems with money and resources.

\hypertarget{contest-procedure}{%
\subsection{2.3 Contest Procedure}\label{contest-procedure}}

You choose an \textbf{ability} relevant to the conflict at hand,
describe exactly what you are trying to accomplish, and how. Your GM may
modify these suggested actions to better fit the fictional
circumstances, and describe the actions of the NPCs or forces on the
other side of the conflict.

\hypertarget{framing-the-contest}{%
\subsubsection{2.3.1 Framing the Contest}\label{framing-the-contest}}

\hypertarget{contest-framing-overview}{%
\paragraph{2.3.1.1 Contest Framing
Overview}\label{contest-framing-overview}}

When a conflict arises during the game, you and your GM start by clearly
agreeing on:

\begin{itemize}
\tightlist
\item
  What goal you are trying to achieve. We call this the \textbf{prize}.
\item
  What the \textbf{story obstacle} is you are trying to overcome or
  \textbf{story question} you are trying to answer.
\item
  What \textbf{tactic} you are using to and overcome it.
\end{itemize}

This process is called \textbf{framing the contest}.

\hypertarget{conflict-goals-vs-obstacles}{%
\paragraph{2.3.1.2 Conflict: Goals vs
Obstacles}\label{conflict-goals-vs-obstacles}}

\textbf{Contests} in \emph{QuestWorlds} don't simply tell you how well
you performed at a particular task: they tell you whether or not you
overcame a \textbf{story obstacle}, or learned the answer to a
\textbf{story question} which moves the story in a new direction. Unlike
some other roleplaying games, a \textbf{contest} in \emph{QuestWorlds}
does not resolve a task, it resolves the whole \textbf{story obstacle}
or provides a complete answer to the \textbf{story question}.

If you need secret records which are stored in a vault within a
government compound, your goal is to get the information to answer a
\textbf{story question}. Answering that \textbf{story question} may
involve many possible tasks, evading guards, lock-picking, forging
credentials, etc. - but the \textbf{contest} doesn't address those
individually. The \textbf{contest} is framed around the entire conflict
against the \textbf{story question} as a whole.

In a fight, your \textbf{story obstacle} may be the opponents
themselves, who you are fighting to capture or kill. Just as often you
are seeking another goal and you might just as easily attain it by
incapacitating or evading your foes. In this case, beating the enemy is
a task, not the \textbf{story obstacle}. For example, if an
\textbf{ally} has been accused of treason by the King, your goal could
be to prove the \textbf{ally's} innocence. The power of the King
threatening your \textbf{ally} is a \textbf{story obstacle} to be
overcome, and a trial by combat could be a \textbf{contest} to resolve
the conflict with an \textbf{ability} like ``Knight Errant.''

In a court trial, your goal is likely a particular verdict, while the
\textbf{story obstacle} might be the opposing lawyer, an unjust law, or
even the justice system itself. In this case, jury selection, a closing
argument, revelatory evidence, or legal procedural challenges are tasks,
not the entire \textbf{story obstacle}. The overall conflict encompasses
all those things.

A conflict to overcome a \textbf{story obstacle} or \textbf{story
question} moves the story forward when it is resolved. If it is merely a
step toward resolving a \textbf{story obstacle} it is a task and not a
conflict. While those component tasks may be interesting parts of
narrating \textbf{tactics} and \textbf{results}, your GM should be sure
to look for the \textbf{story obstacle} or \textbf{story question} in
conflict when framing a \textbf{contest}.

If there is no \textbf{story obstacle} or \textbf{story question} to
your actions, your GM should not call for a \textbf{contest} but simply
let you narrate what you do, provided that seems credible.

For example, you are traveling from one star system to another. In the
next star system you hope to confront the aged rebel who holds
long-forgotten secrets that could bring freedom to the galaxy. Your GM
feels there is no useful \textbf{story obstacle} for you to
\textbf{contest} against, and so lets you describe heading down to the
spaceport to secure a ship, meeting the captain and crew of your vessel,
and traveling to the next world. Your GM encourages you to summarize
what happens quickly so you can get to the meeting with the old rebel.
Your GM knows their \textbf{story question} to convince the old rebel to
part with their secrets is the real drama.

\hypertarget{no-repeat-attempts}{%
\paragraph{2.3.1.3 No Repeat Attempts}\label{no-repeat-attempts}}

A \textbf{contest} represents all of your attempts to overcome a
\textbf{story obstacle} or reveal the answer to the \textbf{story
question}. If you lose it means that no matter how many times you tried
to solve the problem, you finally had to give up. You can try again only
if you use a new \textbf{tactic} to overcome the \textbf{story obstacle}
or answer the \textbf{story question}.

\hypertarget{tactics}{%
\paragraph{2.3.1.4 Tactics}\label{tactics}}

You either choose an \textbf{ability} that represents any `key moment'
in overcoming that \textbf{story obstacle} or answering that
\textbf{story question}, or a broad \textbf{ability} that lets you
overcome the whole \textbf{story obstacle} or solve the mystery of the
\textbf{story question}. We call this choosing a \textbf{tactic}.

Your \textbf{tactic} might describe your using an \textbf{ability} that
helps you overcome a task within the \textbf{story obstacle} or
\textbf{story question}: sneaking past the guards, picking the locks,
choosing the right jury or skewering your opponent with your foil. Or,
your \textbf{tactics} might describe using a broad ability like
``Ninja'', ``Lawyer'', or ``Fencer'' to overcome all those challenges
that might form part of the \textbf{story obstacle} or \textbf{story
question}.

Either way, if you succeed at that roll, you overcome the whole
\textbf{story obstacle} or learn the answer to the \textbf{story
question}. Or by failing at that roll, you fail to overcome the
\textbf{story obstacle} or reveal the answer to the \textbf{story
question}, not just fail at one task.

When deciding on your \textbf{tactic}, focus on how your unique
abilities would help you overcome the \textbf{story obstacle} or reveal
the answer to the \textbf{story question}. This as the ``key moment''
where we focus on your PC. Use this moment to reveal your PC's strengths
to the group.

Your GM will determine if your \textbf{tactic} passes a
\textbf{credibility test}. Credibility depends on the genre, as what is
not credible in a gritty police procedural might be in pulp. If in
dispute, your GM should discuss with the group whether they consider
your \textbf{tactic} credible for the genre. If your action is not
credible, your GM will ask you to choose a different \textbf{tactic}.

\textbf{Incredible abilities} in some genres give you the capability to
do the incredible. For example in a superhero genre you might fly or be
invulnerable to bullets, in a fantasy genre hurl magical lightning
bolts. A genre pack for the game should help define what incredible
\textbf{tactics} are allowed for that game as part of an
\emph{Incredible Powers Framework}.

The GM can narrate the remaining tasks that make sense of the story
depending on your \textbf{success} with that roll, or have them occur
`off-stage' for speed. Think of the way TV or Cinema often cuts to the
key moment of drama in a break-in, over showing us the whole heist from
beginning to end.

\hypertarget{target-number-bonuses-and-penalties}{%
\subsubsection{2.3.2 Target Number, Bonuses and
Penalties}\label{target-number-bonuses-and-penalties}}

Your \textbf{ability's} \textbf{rating} may be modified by a number of
factors. Your \textbf{target number}---the number you must roll under or
equal to on a D20 to succeed---is your \textbf{rating} with any
applicable \textbf{modifiers}. Positive \textbf{modifiers} are
\textbf{bonuses}; negative \textbf{modifiers} are \textbf{penalties}.

\textbf{Bonuses}, may raise your \textbf{target number} high enough to
gain a \textbf{mastery}. \textbf{Penalties}, may lower an \textbf{target
number} to the point where it loses one or more \textbf{masteries}.

The following rules sections describe sources of \textbf{modifiers}:
\textbf{augments} (see §2.6), \textbf{hindrances} (see §2.7),
\textbf{stretches} and \textbf{situational modifiers} (see §2.5),
\textbf{consequences} and \textbf{benefits} (see §2.8).

\textbf{Modifiers} apply to PC's \textbf{target numbers} only (see
§2.3.3). The GM applies a \textbf{bonus} or \textbf{penalty} to the
\textbf{resistance} to reflect the needs of the story, but after that it
is not further modified.

If \textbf{penalties} reduce your \textbf{target number} to 0 or less,
any attempt to use it automatically \textbf{results} in
\textbf{failure}. You must find another way to achieve your aim.

\hypertarget{resistance}{%
\subsubsection{2.3.3 Resistance}\label{resistance}}

Your GM chooses a \textbf{resistance} to represent the difficulty of the
\textbf{story obstacle} or \textbf{story question}. By default, the
\textbf{base resistance} starts at 10.

When setting \textbf{resistance}s it is important to understand that
whilst traditional roleplaying games simulate an imaginary reality,
\emph{QuestWorlds} emulates the techniques of fictional storytelling.

Understanding this distinction will help you to play the game in a
natural, seamless manner.

For example, let's say that your GM is playing a game inspired by
fast-paced, non-fantastic, martial arts movies in a contemporary
setting. You are running along a bridge, pacing a hovercraft, piloted by
the main bad guy. You want your character, Joey Chun, to jump onto the
hovercraft and punch the villain's lights out.

Your GM starts with the proposed action's position in the storyline.
They consider a range of narrative factors, from how entertaining it
would be for you to have a succeed, how much failure would slow the
pacing of the current sequence, and how long it has been since you last
scored a thrilling victory. If, after this, they need further reference
points, your GM can draw inspiration more from martial arts movies than
the physics of real-life jumps from bridges onto moving hovercraft.
Having decided how difficult the task ought to be dramatically,your GM
will then supply the physical details as color, to justify their choice
and create suspension of disbelief, the illusion of authenticity that
makes us accept fictional incidents as credible on their own terms. If
they want Joey to have a high chance of success, your GM describes the
distance between bridge and vehicle as impressive (so it feels exciting
if you make it) but not insurmountable (so it seems believable if you
make it).

In \emph{QuestWorlds} your GM will pick a \textbf{resistance} based on
dramatic needs and then justify it by adding details into the story.

Your GM determines the \textbf{resistance} from a \textbf{base
resistance}. If your GM feels that it is hard then they will increase
the \textbf{resistance} by a \textbf{modifier} depending on their view
of how difficult the obstacle is for you (see Modifiers in §2.4).
Increasing \textbf{modifiers} make it harder to succeed, and decreasing
\textbf{modifiers} easier.

The \textbf{modifier} never reduces the \textbf{resistance} value below
0. If the GM assesses a \textbf{modifier} for the resistance that would
take the \textbf{target number} below 0, it becomes an \textbf{assured
contest} (see below).

All \textbf{contests} use the \textbf{base resistance} + \emph{optional}
\textbf{modifier}, except for \textbf{contests} to determine
\textbf{augments}. \textbf{Augmenting} always faces the \textbf{base
resistance}.

\hypertarget{resistances-table}{%
\paragraph{2.3.3.1 RESISTANCES TABLE}\label{resistances-table}}

\begin{longtable}[]{@{}ccc@{}}
\toprule
Resistance & TN & Modifier \\
\midrule
\endhead
Simple & 0 & -20 \\
Easy & 0 & -15 \\
Routine & 0 & -10 \\
Straightforward & 5 & -5 \\
Base & 10 & - \\
Challenging & 15 & 5 \\
Hard & 20 & 10 \\
Punishing & 5M & 15 \\
Exceptional & 10M & 20 \\
\bottomrule
\end{longtable}

Although a TN of 0 is treated as an \textbf{assured contest} (see
below), further \textbf{modifiers} (see §2.4) may adjust that value so
that there is a \textbf{TN} above 0.

We show the \textbf{target number} for the \textbf{base resistance} of
10, and the \textbf{modifier} value to use if you are using Resistance
Progression (see §2.13) to figure this from the new \textbf{base
resistance}.

\hypertarget{resolution-methods}{%
\subsubsection{2.3.4 Resolution Methods}\label{resolution-methods}}

The basic resolution methods are as follows:

\hypertarget{contest}{%
\paragraph{2.3.4.1 Contest}\label{contest}}

The \textbf{contest} is \emph{QuestWorlds}' primary resolution mechanic
for overcoming \textbf{story obstacles}, and is used the most often
where the outcome is uncertain. It also provides the foundation for
other types of uncertain \textbf{contest}, including several
\textbf{sequences} (see §5.0). As such, it receives both an overview of
key concepts here as well as a more detailed treatment in §4.

A \textbf{contest} can be summarized as follows:

\begin{enumerate}
\def\labelenumi{\arabic{enumi}.}
\tightlist
\item
  You and your GM agree upon the terms of the \textbf{contest}.
\item
  Your \textbf{target number (TN)} is your \textbf{rating}, adding any
  \textbf{augments} (see §2.6), \textbf{hindrances} (see §2.7),
  \textbf{stretches} and \textbf{situational modifiers} (see §2.5),
  \textbf{consequences} and \textbf{benefits} (see §2.8).
\item
  You roll a D20 vs your relevant \textbf{target number}, while your GM
  rolls a D20 vs the \textbf{resistance}.
\item
  Your GM determines the difference in the \textbf{successes} between
  the two rolls to assesses the \textbf{outcome} (see §2.3.6).
\item
  Your GM then narrates the \textbf{outcome} of the conflict as
  appropriate and assesses any \textbf{benefits} or
  \textbf{consequences} that arose (see §2.8).
\end{enumerate}

If you enter into conflict with another player rather than a
\textbf{story obstacle} or \textbf{story question} presented by your GM,
you both roll your relevant abilities for the \textbf{contest} instead
of against a GM-set \textbf{resistance}, and your GM interprets the
\textbf{results}, as described above.

\hypertarget{assured-contest}{%
\paragraph{2.3.4.2 Assured Contest}\label{assured-contest}}

Some \textbf{obstacles} don't require a roll to overcome. You'll just do
it and keep going, much as you get dressed in the morning or drive your
car to work. We call these kinds of contests \textbf{assured} contests
because your \textbf{victory} is assured. Your GM may want to describe
your \textbf{victory} as a sweat inducing challenge for you, even though
there is no risk of \textbf{defeat}, to highlight the heroic struggle of
your PC to beat the obstacle, nonetheless.

As your character advances, the challenges that qualify for
\textbf{assured contests} will become more complex. If you face a
driving challenge, the bar for assured will be much lower for a champion
Formula 1 racer than a typical commuter.

\textbf{Assured contests} are the GM's primary tool to establish your
character's competence. This makes them one of the most powerful and
frequently used tools in a GM's tool chest. Remember, your GM doesn't
have to, and usually shouldn't advise you you're involved in an
\textbf{assured contest}, so it's best to treat all \textbf{contests} as
if your skin is on the line.

Your GM may also use an \textbf{assured contest} when there is no
interesting story branch from \textbf{defeat}. If failing to open the
derelict spaceship's hatch means that the story of your exploration of
the ancient space hulk would end abruptly, your GM may choose to make it
an \textbf{assured contest}. \textbf{Assured contests} may be used to
find clues when your GM is running a mystery and correct application of
one of your \textbf{abilities} should reveal the information and allow
the story to continue, over becoming mired due to a missed roll and
missing clue.

Sometimes your GM will decide potential complications could arise in
overcoming an \textbf{story obstacle} or answering a \textbf{story
question}. Or they may want to give you a boost if you do particularly
well. If so, they will call for you to make a die roll even though your
\textbf{victory} is not in question. Your GM will use your die roll
\textbf{outcome} (see §2.3.7) to decide if any unforeseen
\textbf{consequences} or \textbf{benefits} arose from your actions, but
still gives you a \textbf{victory}.

An \textbf{assured contest} can be summarized as follows:

\begin{enumerate}
\def\labelenumi{\arabic{enumi}.}
\tightlist
\item
  You and your GM agree upon the terms of the \textbf{contest}.
\item
  The GM may decide that you simply gain the \textbf{victory} and there
  are no \textbf{consequences} or \textbf{benefits} beyond that.
\item
  If not the GM conducts a contest.
\item
  Your \textbf{target number (TN)} is your \textbf{rating}, adding any
  \textbf{augments} (see §2.6), \textbf{hindrances} (see §2.7),
  \textbf{stretches} and \textbf{situational modifiers} (see §2.5),
  \textbf{consequences} and \textbf{benefits} (see §2.8).
\item
  You roll a D20 vs your relevant \textbf{target number}, while your GM
  rolls a D20 vs the \textbf{resistance}.
\item
  Your GM compares the difference \textbf{successes} between the two
  rolls to assesses the \textbf{outcome} (see §2.3.7).
\item
  Your GM then narrates how you obtained your \textbf{victory} and any
  \textbf{benefits} or \textbf{consequences} that arose.
\end{enumerate}

\hypertarget{die-rolls}{%
\subsubsection{2.3.5 Die Rolls}\label{die-rolls}}

To determine how well you use an \textbf{ability}, roll a 20-sided die
(D20). At the same time, your GM rolls for the \textbf{resistance}.

Compare your rolled number with your \textbf{target number (TN)} to
determine the \textbf{result}. Remember that \textbf{bonuses} and
\textbf{penalties} may mean your \textbf{TN} gains or loses a
\textbf{mastery}.

\begin{itemize}
\tightlist
\item
  \textbf{Big Success}: If the die roll is equal to the \textbf{TN}, you
  succeed brilliantly, and gain \emph{two} \textbf{successes}. This is
  the best \textbf{result} possible.
\item
  \textbf{Success}: If the die roll is less than the \textbf{TN}, you
  succeed, but there is nothing remarkable about the success. You gain
  \emph{one} \textbf{success}.
\item
  \textbf{Failure}: If the die roll is greater than the \textbf{TN}, you
  fail. Things do not happen as hoped. You gain \emph{zero}
  \textbf{successes}
\end{itemize}

\hypertarget{additional-successes}{%
\subsubsection{2.3.6 Additional Successes}\label{additional-successes}}

You can gain additional \textbf{successes} beyond the dice roll. An
additional \textbf{success} comes from one of two sources.

\begin{itemize}
\tightlist
\item
  Each \textbf{mastery} you have gives you an additional
  \textbf{success}.
\item
  You can spend a \textbf{story point} (see §7.0) to receive an
  additional \textbf{success}.
\end{itemize}

If you have multiple \textbf{masteries} you receive additional successes
for \textbf{each} of them.

The additional \textbf{successes} add to any \textbf{success} obtained
on the die roll.

If a PC with a \textbf{mastery} rolls a \textbf{big success} they
receive an \textbf{additional success} for the \textbf{mastery} and two
\textbf{successes} for the \textbf{big success}, making a total of
\emph{three} \textbf{successes}.

\hypertarget{outcome}{%
\subsubsection{2.3.7 Outcome}\label{outcome}}

Your \textbf{successes} and the \textbf{resistance's} successes are
compared to determine your overall \textbf{outcome} which will be either
\textbf{victory} or \textbf{defeat} for the \textbf{contest} as a whole.

For a contest:

\begin{itemize}
\tightlist
\item
  If you have a \emph{more} \textbf{successes} than the GM, then you
  have a \textbf{victory} and you gain the \textbf{prize}.
\item
  If you have a \emph{fewer} \textbf{successes} than the GM, then you
  are \textbf{defeated} and do not gain the \textbf{prize}.
\item
  If you both have the \emph{same} number \textbf{successes}, including
  if you both have \emph{zero} \textbf{successes}, then the higher roll
  has a \textbf{victory} and gains the \textbf{prize}. If your rolls
  tie, then there is a standoff with neither side able to take control
  of the \textbf{prize}.
\end{itemize}

For an assured contest:

\begin{itemize}
\tightlist
\item
  You have a \textbf{victory} and you gain the \textbf{prize} set out
  when the \textbf{contest} was framed.
\end{itemize}

\hypertarget{degrees-of-victory}{%
\paragraph{2.3.7.1 Degrees of Victory}\label{degrees-of-victory}}

Your \textbf{degree} is the difference between your \textbf{successes}
and the \textbf{resistance's} \textbf{successes}. It is a
\textbf{degree} of \textbf{victory} if you win and a \textbf{degree} of
\textbf{defeat} if you lose.

If you have \emph{two} \textbf{successes} and the resistance has
\emph{zero} \textbf{successes} you have \emph{two} \textbf{degrees} of
\textbf{victory}. If you have \emph{zero} \textbf{successes} and the
\textbf{resistance} has \emph{one} \textbf{success} you have \emph{one}
\textbf{degree} of defeat. If you have \emph{one} \textbf{success} and
the \textbf{resistance} has \emph{one} \textbf{success} you have
\emph{zero} \textbf{degrees} and \textbf{victory} belongs to the high
roll.

A lot of the time your GM won't need to figure out the \textbf{degrees}
as knowing you won or lost is enough.

\hypertarget{narrating-outcomes}{%
\paragraph{2.3.7.2 Narrating Outcomes}\label{narrating-outcomes}}

Your GM narrates the contest \textbf{outcome}. Their narration should
take into account the \textbf{prize} and the \textbf{tactics} used by
each side. Your GM may invite you to contribute more detail on your
actions as part of that narration, if they wish. But the GM is the final
arbiter of how the story progresses as a result of the rolls - provided
they respect the \textbf{outcome} in which you win or lose the
\textbf{prize}.

Your GM should bear in mind your \textbf{result} when describing the
outcome. The \textbf{degree} is a guide for the GM when narrating the
outcome as to how convincing a victory was. If you have \emph{zero}
\textbf{degrees}, the GM should describe your actions as successful, but
the \textbf{resistance} as competent. If you have \emph{two}
\textbf{degrees}, your GM should describe a convincing \textbf{victory}
in which your adversary is clearly outclassed.

The GM is narrating a car chase through the busy streets of New Los
Angeles. The PCs are trying to catch the demon-worshipper Ath'Zul who
has stolen The Eye of Lorus from a museum. Some examples of how the GM
might interpret \textbf{outcomes} as follows:

\begin{itemize}
\tightlist
\item
  PC one \textbf{success} vs.~Ath'Zul one \textbf{success}, the PC has
  the higher roll, \emph{zero} \textbf{successes} difference, and
  \emph{zero} \textbf{degrees} of victory: Ath'Zul tries to shake the
  PCs, his hover bike, weaving in and out of traffic, but the PCs are
  always on his tail, and catch him at the lights on Bradbury Junction.
\item
  PC one \textbf{success} vs.~Ath'Zul zero \textbf{successes}, the PC
  has \emph{one} success difference and \emph{one} \textbf{degree} of
  \textbf{victory}: Ath'Zul tries to shake the PCs, his hover bike,
  weaving in and out of traffic, but the PCs force him off the road,
  where his bike loses repulsor lift and halts.
\item
  PC two \textbf{successes} vs.~Ath'Zul zero \textbf{successes}, the PC
  has \emph{two} successes difference \emph{two} \textbf{degrees} of
  \textbf{victory} : Ath'Zul tries to shake the PCs, his hover bike,
  weaving in and out of traffic, but he crashes into a parked car,
  spilling Ath'Zul and the stolen artefact over the road.
\end{itemize}

Your GM should avoid robbing your PC of competence by describing your
\textbf{defeat} as due to your incompetence when you may have rolled a
\textbf{success}.

The \textbf{degrees} may be more directly used when considering
\textbf{consequences} and \textbf{benefits} (see §2.8)

\hypertarget{confusing-ties}{%
\paragraph{2.3.7.3 Confusing Ties}\label{confusing-ties}}

Your GM will describe most tied \textbf{outcomes} as inconclusive
standoffs, in which neither of you gets what you wanted.

In some situations, ties become difficult to visualize. Chief among
these are \textbf{contest}s with binary \textbf{outcomes}, where only
two possible results are conceivable.

Your GM can either change the situation on such a tie, introducing a new
element that likely renders the original \textbf{prize} irrelevant to
both participants, or they can resolve the ties in your favor as a
\textbf{victory}.

\hypertarget{victory-at-a-price}{%
\paragraph{2.3.7.4 Victory at a Price}\label{victory-at-a-price}}

Your GM may treat \emph{zero} \textbf{degrees} of \textbf{victory} as
`victory at a price' and \emph{zero} \textbf{degree} of \textbf{defeat}
as `defeat with a boon'. The `price' is a cost that the victor pays for
obtaining the prize, a `boon' is something positive the loser takes
away. Your GM may ignore this option, and simply award you or deny you
the \textbf{prize}, if they cannot think of a dramatically interesting
reason to provide a `price' or `boon'. Your GM may use
\textbf{consequences} to represent a `price' and \textbf{benefits} to
represent a `boon', see §2.8. Your GM may also decide that the `price'
or `boon' is represented by the narration.

\begin{itemize}
\tightlist
\item
  PC one \textbf{success} vs.~Ath'Zul one \textbf{success}, the PC has
  the higher roll, \emph{zero} \textbf{successes} difference, and
  \emph{zero} \textbf{degree} of \textbf{victory}: Ath'Zul tries to
  shake the PCs, his hover bike, weaving in and out of traffic, but the
  PCs are always on his tail, and catch him at the lights on Bradbury
  Junction, \emph{by ramming their pursuit car into Ath'Zul's bike,
  damaging both vehicles}. The GM may award a \textbf{consequence} (see
  §2.8) to represent the damage to the PC's car, injuries from the
  crash, or displeasure from their commander for damaging more police
  property.
\end{itemize}

\hypertarget{one-outcome-for-the-pc}{%
\paragraph{2.3.7.5 One Outcome, For the
PC}\label{one-outcome-for-the-pc}}

There is one \textbf{outcome} to a \textbf{contest} and it always
applies to the PC. Your GM does not consider a separate outcome for the
\textbf{resistance}, instead they narrate the \textbf{outcome} based on
whether the your PC obtains the \textbf{prize} and describe how the
story branches for the \textbf{resistance} based on their interpretation
of that. Rules on \textbf{benefits} and \textbf{consequences} in §2.8
and narration considerations such as `victory at a price', only apply to
your PC, not to the resistance.

The only exception to this is a PC vs.~PC contest, where each sides
\textbf{outcome} needs to be determined.

\hypertarget{modifiers}{%
\subsection{2.4 Modifiers}\label{modifiers}}

Your GM uses \textbf{modifiers} throughout the game. \textbf{Modifiers}
increment by 5. See Table 2.4.1, Modifiers, for details of the scale.

A positive \textbf{modifier} is a \textbf{bonus} and a negative
\textbf{modifier} is a \textbf{penalty}.

\textbf{Modifiers} represent:

\begin{itemize}
\tightlist
\item
  The \textbf{modifier} to apply to the \textbf{base resistance} when
  determining the \textbf{resistance's} \textbf{TN}; harder challenges
  have a \textbf{bonus} added to the \textbf{base resistance}; easier
  challenges have \textbf{penalty} subtracted from the \textbf{base
  resistance} (see \$2.3.3).
\item
  A \textbf{stretch} based on a lack of credible abilities to overcome
  an obstacle (see §2.5)
\item
  A \textbf{situational modifier} based on the \textbf{tactics} chosen
  by a PC (see §2.5)
\item
  The \textbf{outcome} of an \textbf{augment} (see §2.6) or
  \textbf{hindrance} (see §2.7)
\item
  The \textbf{bonus} of a \textbf{benefit} or \textbf{penalty} of
  \textbf{consequences} (see §2.8).
\end{itemize}

\hypertarget{modifiers-scale-table}{%
\subsubsection{2.4.1 MODIFIERS SCALE
TABLE}\label{modifiers-scale-table}}

\begin{longtable}[]{@{}cc@{}}
\toprule
Penalty & Bonus \\
\midrule
\endhead
-20 & - \\
-15 & - \\
-10 & - \\
-5 & - \\
- & - \\
- & 5 \\
- & 10 \\
- & 15 \\
- & 20 \\
\bottomrule
\end{longtable}

\hypertarget{bonuses-and-penalties-from-tactics}{%
\subsection{2.5 Bonuses and Penalties from
Tactics}\label{bonuses-and-penalties-from-tactics}}

Your GM can give you \textbf{bonuses} and \textbf{penalties} to alter
your \textbf{target number} due to unusual circumstances you helped to
create, or have some control over. Commonly, your GM gives you a
\textbf{bonus} or \textbf{penalty} because the \textbf{tactic} that you
chose seems likely to give you an advantage, or a penalty. This will be
a \textbf{stretch} or a \textbf{situational modifier}. If an unusual
situation applies to a \textbf{resistance}, the GM should choose a
\textbf{resistance} that reflects that.

\hypertarget{stretches}{%
\subsubsection{2.5.1 Stretches}\label{stretches}}

When you propose an action using an \textbf{ability} that seems
completely inappropriate, your GM rules it impossible. If you went ahead
and tried it anyway, you'd automatically fail---but you won't, because
that would be silly.

In some cases, though, your proposed match-up of action and
\textbf{ability} is only somewhat implausible. A successful attempt with
it wouldn't completely break the illusion of fictional reality---just
stretch it a bit.

Using a somewhat implausible \textbf{ability} is known as a
\textbf{stretch}. If your GM deems an attempt to be a \textbf{stretch},
the PC suffers a -5 or --10 \textbf{penalty} to their \textbf{target
number}, depending on how incredible the \textbf{stretch} seems to the
GM and other players. Your GM should \textbf{penalize} players who try
to create a `do anything' \textbf{ability} that they then
\textbf{stretch} to gain from raising fewer \textbf{abilities} in
advancement to ensure balance with other PCs.

The definition of \textbf{stretch} is elastic, depending on genre.

Your GM should not impose \textbf{stretch penalties} on action
descriptions that add flavor and variety to a scene, but do not
fundamentally change what you can do with your \textbf{ability}. These
make the scene more fun but don't really gain any advantage.

\hypertarget{situational-modifiers}{%
\subsubsection{2.5.2 Situational
Modifiers}\label{situational-modifiers}}

Your GM may also impose \textbf{situational modifiers} when, given the
description of the current situation, believability demands that you
should face a notable \textbf{bonus} or \textbf{penalty}. Your GM should
use the scale in §2.3 Modifiers when determining a \textbf{bonus} or
\textbf{penalty}; most modifiers should +/5 or +/-10.. \textbf{Bonuses}
and \textbf{penalties} of less than 5 don't exert enough effect to be
worth the bother. Those higher than 10 give the \textbf{situational
modifier} a disproportionate role in determining \textbf{outcomes}.

During a \textbf{sequence} (see §5.0), they should typically last for a
single \textbf{round}, and reflect clever or foolish choices.

\hypertarget{augments}{%
\subsection{2.6 Augments}\label{augments}}

You may sometimes face \textbf{contests} where more than one
\textbf{ability} may be applicable to the conflict at hand. In such
cases, you may attempt to use one \textbf{ability} to give a supporting
bonus to the main ability you are using to frame the \textbf{contest}.
This is called an \textbf{augment}. It results in a \textbf{bonus} to
your \textbf{target number}. For example, if your character has the
\textbf{abilities} ``The Queen's Intelligencer'' and ``Master of
Disguise'', you might use the latter to \textbf{augment} the former when
infiltrating a rival nation's capitol. Similarly, a character with
``Knight Errant'' and ``My Word is my Bond'' \textbf{abilities} might
use one to \textbf{augment} the other when in conflict with a
\textbf{story obstacle} the character has sworn to overcome.

Abilities that represent special items, weapons, armor, or other
noteworthy equipment can be a common source of \textbf{augments}.
However, this grows tired if over-used and you should try and restrict
repeated use of equipment in this way to \textbf{contests} where they
are particularly interesting or apropos.

\textbf{Augments} can also come from other characters'
\textbf{abilities} if one character uses an \textbf{ability} to support
another's efforts rather than directly engaging in the \textbf{contest}.
\textbf{Augments} can even come from outside resources like support from
a community, see §8, or other circumstantial help.

If you have a good idea for an \textbf{augment}, propose it to your GM
while the \textbf{contest} is being framed. When making your proposal,
describe how the \textbf{augmenting ability} supports the main one in a
way that is both \emph{entertaining} and \emph{memorable}. Don't just
hunt for mechanical advantage, show your group more about your PC when
you \textbf{augment}, their attitudes, passions, or lesser known
\textbf{abilities}. If you are \textbf{augmenting} with a \textbf{broad
ability} like ``Fool's Luck'', be prepared to describe the unlikely
events that tilt the scales in your favor. Your GM will decide whether
the \textbf{augment} is justified and can refuse boring and uninspired
attempts to \textbf{augment}, where you are just looking for a bonus to
your roll and not adding to the story.

You may only use one of your own \textbf{abilities} to \textbf{augment}
the \textbf{ability} you are using in the \textbf{contest}, and you may
not use an \textbf{ability} to \textbf{augment} itself. You may not use
a \textbf{breakout} to augment it's parent \textbf{keyword} (see
§2.1.1), or another \textbf{breakout} from the parent \textbf{keyword}.
Another player character may also augment you, however,
\textbf{augments} from other player characters supporting you, only give
you \emph{one} \textbf{bonus} to add to your \textbf{target number},
regardless of the number of supporters you have.

Your GM should bear in mind the credibility of more than one PC helping
you. When persuading someone a cacophony of voices may not help, unless
you are trying to intimidate; when fighting someone, only so much backup
helps you take your opponent down; when flying a starship into the cave
on the asteroid, only some crew activities provide credible help. The GM
may thus decide to limit the number of augments from other PCs. Consider
a group contest (see §4.2) instead if many PCs want to act against the
\textbf{resistance}.

If your GM accepts your \textbf{augment} proposal, it will be resolved
by the method below. The main \textbf{contest} then proceeds as normal,
with any bonus from the \textbf{augment} added onto the \textbf{rating}
of the \textbf{ability} chosen when \textbf{framing the contest}. The
\textbf{augment} remains in effect for the duration of the
\textbf{contest}.

\hypertarget{augment-procedure}{%
\subsubsection{2.6.1 Augment Procedure}\label{augment-procedure}}

Your GM treats an \textbf{augment} as an \textbf{assured contest}.

If the use of the \textbf{ability} to augment seems unlikely to fail,
your GM simply awards you a \textbf{bonus}, or +5. If your description
of how you were using the \textbf{augmenting ability} was dramatic or
entertaining, your GM may increase this to a \textbf{bonus} of +10.

As with any \textbf{assured contest} GM might still ask you to roll if
there is a risk that the \textbf{augment} results in a \textbf{penalty}
to other \textbf{abilities} such as resources or \textbf{relationships}
(see §6.0), which become stressed in providing the \textbf{augment}, or
that more variation in \textbf{bonuses} is possible (such as the
\textbf{augmenting} ability having several \textbf{masteries}).

On a \textbf{victory} base the \textbf{bonus} of the \textbf{augment}
off the \textbf{degree} of the \textbf{victory} (see §2.3). So
\emph{zero} \textbf{degrees} of \textbf{victory} yields +5 bonus. On a
\textbf{defeat} still award a +5 bonus, but apply a \textbf{penalty}
related to \textbf{degree} of the \textbf{defeat} to the \textbf{tactic}
used to \textbf{augment}, as described above.

\hypertarget{triggering-flaws}{%
\subsection{2.7 Triggering Flaws}\label{triggering-flaws}}

During play, either you, or your GM, may decide that your \textbf{flaw}
has been triggered. A \textbf{flaw} might apply to the \textbf{tactic}
you are using in upcoming \textbf{contest}, when it is called a
\textbf{hindrance} (see §2.7.1). Alternatively a \textbf{flaw} might
simply come into play when you want to describe your PC acting in a
certain way, and you or your GM feels that one of your \textbf{flaws}
could prevent this, or you, or your GM, feel that a situation raises a
challenge that means one of your \textbf{flaws} would lead to you
responding in a certain way.

\hypertarget{hindrance}{%
\subsubsection{2.7.1 Hindrance}\label{hindrance}}

If you describe a \textbf{tactic} that is in conflict with a
\textbf{flaw}, your GM may decide to impose a \textbf{penalty} called a
\textbf{hindrance} against you in the upcoming \textbf{contest}. You
should choose to remind your GM when you feel a flaw might be triggered.
Your GM may also use an \textbf{ability} on your character sheet against
you in this way too, if appropriate. This may be the case for
relationships you have, philosophies you espouse, or groups you belong
to.

Your GM should follow a similar approach to \textbf{augments} when
applying a \textbf{hindrance}. They should ask themselves if it is
\emph{fresh}, \emph{interesting} or \emph{illuminates character}. In a
movie of book would your \textbf{flaw} be prominent here?

If your GM feels that there is no uncertainty as to whether the
\textbf{flaw} applies to your \textbf{tactic} in the contest they apply
a \textbf{penalty} of -5 or a \textbf{penalty} of -10 depending on how
serious a handicap the \textbf{flaw} is. (In effect an \textbf{assured
contest} for the flaw).

If your GM feels that it is uncertain as to whether the \textbf{flaw}
hinders you, or you are able to overcome it, and you agree that you wish
to try, treat it as a \textbf{contest}. Roll the rating of your
\textbf{flaw} against the \textbf{base resistance}. On a
\textbf{victory}, you receive a \textbf{penalty} from the
\textbf{degree} of your \textbf{victory} (see §2.3). For example, if you
get \emph{zero} \textbf{successes} use a \textbf{penalty} of -5, on
\emph{one} \textbf{success}, use a \textbf{penalty} of -10 and so on. On
a \textbf{defeat}, you overcome the \textbf{flaw}.

When you experience a \textbf{penalty} due to a flaw, you gain an
\textbf{experience point} (see §8.1).

\hypertarget{act-according-to-your-flaw}{%
\subsubsection{2.7.2 Act according to your
flaw}\label{act-according-to-your-flaw}}

At times the direction of the story you are all telling may place your
PC in situations when it seems likely they would act according to their
\textbf{flaw}. The addict may reach for drink or drugs following an
emotional setback, a lust for vengeance may come between your PC and
showing mercy, prejudices or bigotry may prevent you from seeing others
positively.

If you chose to act according to your \textbf{flaw} there is no contest,
simply describe your character behaving as the \textbf{flaw} dictates.
This might result in a \textbf{hindrance} to further actions (see
§2.6.1)

If you wish to act against your \textbf{flaw}, your \textbf{tactic} must
pass a \textbf{credibility test} as to how you try overcome your
\textbf{flaw} in this instance. In effect, pick an \textbf{ability} to
resist the \textbf{flaw} with. Then you must obtain a \textbf{victory}
in a \textbf{contest} against your \textbf{flaw}. On a \textbf{victory}
you may act in a way that contradicts your \textbf{flaw}.

If you submit to your \textbf{flaw}, your GM might impose a
\textbf{hindrance} on further actions (see §2.6.1). You should not
contest this \textbf{hindrance} unless the situation is not related to
the one which triggered your \textbf{flaw} in this instance, or
significant time has now passed.

Your GM may impose a \textbf{penalty} against an \textbf{ability} if you
gain the \textbf{victory} against your \textbf{flaw} representing your
struggle against your inner nature, violating dearly held principles, or
letting down dependents. This is often true where the GM invokes a flaw
from a \textbf{keyword}. For example, if you had they \textbf{keyword}
``Gangster'' and decide to inform on a fellow mobster, your GM might
invoke the \textbf{flaw} of ``Code of Silence'' even if it is not a
\textbf{breakout} under you \textbf{keyword}; this is particularly
appropriate where facts such as the ``Code of Silence'' have been
established in game. Even if you overcome your \textbf{flaw}, and inform
on your fellow mobster, the GM might still impose a \textbf{penalty} on
use of the \textbf{keyword} to interact with your crime family for
having breached the ``Code of Silence.''

Similarly, your GM might give you a \textbf{bonus} for acting according
to your \textbf{flaw}, representing the sacrifices you have made for
dependents or a temporary boost from satisfying your inner demons. For
example, if your superhero ``Speedster'' goes to see the premiere of his
partner's new play, instead of heading to the docks to stop Dr.~Squid's
shipment of Vibrium, your GM might award you a \textbf{bonus} to your
relationship to your partner.

If you choose to, or your GM compels you to, act according to a
\textbf{flaw}, you gain an \textbf{experience point} (see §8.1).

\hypertarget{benefits-and-consequences}{%
\subsection{2.8 Benefits and
Consequences}\label{benefits-and-consequences}}

\textbf{Contests}, in addition to deciding whether you overcome a
\textbf{story obstacle} or answer a \textbf{story question}, gaining the
\textbf{prize}, may carry additional \textbf{consequences} or
\textbf{benefits} related to the PC's \textbf{outcome}.

Your GM may simply determine narrative \textbf{consequences} and
\textbf{benefits} from what makes fictional sense, given the agreed
\textbf{prize} for the \textbf{contest}, as described above. Optionally,
your GM may assign you ongoing \textbf{bonuses} and \textbf{penalties}
that may affect related future \textbf{contests}, related to the
\textbf{outcome} of this \textbf{contest}. Your GM should always respond
to the flow of the story, if narrative \textbf{consequences} are enough,
they should not reach for additional mechanical \textbf{bonuses} or
\textbf{penalties}. Your GM should use mechanical \textbf{bonuses} or
\textbf{penalties} where it strains credibility that there is no ongoing
\textbf{consequence} or \textbf{benefit} from the \textbf{outcome} of
the \textbf{contest}.

In a fight, it may strain credibility that a \textbf{defeat} does not
leave you impaired for further physical activity. In a display or
oratory before the assembled townsfolk, it may strain credibility if
they would not later act according to your rousing words. In a romance,
it may strain credibility if the wonderful date night does not improve
your chances of taking your relationship to the next level.

\hypertarget{consequences}{%
\subsubsection{2.8.1 Consequences}\label{consequences}}

After a \textbf{contest}, you may suffer \textbf{consequences}: literal
or metaphorical injuries.

\begin{itemize}
\tightlist
\item
  In a fight or test of physical mettle, you wind up literally wounded.
\item
  In a social contest, you suffer damage to your reputation.
\item
  If commanding a war, you lose battalions, equipment, or territories.
\item
  In an economic struggle, you lose money, other resources, or
  opportunities.
\item
  In a morale crisis, you may suffer bouts of crippling self-doubt.
\end{itemize}

Your GM may assign a \textbf{penalty} to reflect this
\textbf{consequence}. The \textbf{penalty} will depend on how severe
they feel the \textbf{consequences} are. If your opponent defeats you,
your GM can use the \textbf{degree} of the \textbf{outcome} as the
\textbf{penalty}(see §2.3). Whilst this is a good `rule of thumb' a GM
can use their discretion as to the story needs and assign a different
\textbf{penalty}.

If you \textbf{defeat} your opponent, your GM may still decide that you
suffer a \textbf{consequence}, representing fatigue, exhaustion,
disapproval or other expenditure of resources on earning the
\textbf{prize}.

\begin{itemize}
\tightlist
\item
  In a fight, you are left bruised and battered.
\item
  In a social contest, you sacrifice the trust of a marginalized group.
\item
  If commanding a war, you must sacrifice some of your forces for
  victory.
\item
  In an economic struggle, you take significant losses to win market
  share.
\item
  In a morale crisis, your resolve alienates the cowardly.
\end{itemize}

If you have \emph{zero} \textbf{degrees} of \textbf{victory} your GM
might assign a \textbf{penalty} of -5 or -10 to represent effort
expended in the \textbf{victory}. If you have \emph{one} \textbf{degree}
of \textbf{victory}, your GM might assign a \textbf{penalty} of -5 , for
similar reasons, if it makes dramatic sense.

\hypertarget{ending-a-pcs-story}{%
\paragraph{2.8.1.2 Ending a PC's story}\label{ending-a-pcs-story}}

Your GM should not impose a narrative \textbf{consequence} on your PC
that takes them permanently out of the game, such as by death, without
discussion. Some games allow characters to be taken out of the story by
the result of a dice roll, but QuestWorlds is a co-operative
storytelling game where a failed dice roll should not automatically
remove a character from play. However, you, or the GM, might feel that
your PC's story has come to an end with this failure, and you can
consent to that \textbf{outcome}. Usually, your GM should refrain from
suggesting this option unless the story itself suggests it.

A story-ending \textbf{outcome} may not just be death. It can include
anything that takes the PC out of play, such as exile, dismissal from
the secret agency, a broken heart. In some cases the ending to your PCs
story could be ambiguous, allowing the PC to return at a future point
when the story makes their salvation possible.

Your GM must declare that the stakes of a particular \textbf{contest}
place a PC at risk of this being a story ending moment, before the dice
are rolled. This may be important for credibility in the story that the
group is telling. In this case there should be an option for the PC to
avoid, or backdown from a \textbf{contest}, that has a risk of ending
their story. You should usually use a \textbf{sequence} for any conflict
where a PC's continuation in the story is at stake. This should be a
dramatic moment, truly worth focusing at a task level on, not rolled up
into conflict resolution by a \textbf{contest}.

\hypertarget{benefits}{%
\subsubsection{2.8.2 Benefits}\label{benefits}}

Just as when you can experience ongoing ill effects from a
\textbf{contest}, you can gain ongoing benefits from a \textbf{contest}.

\begin{itemize}
\tightlist
\item
  In a fight or test of physical mettle, surging adrenaline leaves you
  sharp for the next encounter.
\item
  In a social contest, you gain confidence and admiration from your
  triumph.
\item
  If commanding a war, you gain strategic advantage over your enemy.
\item
  In an economic struggle, your profits can be re-invested, or you drive
  competitors into the ground.
\item
  In a morale crisis, you are buoyed up by success, nothing can stop you
  now.
\end{itemize}

Remember that the \textbf{benefit} does not have to be directly related
to the \textbf{ability} used. Look to the goal of the \textbf{contest}.
The abilities or situation should reflect the \textbf{story obstacle}
that was overcome, \textbf{story question} that was answered, or the
\textbf{tactic} used to overcome it.

\begin{itemize}
\tightlist
\item
  In a fight or test of physical mettle, your triumph has everyone
  rallying to your cause.
\item
  In a social contest, you win powerful \textbf{allies} who will
  strengthen you in your fight against your enemies.
\item
  If commanding a war, you pillage the enemy city and enrich your army.
\item
  In an economic struggle, you gain status as one of the wealthy elite.
\item
  In a morale crisis, your rallied troops strengthen your army.
\end{itemize}

Your GM may assign a \textbf{bonus} to reflect this \textbf{benefit}. If
you win the \textbf{prize}, your GM may choose to use the
\textbf{degree} of your \textbf{outcome} to determine the \textbf{bonus}
(see §2.3). Whilst this is a good `rule of thumb' a GM can use their
discretion as to the story needs and assign a different \textbf{bonus}.

If you lost the \textbf{prize}, your GM may still decide that you gain a
\textbf{benefit}, representing learning, gratitude, or resolve developed
from losing the \textbf{prize}.

\begin{itemize}
\tightlist
\item
  In a fight or test of physical mettle, you learn your opponent's
  weaknesses.
\item
  In a social contest, many feel sympathy for you though they cannot
  support you.
\item
  If commanding a war, you win the trust of your soldiers through shared
  suffering.
\item
  In an economic struggle, your organization becomes leaner and fitter.
\item
  In a morale crisis, you reflect on your failure and gain new inner
  strength.
\end{itemize}

On \emph{zero} \textbf{degrees} of \textbf{defeat} your GM might assign
a \textbf{bonus} of +5 or +10 to represent a glimmer of hope for the PC
despite the \textbf{defeat}, such as gaining an insight into the
\textbf{resistance's} weakness. If you have \emph{one} \textbf{degree}
of \textbf{defeat} your GM may assign a \textbf{bonus} of +10, for
similar reasons, if it makes dramatic sense.

\hypertarget{recovery}{%
\subsubsection{2.8.3 Recovery}\label{recovery}}

\textbf{Consequences} lapse on their own with the passage of time. Your
GM will determine when the \textbf{consequences} have faded, and you
should ask about whether they still apply at each new game session. The
worse the \textbf{consequence}, the longer it may last, though the GM
may reduce the \textbf{penalty} in increments of -5 as you recover,
reflecting the passage of time. However, you'll often want to remove
them ahead of schedule, with the use of \textbf{abilities}.

\hypertarget{recovery-abilities}{%
\paragraph{2.8.3.1 Recovery Abilities}\label{recovery-abilities}}

When deciding the \textbf{tactic} to use for recovery, the
\textbf{ability} used to bring about recovery from a
\textbf{consequence} must relate to the type of \textbf{consequence}.

\begin{itemize}
\tightlist
\item
  You can aid recovery from physical injuries with medical
  \textbf{abilities} or incredible healing \textbf{abilities} such as
  magic, regenerative powers, or super-science.
\item
  You can remove mental traumas, including those of confidence and
  morale, with mundane psychology \textbf{abilities} or through
  \textbf{incredible abilities} such as telepathy. You might also remove
  them through a dramatic confrontation between the victim and the
  source of the psychic injury.
\item
  You can remove social injury through social \textbf{abilities} or
  incredible social \textbf{abilities} such as charm spells, love
  potions, or mind control. You probably have to make a public apology
  of some sort, often including a negotiation with the offended parties
  and the payment of compensation, either in disposable wealth or
  something more symbolic.
\item
  You can fix damage to items and equipment with some sort of repair
  \textbf{ability}. If you want to fix an incredible item, you may
  require genre-specific expertise: a broken magic ring may require a
  ritual to reforge.
\end{itemize}

\hypertarget{recovery-resistances}{%
\paragraph{2.8.3.2 Recovery Resistances}\label{recovery-resistances}}

The \textbf{resistances} to remove a states of adversity is the
\textbf{base resistance} modified by the \textbf{bonus} equal and
opposite to the \textbf{penalty}. So if you were suffering from a
penalty of -10, you modify the \textbf{base resistance} +10.

\hypertarget{recovery-contests-and-sequences}{%
\paragraph{2.8.3.3 Recovery Contests and
Sequences}\label{recovery-contests-and-sequences}}

Your GM should almost always resolve healing attempts as
\textbf{contests}. An exception might be a medical drama, in which
surgeries would comprise the suspenseful set-piece sequences of the
game, and your GM might chose a \textbf{sequence}. If the source of
recovery is an NPC, the contest should always relate to the
\textbf{tactic} used to obtain the services of the NPC.

Any \textbf{victory} on a recovery attempt, clears the
\textbf{consequence} (and so remove any \textbf{penalty}). On a
\emph{zero} \textbf{degree} \textbf{victory} your GM might choose to
apply a \textbf{consequence} to the \textbf{ability} used to recover
representing exhaustion (of supplies, energy, etc.).

\hypertarget{waning-benefits}{%
\subsubsection{2.8.4 Waning Benefits}\label{waning-benefits}}

Just as you recover from \textbf{consequences} with time, or through
recovery, so \textbf{benefits} fade with time.

At the beginning of a session, especially when a significant period of
game-world time passes between the conclusion of one session and the
beginning of the next, the GM may declare that all \textbf{benefits}
have expired or waned. A waning benefit may reduce its \textbf{bonus}
with time, as the effect fades. You are no longer charged with the
confidence of your recent victory, the fans have forgotten your last
concert, or the people of the village have started to think once again
about the day-to-day struggle of their lives not how the stranger helped
them. Benefits always decline in increments of 5 (§ see 2.3).

An expired benefit no longer gives you a \textbf{bonus}, your past
victories no longer bring you solace, your fickle fans have moved on to
the latest sensation.

\hypertarget{multiple-benefits-and-consequences}{%
\subsubsection{2.8.5 Multiple Benefits And
Consequences}\label{multiple-benefits-and-consequences}}

A PC may apply \textbf{bonuses} from multiple \textbf{benefits} to a
single \textbf{contest}, or apply \textbf{penalties} from multiple
\textbf{consequences} to a single \textbf{contest}. \textbf{Benefits and
consequences} may cancel each other out.

Because it is confusing to track both \textbf{benefits and consequences}
against the same \textbf{ability} your GM may simply rule that one
cancels the other out. This is particularly true of social
\textbf{contests} where a moment of shame can erase your previous
triumphs, or your confidence eroded by a \textbf{failure}. Physical
benefits may cancel out, flushed with victory you may be able to ignore
pain, but it may defy credibility for wounds to be healed by an athletic
performance.

Your GM may simply rule that \textbf{benefits} and \textbf{consequences}
cancel out, or they may take the difference between the two benefits and
create a new one. For example if you have a +10 \textbf{benefit} from
impressing the crowd with your previous performance in the dance
\textbf{contest}, but then suffer an injured ankle with a
\textbf{consequence} of -5, your GM may rule that your twisted ankle
cancels out your energy from the last performance, or your GM might rule
that your success sees you through the pain, but you are now only +5
\textbf{benefit} to impress the crowd.

\hypertarget{benefits-and-consequences-by-degree}{%
\subsubsection{2.8.6 BENEFITS AND CONSEQUENCES BY
DEGREE}\label{benefits-and-consequences-by-degree}}

\begin{longtable}[]{@{}cccc@{}}
\toprule
Degree of Victory & Degree of Defeat & Penalty & Bonus \\
\midrule
\endhead
- & 3 & -20 & - \\
- & 2 & -15 & - \\
- & 1 & -10 & - \\
- & 0 & -5 & - \\
(Tie) & (Tie) & - & - \\
0 & - & - & 5 \\
1 & - & - & 10 \\
2 & - & - & 15 \\
3 & - & - & 20 \\
\bottomrule
\end{longtable}

Although your GM is at liberty to assign any \textbf{bonus} or
\textbf{penalty} they believe is credible, this table offers suggested
\textbf{modifiers} for different \textbf{degrees of victory and defeat}.

\hypertarget{mismatched-and-graduated-goals}{%
\subsection{2.9 Mismatched and Graduated
Goals}\label{mismatched-and-graduated-goals}}

Sometimes, the two sides in a \textbf{contest} may have goals that do
not directly conflict one another. A huntsman pursues a nurse, who is
trying to escape through the forest with two small children. The
huntsman wants to capture the nurse. The nurse wants to save the
children.

When encountering \textbf{mismatched goals}, your GM should determine
whether the mismatch is complete, or partial.

In a \textbf{complete mismatch}, neither side is at all interested in
preventing the other's goal. A \textbf{complete mismatch} does not end
in a \textbf{contest}; your GM asks what you are doing, and then
describes each participant succeeding at their goals.

In most instances, the \textbf{contest} goals are not actually
\textbf{mismatched}, but \textbf{graduated}. You have both a
\textbf{primary} and a \textbf{secondary} goal. In this case, your GM
frames the \textbf{contest}, identifying which goal is which. It is
possible to have \textbf{tertiary} goals and so on, but avoid needless
sub-division.

On a \textbf{victory} you choose one more \textbf{graduated} goals than
you have \textbf{degrees}. On \emph{zero} \textbf{degrees} of
\textbf{victory} you choose one; on \emph{one} \textbf{degree} of
victory you choose two, and so on. Normally, the GM should give the
player the choice of which goals they wish to choose. This goes to the
heart of character - what is more important to you?

On a \textbf{defeat} you normally get nothing, but when there are
\textbf{graduated} goals on \emph{zero} \textbf{degrees} of defeat, your
GM may choose to pick one of your goals to give you. The GM may choose
to go against the PC's likely preference here - the PC doesn't get the
choice they would have made given they had to sacrifice something.

The nurse has \textbf{graduated} goals: escape the huntsman and save the
children. On a \emph{zero} \textbf{degree} of \textbf{victory} she will
have to decide between capture and the safety of the children. On
\emph{one} \textbf{degrees} of \textbf{victory}, she can have both.

\hypertarget{mobs-gangs-and-hordes}{%
\subsection{2.10 Mobs, Gangs, and Hordes}\label{mobs-gangs-and-hordes}}

Sometimes you will be outnumbered by your opponents. Your GM can treat
many as one. Your GM treats a crowd as a single \textbf{resistance} with
one \textbf{rating}. When selecting a \textbf{resistance} your GM should
factor their numbers into the \textbf{rating}. Numbers are not always an
advantage. Whilst in a fight, numbers give you a significant advantage -
provided the mob has room to maneuver - in an attempt to win `hearts and
minds' too many voices can be counterproductive, unless you are trying
to intimidate. The final decision on the amount of help provided by
numerical advantage rests with your GM.

In a sequence, when the mob loses an exchange, your GM describes
individuals within it as being hurt or falling away. When it wins, they
describe them overwhelming you.

\hypertarget{ganging-up}{%
\subsection{2.11 Ganging Up}\label{ganging-up}}

Sometimes you may outnumber your opponent. As above, the GM should alter
the \textbf{resistance} depending on how significantly you outnumber
them and depending on whether numbers provide advantage.

\hypertarget{mass-effort}{%
\subsection{2.12 Mass Effort}\label{mass-effort}}

Clashes of massive forces resolve like any other \textbf{contest} or
\textbf{sequence}. These include:

\begin{itemize}
\tightlist
\item
  Military engagements
\item
  Corporate struggles for market share
\item
  Building competitions
\item
  Efforts to spread a faith or ideology
\item
  Dance competitions
\end{itemize}

If you are not participating in the \textbf{contest} and have no stake
in its \textbf{outcome}, then your GM doesn't bother to run a
\textbf{contest}. The GM just chooses an \textbf{outcome} for dramatic
purposes.

Otherwise, your GM will start by determining your degree of influence
over the \textbf{outcome}. They are either:

\begin{itemize}
\tightlist
\item
  \textbf{Directing}: The success of the effort depends mostly on your
  leadership. For example, you might be a military leader facing a force
  of roughly equal potency. As all else is equal, the better general
  will win the day. In this instance, if you are in command, your
  \textbf{tactic} should be a relevant leadership \textbf{ability}.
\item
  \textbf{Contributors}: The outcome of the battle hangs in the balance
  and your efforts may tip the balance in favor of your side. You roll
  against a \textbf{resistance} determined by the GM, either in a
  \textbf{contest} (see §4.0) or \textbf{sequence} (see §5.0)
  representing the odds of your shifting the odds in your favor. If a PC
  leads the effort your GM should award them a \textbf{bonus} or
  \textbf{penalty} reflecting the \textbf{outcome} of the contest. They
  should then roll as described in Directing above. If an NPC leads the
  effort the \textbf{outcome} is that of the \textbf{contest} or
  \textbf{sequence}, and the PCs effort is assumed to tip the scale
  either way.
\item
  \textbf{Participants}: You have little influence over the
  \textbf{outcome}, but must struggle to prosper within the conflict.
  The GM predetermines the \textbf{outcome} of the overall competition
  on dramatic grounds. To determine your fate in the battle, you roll
  against a \textbf{resistance} determined by the GM from the overall
  battle \textbf{outcome}, in a \textbf{contest} (see §4.0) or
  \textbf{sequence} (see §5.0).
\end{itemize}

\hypertarget{resistance-progression}{%
\subsection{2.13 Resistance Progression}\label{resistance-progression}}

Your GM may decide that \textbf{resistance} to your actions gets harder,
as the campaign progresses. This reflects the trope of the type of
challenges you face getting tougher as you improve.

Your GM should adopt a strategy that mimics a TV show where the
\textbf{resistance} does not increase during a season of the show,
allowing our protagonists to get more competent as the show progresses
towards its climax. In the next season though the \textbf{resistance}
usually goes up, and the writers reflect this with more challenging
opposition in the new season of the show. At the same time, the
opposition that was tough in the first season, now become mooks that can
be easily dispatched to show the increased competence of the
protagonists.

In that case your GM should increment the \textbf{base resistance} by +5
to +10 for the next campaign you play with the same characters. The size
of the change should reflect the increase in your previous
\textbf{abilities} in the last campaign. For example, if in the last
season you increased your \textbf{occupation keyword} by +10, your GM
may decide to increase the \textbf{resistance} by +5 or +10 to reflect
the more challenging opposition in the new campaign. The GM should
consider triggering \textbf{resistance progression} when your PCs find
it difficult to earn \textbf{experience points} because they too
regularly outclass even the climatic encounters (the boss monsters) of
their game.

Your GM should also take into account that the opposition you were
improving with respect to the previous season should now be considered
more-easily defeated mooks, and use lower \textbf{ratings} for them when
they appear in the story or even allow them to be taken out with an
\textbf{assured contest}.

\hypertarget{no-progression}{%
\subsubsection{2.13.1 No Progression}\label{no-progression}}

Your GM may also decide that the \textbf{resistances} do not get harder
as the campaign progresses, reflecting the PCs \textbf{ability} to
disregard minor challenges, and simply choose harder
\textbf{resistances} to challenge the players and allow them to earn
\textbf{experience points}.

\hypertarget{character-creation}{%
\section{3.0 Character Creation}\label{character-creation}}

The first step in creating your character is to come up with a concept
that fits in with the genre of the game that your GM intends to run.
With that, you can assign \textbf{abilities}, \textbf{ratings} for those
\textbf{abilities}, and if required \textbf{flaws}.

In addition, you will want to give your character a name, and provide a
physical description. We recommend focusing on three physical things
about your PC that others would immediately notice, over anything more
detailed.

Your GM should not use this method for creating NPCs. NPCs do not
require definition via \textbf{abilities} and \textbf{keywords}.
Instead, your GM simply describes the NPC, and picks an appropriate
\textbf{resistance} in any contest with them, based on their feeling for
what would be \textbf{credible} for that NPC. If in doubt the GM just
uses the \textbf{base resistance} for a mook, with a suitably higher
\textbf{resistance} for a boss. The design intent is to remove the need
for the GM to prepare stat blocks, making improvisation of NPCs easier,
and shifting focus to the NPCs personality or role in the story instead.

\hypertarget{as-you-go-method}{%
\subsection{3.1 As-You-Go Method}\label{as-you-go-method}}

\begin{enumerate}
\def\labelenumi{\arabic{enumi}.}
\tightlist
\item
  Choose a \textbf{concept}. Your \textbf{concept} is a brief phrase,
  often just a couple of words that tells the GM and other players what
  you do and how you act. Start with a noun or phrase indicating your
  \textbf{occupation keyword} or area of expertise, and modify it with
  an adjective suggesting a \textbf{distinguishing characteristic}, a
  personality trait that defines you in broad strokes:
\end{enumerate}

\hypertarget{example-concepts}{%
\subsubsection{3.1.1 EXAMPLE CONCEPTS}\label{example-concepts}}

\begin{longtable}[]{@{}cl@{}}
\toprule
Distinguishing Characteristic & Occupation \\
\midrule
\endhead
Haughty & Priestess \\
Hotshot & Lawyer \\
Noble & Samurai \\
Remorseful & Assassin \\
Sardonic & Ex-mercenary \\
Slothful & Vampire \\
Naive & Warrior \\
\bottomrule
\end{longtable}

\begin{enumerate}
\def\labelenumi{\arabic{enumi}.}
\setcounter{enumi}{1}
\tightlist
\item
  Provide your character with a name.
\item
  You have 12 \textbf{ability} slots and up to three slots for
  \textbf{flaws}.
\item
  Your character concept uses two slots: one for their occupational
  \textbf{keyword} and one for their \textbf{distinguishing
  characteristic}.
\item
  The first time you use an \textbf{ability} (including your
  \textbf{distinguishing characteristic} and \textbf{occupational
  keyword}), assign a \textbf{rating} to it (see §3.4).
\item
  If the series uses other \textbf{keywords}, such as those for culture
  or religion, you will need to spend an \textbf{ability} slots on
  those.
\item
  When events in the story put you in a situation where you want to
  overcome a \textbf{story obstacle}, or discover the answer to a
  \textbf{story question}, make up an applicable \textbf{ability} on the
  spot. This may be a \textbf{breakout ability} from a \textbf{keyword}.
  You are restricted to only one \textbf{sidekick}.
\item
  If you want, describe any \textbf{flaws}.
\item
  When you have used all your \textbf{ability} slots, you must use
  \textbf{experience points} to buy more. Once you have used all of your
  \textbf{flaw} slots, you must buy off an existing \textbf{flaw} with
  \textbf{experience points} to gain more.
\end{enumerate}

\hypertarget{keywords-1}{%
\subsection{3.2 Keywords}\label{keywords-1}}

We recommend that you build your PC around one or more \textbf{keywords}
(see §2.1.1).

\textbf{Keywords} are best suited for use as the PC's occupation,
heritage, beliefs, or memberships of a community.

In certain genres, your GM may require that your PC has multiple
\textbf{keywords}: for example, one for \textbf{occupation}, another for
species or culture, and perhaps a third for religious affiliation.

\hypertarget{doubling-up}{%
\subsubsection{3.2.2 Doubling Up}\label{doubling-up}}

In some settings, an \textbf{ability} may be listed in more than one of
a PC's \textbf{keywords}. You should choose only one to list it under.

If your \textbf{distinguishing characteristic} is an \textbf{ability}
that fits under a \textbf{keyword} then you can make it a breakout
there.

\hypertarget{flaws-1}{%
\subsection{3.3 Flaws}\label{flaws-1}}

You may assign up to three \textbf{flaws} to their PC. Common flaws
include:

\begin{itemize}
\tightlist
\item
  Personality traits: surly, petty, compulsive.
\item
  Physical challenges: blindness, lameness, diabetes.
\item
  Social hurdles: outcast, ill-mannered, hated by United supporters.
\end{itemize}

Certain \textbf{keywords} include \textbf{flaws}. \textbf{Flaws} gained
through \textbf{keywords} do not count against the limit of three chosen
\textbf{flaws}.

\hypertarget{assigning-ability-ratings}{%
\subsection{3.4 Assigning Ability
Ratings}\label{assigning-ability-ratings}}

You have now defined your \textbf{abilities} and \textbf{flaws}. These
tell everyone what you can do.

Now assign numbers to each \textbf{ability}, called \textbf{ratings},
which determine how well you can do these things.

Assign a starting \textbf{rating} of 15 to the \textbf{ability} you find
most important or defining. Although most players consider it wisest to
assign this \textbf{rating} to their \textbf{occupational keyword}, you
don't have to do this. Assign a \textbf{rating} of 15 to your
\textbf{distinguishing characteristic}.

All other \textbf{abilities} start at a \textbf{rating} of 10.

A \textbf{breakout} from a \textbf{keyword} starts at +5. In some cases,
you may treat your \textbf{distinguishing characteristic} as a
\textbf{breakout ability} from a \textbf{keyword} in this case, treat it
as a +10.

\textbf{Flaws} are assigned a \textbf{rating} equivalent to your
\textbf{abilities}. The first \textbf{flaw} is rated at the highest
\textbf{ability}, the second shares the same \textbf{rating} as the
second-highest \textbf{ability}, and the third equals the lowest
\textbf{ability}. As \textbf{abilities} improve, so the \textbf{rating}
of your \textbf{flaw} changes to reflect those changes.

\textbf{Flaws} should include \textbf{breakouts} when figuring the PC's
highest \textbf{ability}.

All \textbf{flaws} after the third are given the same \textbf{rating} as
the lowest \textbf{ability}. You may designate \textbf{flaws} from
\textbf{keywords} as your first or second-ranked \textbf{flaw}.

Now, you get 20 \textbf{improvement points} to spend to improve your
\textbf{abilities}. Each \textbf{improvement point} increases an ability
by +1. You cannot increase a \textbf{breakout} with \textbf{improvement
points}. The maximum starting \textbf{ability rating} is 5M; the maximum
starting \textbf{breakout rating} is +5, or +10 for a
\textbf{distinguishing characteristic} (see §3.0) agreed with your GM to
be under a \textbf{keyword}.

You can modify an ability using \textbf{improvement points} after a
roll. This may allow you to \textbf{succeed} and not \textbf{fail}.

Some genre packs may require you to have additional \textbf{keywords}
that reflect the setting. These additional \textbf{keywords} come from
the 12 \textbf{abilities} allowance, so in many genres you will have
fewer wildcard \textbf{abilities} but better fit the setting.

\hypertarget{prose-method}{%
\subsection{3.5 Prose Method}\label{prose-method}}

\hypertarget{alternative-character-creation}{%
\subsubsection{3.5.1 Alternative Character
Creation}\label{alternative-character-creation}}

This is the an alternative character creation method in which you write
a piece of prose and then pull \textbf{abilities} from that. Its intent
is to emulate a character description in fiction, and indeed PCs can be
built by copying text from a story and then identifying
\textbf{keywords}. It is the least `fair' of the character creation
options.

Your GM should choose either ``As You Go'' or ``Prose Method'' not both.
Choose the ``Prose Method'' only if your players are comfortable writing
a short biography for their character.

\hypertarget{the-prose-method}{%
\subsubsection{3.5.2 The Prose Method}\label{the-prose-method}}

You write a paragraph of text like you would see in a story outline,
describing the most essential elements of your character. Include
\textbf{keywords}, personality traits, important possessions,
relationships, and anything else that suggests what you can do and why.
The paragraph should be about 100 words long.

Compose the description in complete, grammatical sentences. No lists of
\textbf{abilities}; no sentence fragments. Your GM may choose to allow
sentences like the previous one for emphasis or rhythmic effect, but not
simply to squeeze in more cool things you can do.

Once your narrative is finished, convert the description into a set of
\textbf{abilities} and \textbf{flaws}. Mark any \textbf{keywords} with
double underlines. Mark any other word or phrase that could be an
\textbf{ability} or \textbf{flaw} with a single underline. Then write
these \textbf{keywords}, \textbf{abilities} or \textbf{flaws} on your
character sheet. Remember that some \textbf{abilities} and
\textbf{flaws} may be a \textbf{breakout} from a \textbf{keyword}.

There is no limit to the number of \textbf{abilities} you can gain from
a single sentence, as long as the sentence is not just a list of
\textbf{abilities}. If your GM decides a sentence is just a list, they
may allow the first two \textbf{abilities}, or they may tell the player
to rewrite the sentence. Note, however, that you cannot specify more
than one \textbf{sidekick} in your prose description.

\hypertarget{contests}{%
\section{4.0 Contests}\label{contests}}

A \textbf{contest} is the default resolution method for \textbf{story
obstacles} or \textbf{story question} where there is a dramatic branch
from uncertainty. If the branch does not lead to new story, just use an
\textbf{assured contest}.

A \textbf{contest} is conflict resolution, we don't resolve the
individual tasks that form part of the \textbf{story obstacle} or
\textbf{story question} instead we resolve the whole \textbf{story
obstacle} or \textbf{story question} in one roll. When you pick your
\textbf{tactic} it may encompass your approach to the whole, or be a
spotlight moment, where the part stands for the whole. In the latter
case your chosen \textbf{tactic} is the focus of the key moment. It all
comes down to this moment, where you win the \textbf{prize} or your
plans go awry. Your GM will narrate your passage through the other
obstacles once the outcome is known, but the focus of win and lose
always hinges on this moment.

Using \textbf{contest} as the default speeds up play, and keeps the
story hitting major events, reinforcing the sense of adventure.

\hypertarget{contest-1}{%
\subsection{4.1 Contest}\label{contest-1}}

\hypertarget{procedure}{%
\subsubsection{4.1.1 Procedure}\label{procedure}}

A \textbf{contest} can be summarized as follows:

\begin{enumerate}
\def\labelenumi{\arabic{enumi}.}
\tightlist
\item
  You and your GM agree upon the terms of the \textbf{contest}.
\item
  You choose a \textbf{tactic}. Your \textbf{target number (TN)} is your
  \textbf{rating}, adding any \textbf{augments} (see §2.6),
  \textbf{hindrances} (see §2.7), \textbf{stretches} and
  \textbf{situational modifiers} (see §2.5), \textbf{consequences} and
  \textbf{benefits} (see §2.8).
\item
  Your GM determines the \textbf{resistance}. If two PCs contend, your
  opponent figures their \textbf{TN} as described in step 2.
\item
  You roll a D20 vs your relevant \textbf{ability}, while your GM rolls
  a D20 vs the \textbf{resistance}.
\item
  Your GM compares the difference \textbf{successes} between the two
  rolls to assesses the \textbf{outcome} (see §2.3.7).
\item
  Your GM then narrates the \textbf{outcome} of the conflict as
  appropriate and assesses any \textbf{benefits} or
  \textbf{consequences} that arose (see §2.8).
\item
  Award \textbf{experience points} if appropriate (see §8.1).
\end{enumerate}

\hypertarget{group-contest}{%
\subsection{4.2 Group Contest}\label{group-contest}}

In the \textbf{group contest}, multiple participants take part in a
\textbf{contest}. Each of you in your group conducts an individual
\textbf{contest} against the GM, and the \textbf{outcomes} for each side
are collated to determine the victor.

A \textbf{group contest} may pit all of you against a single
\textbf{resistance}, representing one \textbf{story obstacle} or
\textbf{story question}. Alternatively, a \textbf{group contest} may be
a series of paired match-ups between two groups of contestants. If your
PC is matched against multiple opponents, your GM should consider
increasing the resistance instead of rolling multiple times (see Mobs,
Gangs, and Hordes §2.12). If the resistance is forced to face off
against multiple PCs the GM should consider, lowering the resistance
(see Ganging Up §2.11).

\hypertarget{procedure-1}{%
\subsubsection{4.2.1 Procedure}\label{procedure-1}}

A \textbf{group contest} can be summarized as follows:

\begin{enumerate}
\def\labelenumi{\arabic{enumi}.}
\tightlist
\item
  Your GM \textbf{frames the contest}.
\item
  You choose a \textbf{tactic}. Your \textbf{target number (TN)} is your
  \textbf{rating}, adding any \textbf{augments} (see §2.6),
  \textbf{hindrances} (see §2.7), \textbf{stretches} and
  \textbf{situational modifiers} (see §2.5), \textbf{consequences} and
  \textbf{benefits} (see §2.8).
\item
  Your GM determines the \textbf{resistance}. If two PCs contend, your
  opponent figures their \textbf{TN} as described in step 2.
\item
  For each of your group

  \begin{enumerate}
  \def\labelenumii{\arabic{enumii}.}
  \tightlist
  \item
    Roll a D20 vs your relevant \textbf{ability}, while your GM rolls a
    D20 vs the \textbf{resistance}. Determine the \textbf{degree} of
    \textbf{victory} or \textbf{defeat} for the individual PC from the
    \textbf{outcome} (see §2.3.7).
  \item
    Add the number of \textbf{successes} scored in the contest to a
    running total of \textbf{successes} for each group, regardless of
    the individual \textbf{outcome}. \textbf{Victory} or
    \textbf{defeat}, \textbf{successes} count towards the group
    \textbf{outcome}.
  \item
    Award \textbf{experience points} if appropriate (see §8.1).
  \end{enumerate}
\item
  The side with the highest number of \textbf{successes} is the overall
  victor in the \textbf{contest}. Award \textbf{experience points} if
  appropriate (see §8.1). If the number of \textbf{successes} is tied
  the contest ends in a stalemate, with neither side gaining control of
  the \textbf{prize}.
\item
  Determine \textbf{degree of victory} from the difference between the
  \textbf{successes} scored by each side.
\item
  Describe the \textbf{outcome} based on the agreed \textbf{prize}.
\end{enumerate}

If the result is a tie, but it does not make sense for there to be no
outcome, then award the PCs group a \emph{zero} \textbf{degree victory}

It is possible that you suffer a \textbf{defeat}, even though your side
gains the \textbf{victory}. It is possible that, as a result, that your
PC will suffer a \textbf{consequence} (see §2.8) related to your
\textbf{defeat}, even though your side won. If your side loses, then you
may suffer both a \textbf{consequence} for your own individual
\textbf{contest}, and a \textbf{consequence} for the overall
\textbf{contest}.

It is possible that you gain a \textbf{victory}, even though your side
suffers a \textbf{defeat}. It is possible that, as a result, that your
PC will obtain a \textbf{benefit} (see §2.8) related to your
\textbf{victory}, even though your side lost. If your side won, then you
may gain both a \textbf{benefit} for your own individual
\textbf{contest}, and a \textbf{benefit} for the overall
\textbf{contest}.

\hypertarget{multiple-contestants-one-prize}{%
\subsection{4.3 Multiple Contestants, One
Prize}\label{multiple-contestants-one-prize}}

Sometimes, there can be only one. When you are in a contest with
multiple contenders, but only one of you can win the \textbf{prize}, you
have to beat everyone else. Typical examples include athletic contests,
beauty contests, a game show, or an election.

When you compete for a single \textbf{prize} you compare your
\textbf{result} with the other contestants. Your GM should find the
highest number of \textbf{successes} obtained by any of the contestants.
If there is only one contestant with that number, the GM awards them the
\textbf{prize}. Otherwise, your GM will compare the \textbf{rolls} of
everyone who shares that number of successes. The contestant with the
highest \textbf{roll} wins the \textbf{prize}. If two contestants match
on the highest \textbf{roll}, then the \textbf{contest} finishes in a
\textbf{tie}, and the winners must share the \textbf{prize}. If sharing
the \textbf{prize} does not make sense, then your GM awards it, in order
of preference, to the highest \textbf{ability}, a \textbf{PC} over an
\textbf{NPC}, GM's choice.

It is possible that all the contestants lose. In this case, the GM needs
to decide if it possible for no one to win, because they failed to
finish the race or did not answer the questions in the final round of
the game show. If they cannot fail to win, it goes to the highest roll,
as above.

Use a \textbf{contest} when you have multiple contenders for a single
\textbf{prize} over a \textbf{sequence} (see §5.0).

\hypertarget{procedure-2}{%
\subsubsection{4.3.1 Procedure}\label{procedure-2}}

Multiple contestants and one prize can be summarized as follows:

\begin{enumerate}
\def\labelenumi{\arabic{enumi}.}
\tightlist
\item
  Find the highest number of \textbf{successes} amongst the
  participants.
\item
  If only one contestant has that number of \textbf{successes} amongst
  the players, award them the \textbf{prize}.
\item
  If multiple contestants share that number of \textbf{successes} the
  contestant with the highest \textbf{roll} wins the prize.
\item
  If two or more contestants match on the highest \textbf{roll}, there
  is a \textbf{tie} and the contestants with the matching \textbf{roll}
  share the \textbf{prize}.
\item
  If the \textbf{prize} cannot be shared then award to in order: highest
  \textbf{ability}, a \textbf{PC} over an \textbf{NPC}, GM's choice.
\end{enumerate}

\hypertarget{sequences}{%
\section{5.0 Sequences}\label{sequences}}

Most conflicts should be resolved simply and quickly, using the
\textbf{contest} rules. However, every so often, your GM wants to draw
out the resolution, breaking it down into a series of smaller actions,
increasing the suspense you feel as you wait to see if they triumph or
fail.

Think of the different ways a film director can choose to portray a
given moment, depending on how important it is to the story, and how
invested they want us to feel in its outcome. For example, there are two
ways to shoot a scene in which a thief breaks into the bank to steal the
contents of the safe.

\begin{itemize}
\tightlist
\item
  The action can be portrayed quickly, cutting to a moment with the
  thief, their ear pressed against the safe trying to get the tumblers
  to fall into place. Then they sigh with relief, open the safe, and get
  whatever is inside. In this instance, the story is about what happens
  after the thief gets what's in the safe, not about what might happen
  to them if they fail.
\item
  Another film might instead choose to make the bank robbery a pivotal
  turning point in the story, if not its climactic moment. It would
  spend many scenes building up to the safe-cracking sequence: obtaining
  the plans of the bank, learning the movements of the guards, crawling
  through the air conditioning ducts, sliding past the motion sensors
  and pressure plates, and finally cracking the safe itself.
\end{itemize}

A \textbf{contest} mirrors the first approach. A \textbf{sequence}
mirrors the second. If your GM wants to focus on how you complete a
sequence of tasks to overcome the \textbf{story obstacles} or
\textbf{story questions} then then use a \textbf{sequence}.

Even a movie driven by action and suspense will typically include only a
handful of these set-piece sequences. They need the rest of their
running time to build up to their big moments, to make us care about the
characters, and to give us quiet moments to contrast with the
white-knuckle parts.

So pacing may always trump your desire to work through the sequence of
tasks, as your GM may wish to resolve this conflict quickly. This is
especially true if only one player is involved.

Your GM may be tempted, to adjudicate every fight with a
\textbf{sequence}, because fights seem like they should be played out
blow-by-blow. They should resist this temptation, as fights are often
repetitive trading of blows that can drag when everyone repeats actions
from \textbf{round} to \textbf{round}. Only use \textbf{sequences} for
fights where the PCs want to do more than slug it out toe-to-toe with
their opponents until only one is left standing.

\hypertarget{sequence}{%
\subsection{5.1 Sequence}\label{sequence}}

\hypertarget{overview}{%
\subsubsection{5.1.1 Overview}\label{overview}}

A \textbf{sequence} is a succession of \textbf{contests}. It represents
a set of tasks required for the PC to overcome an obstacle.

A \textbf{sequence} is played out in \textbf{rounds}. Each
\textbf{round} represents an attempt by the PC to wear down the
\textbf{resistance} by succeeding at a task, that makes up part of the
goal. In a \textbf{sequence} where a PC thief tries to break into a bank
to steal from the safe, individual rounds might represent:

\begin{itemize}
\tightlist
\item
  Obtaining the plans of the bank
\item
  Surveillance to learn the movements of the guards
\item
  Crawling through the air conditioning ducts
\item
  Sliding past the motion sensors and pressure plates
\item
  Cracking the safe itself.
\end{itemize}

The GM needs to be aware of pacing during this, skipping potential
obstacles to try and time the final roll to be as the thief cracks open
the safe (or fails triggering more complications).

\textbf{Sequences} are longer and more dramatic than \textbf{contests}.
Your GM uses \textbf{sequences} when the drama of the conflict is as
important as the \textbf{outcome} to the story. A \textbf{sequence}
generates suspense with a back-and-forth struggle. It is something you
and your GM should visualize and describe.

A \textbf{sequence} consists of one or more \textbf{rounds}, which you
resolve as \textbf{contests}. However an individual \textbf{round} does
not decide the \textbf{outcome} of the whole \textbf{sequence}, only who
has momentum at that time.

Different types of \textbf{sequence} change how we record who has the
upper hand at any one time, based on the \textbf{degrees} of the
\textbf{contest} outcome.

\begin{itemize}
\tightlist
\item
  In a \textbf{scored sequence}, the first contestant to have five
  `strikes' against them loses. We \emph{tally} \textbf{resolution
  points (RP)}.
\item
  In a \textbf{wagered sequence}, we \emph{tally} \textbf{advantage
  points (AP)}, which represent momentum and position. The contenders
  trade \textbf{APs} until one of them runs out.
\item
  In \textbf{chained sequence} we tally \textbf{resolve}, which
  represents willingness to continue. If you choose \textbf{chained
  sequence} all contests become a \textbf{chained sequence}.
\end{itemize}

Your GM uses this record, either \emph{tallying} or applying
\textbf{consequences}, to determine when to trigger the end of the
\textbf{sequence} and the \textbf{outcome} for the victor.

\hypertarget{procedure-3}{%
\paragraph{5.1.1.1 Procedure}\label{procedure-3}}

\begin{enumerate}
\def\labelenumi{\arabic{enumi}.}
\tightlist
\item
  Your GM \textbf{frames the sequence}.
\item
  You choose a \textbf{tactic}. Your \textbf{target number (TN)} is your
  \textbf{rating}, adding any \textbf{augments} (see §2.6),
  \textbf{hindrances} (see §2.7), \textbf{stretches} and
  \textbf{situational modifiers} (see §2.5), \textbf{consequences} and
  \textbf{benefits} (see §2.8.
\item
  Your GM determines the \textbf{resistance} (see §2.3.3). (If two PCs
  contend, your opponent figures their \textbf{TN} as described in step
  2.)
\item
  Carry out one or more \textbf{rounds}, repeating as necessary.

  \begin{enumerate}
  \def\labelenumii{\arabic{enumii}.}
  \tightlist
  \item
    Your GM decides which contender has the initiative, the `aggressor',
    and describes what they are trying to do to achieve the
    \textbf{prize}. The `defender' describes how they counter the
    aggressor's attempt to seize the \textbf{prize}. If it is not
    obvious from the unfolding narrative, your GM should choose your PC
    as the `aggressor'.
  \item
    Resolve the \textbf{round} as described for the \textbf{sequence}
    type.
  \item
    The outcome determines the new \emph{tally} or applies
    \textbf{consequences}. Tied \textbf{results} leave the score
    unchanged.
  \item
    Determine if an opponent is knocked out of the \textbf{contest} from
    their \emph{tally} or \textbf{consequences}, according to the rules
    for that \textbf{contest} type.
  \item
    The winner has an opportunity to perform a \textbf{parting shot}
    (see §5.1.8).
  \end{enumerate}
\item
  Determine the \textbf{outcome} according to the \textbf{sequence}
  type. Award or deny the \textbf{prize}, and give \textbf{experience
  points} if appropriate (see §8.1).
\item
  Determine \textbf{benefits} or \textbf{consequences}.
\item
  Describe the \textbf{outcome} based on the \textbf{story obstacle} or
  \textbf{story question}
\end{enumerate}

\hypertarget{group-sequence-overview}{%
\subsubsection{5.1.2 Group Sequence
Overview}\label{group-sequence-overview}}

\textbf{Group sequences} proceed as a series of \textbf{sequences}
between pairs of PC and opponents, interwoven so that they happen nearly
simultaneously.

As in a \textbf{sequence} between a single PC and an opponent, each per
pair of adversaries contend in a \textbf{round}. Usually the PCs make up
one team, and their antagonists the other.

A \textbf{group sequence} continues until one side has no active
participants. If you \textbf{defeat} your opponent you can pair with a
new opponent. The new opponent might be unengaged, but might also be
engaged in an existing pairing. When you pair with an unengaged
opponent, you begin a new \textbf{sequence}. If your opponent is already
engaged in a \textbf{sequence}, you participate in the existing
\textbf{sequence} and points tally for that type of \textbf{sequence}.
Alternatively, if you are unopposed, you may choose to help lower the
\emph{tally} of an existing participant with an \textbf{assist}
appropriate to that \textbf{sequence} type. Of course, you may be later
engaged by an opponent who becomes free yourself.

You may lose some pairings amongst the PCs, but still win if the last
participant standing is a PC; otherwise if the last participant belongs
to the opposition you lose.

\hypertarget{group-sequence-procedure}{%
\paragraph{5.1.2.1 Group Sequence
Procedure}\label{group-sequence-procedure}}

\begin{enumerate}
\def\labelenumi{\arabic{enumi}.}
\tightlist
\item
  Your GM \textbf{frames the sequence}.
\item
  You choose a \textbf{tactic}. Your \textbf{target number (TN)} is your
  \textbf{rating}, adding any \textbf{augments} (see §2.6),
  \textbf{hindrances} (see §2.7), \textbf{stretches} and
  \textbf{situational modifiers} (see §2.5), \textbf{consequences} and
  \textbf{benefits} (see §2.8).
\item
  The GM determines the \textbf{resistance} (see §2.3.3). If two PCs
  contend, your opponent figures their \textbf{TN} as described in step
  2.
\item
  The PCs choose their opponents in order of their \textbf{TN} where it
  makes sense. Otherwise your GM will allocate opponents to you
  dependent on what makes narrative sense.
\item
  Establish an order of the paired \textbf{sequences}. The narrative may
  indicate who should go first but use your group's \textbf{TN}s from
  highest to lowest if no other option presents itself.
\item
  For each pairing your GM carries out one \textbf{round}. Then they
  repeat by carrying out more \textbf{rounds} in order, as necessary.

  \begin{enumerate}
  \def\labelenumii{\arabic{enumii}.}
  \tightlist
  \item
    Your GM decides which contender has the initiative, the `aggressor',
    and describes what they are trying to do to achieve the
    \textbf{prize}. The `defender' describes how they counter the
    aggressor's attempt to seize the \textbf{prize}. If it is not
    obvious from the unfolding narrative, your GM should choose your PC
    as the `aggressor'.
  \item
    Resolve the \textbf{round} as described for the \textbf{sequence}
    type.
  \item
    The outcome determines the new \emph{tally} or applies
    \textbf{consequences}. Tied \textbf{results} leave the score
    unchanged.
  \item
    Determine if an opponent is knocked out of the \textbf{contest},
    according to the rules for that \textbf{contest} type.
  \item
    The winner has an opportunity to perform a \textbf{parting shot}
    (see §5.1.8).
  \end{enumerate}
\item
  As one of a pair is eliminated from the \textbf{group sequence}, their
  victorious opponents may then move on to engage new targets, either
  \emph{starting new \textbf{contests}} with an unengaged opponent or
  \emph{joining an existing \textbf{contest}} (see §5.1.9). These new
  contest are added to the end of the existing roster. Alternatively,
  unengaged contests may \textbf{assist} (see §5.1.10) those who are
  engaged.
\item
  The group with the last undefeated contestant wins.
\item
  Award \textbf{experience points} if appropriate (see §8.1).
\item
  If necessary, your GM can determine your group's \textbf{degree of
  victory or defeat}.
\item
  Describe the \textbf{outcome} based on the \textbf{story obstacle} or
  \textbf{story question}.
\end{enumerate}

\hypertarget{no-nesting}{%
\subsubsection{5.1.3 No Nesting}\label{no-nesting}}

Your GM should never ``nest'' one \textbf{sequence} inside another. If a
\textbf{sequence} is in progress and you want to perform an action your
GM should treat it as an \textbf{unrelated action} (see §5.1.8), or
disallow it completely during the current \textbf{sequence}.

\hypertarget{switching-abilities}{%
\subsubsection{5.1.4 Switching Abilities}\label{switching-abilities}}

You may describe an action in a \textbf{sequence} that is not covered by
the \textbf{ability} that you started the \textbf{sequence} with. There
are two possibilities here: either you are trying to provide color to
your actions in the \textbf{round}, without seeking to gain advantage,
or you are seeking to gain advantage over your opponent with a novel
\textbf{tactic}.

In the former case, you can continue to use the \textbf{ability} you
started the contest with, as you should not be penalized for wanting to
enhance the contest with colorful or entertaining descriptions.

In the latter case you should switch \textbf{abilities}, and your GM
must decide if the \textbf{resistance} changes because of your new
\textbf{ability}. Your GM is encouraged to reward \textbf{tactics} that
exploit weaknesses that have been identified in the story so far with a
\textbf{bonus} from a \textbf{situational modifier}. Sometimes your GM
may respond with a higher \textbf{penalty} from a \textbf{situational
modifier} because your \textbf{tactic} looks less likely to succeed due
to conditions already established in the story.

Either way any \emph{tally} does not change.

\hypertarget{asymmetrical-round}{%
\subsubsection{5.1.5 Asymmetrical Round}\label{asymmetrical-round}}

You may choose to briefly suspend your attempt to best your opponent in
a \textbf{sequence}, in order to do something else. An instance where
you are trying to do something else and your opponent is trying to win
the \textbf{contest} is called an \textbf{asymmetrical round}.

In an \textbf{asymmetrical round}, you do not change the \emph{tally}
against your opponent,or inflict \textbf{consequences} if you win the
\textbf{round}. Instead, you succeed at whatever else you were doing.
Your \emph{tally} is still altered, or you suffer \textbf{consequences}
if you lose the \textbf{round}. Often you will be using an
\textbf{ability} other than the one you've been waging the
\textbf{contest} with, one better suited to the task at hand. This
becomes additionally dangerous when the \textbf{TN} associated with your
substitute \textbf{ability} is significantly lower than the one used for
the rest of the \textbf{sequence}.

In addition to secondary objectives, as in the above example, you may
engage in \textbf{asymmetrical round} to \textbf{augment} (see §2.6)
yourself or others.

\hypertarget{disengaging}{%
\subsubsection{5.1.6 Disengaging}\label{disengaging}}

You can always abandon a \textbf{sequence}, but, in addition to failing
at the \textbf{story obstacle}, you may also suffer negative
consequences as your opponent may take a \textbf{parting shot} against
you, as though you had suffered a \textbf{defeat}. Likewise, your
opponent might seek to disengage before being defeated, entitling you to
a \textbf{parting shot}. Taking the \textbf{parting shot} is always
optional as it may worsen the attacker's position.

\hypertarget{unrelated-actions}{%
\subsubsection{5.1.7 Unrelated Actions}\label{unrelated-actions}}

If you are not currently enmeshed in a \textbf{round}, you may take
actions within the scene that do not directly contribute to the
\textbf{defeat} of the other side. These \textbf{unrelated actions} may
grant an \textbf{augment} to yourself or to a teammate. You may achieve
a secondary story objective. This resembles an \textbf{asymmetrical
round}, except that, as you are not targeted by any opponents, there is
no additional risk.

\hypertarget{parting-shot}{%
\subsubsection{5.1.8 Parting Shot}\label{parting-shot}}

Immediately after you \textbf{defeat} an opponent, you may attempt to
worsen the defeat suffered by your opponent by engaging in a
\textbf{parting shot}. This is an attempt (metaphoric or otherwise) to
kick your opponent while he's down:

\begin{itemize}
\tightlist
\item
  Striking an incapacitated enemy
\item
  Attacking a retreating army
\item
  Attaching one more punitive rider to a legal settlement
\item
  Demanding additional money from a business partner
\item
  Delivering one last humiliating insult
\end{itemize}

The \textbf{parting shot} is another \textbf{contest} against your
\textbf{defeated} opponent. The \textbf{ability} you use must relate to
the consequences the opposition will suffer, but needn't be the same one
you used to win the \textbf{contest}. If the loser is a PC they use a
suitable \textbf{ability} to resist; otherwise the GM rolls a suitable
\textbf{resistance} value.

The mechanics for a \textbf{parting shot} differ for each
\textbf{sequence} type, reflecting the \emph{tally}.

\hypertarget{joining-an-in-progress-contest}{%
\subsubsection{5.1.9 Joining an In-Progress
Contest}\label{joining-an-in-progress-contest}}

When you wish to join a \textbf{sequence} in progress, you and your GM
should discuss whether you accept the current framing. If so, you can
participate. You simply select an opponent and start a new
\textbf{round} with them. The mechanics for joining an in-progress
\textbf{contest} differ for each \textbf{sequence} type, reflecting the
\emph{tally}. If you want to achieve something other than the goal
established during framing, you may instead perform \textbf{unrelated
actions}, including \textbf{assists} and \textbf{augments}.

In some types of conflict, many-against-one may have an advantage. In a
melee for example it is more difficult to fight two or more opponents.
In other types of contest many-against-one may actually hinder because
contenders get in each other's way, such as an attempt to persuade
another. In situations in a \textbf{sequence} where numerical advantage
exists, award a \textbf{situational modifier} usually of +/-5 or +/-10.
This is a \textbf{bonus} if the PCs outnumber their opponents and a
\textbf{penalty} if they don't. Don't use this rule if you are treating
the \textbf{resistance} as a mob (see §2.10)

\hypertarget{assists}{%
\subsubsection{5.1.10 Assists}\label{assists}}

You may take an \textbf{unrelated action} to grant an \textbf{assist} to
a teammate enmeshed in a \textbf{round}. Describe what your character is
trying to do to improve the position of the target. For example, your PC
might throw them a weapon, jeer at an opponent, or simply shout words of
encouragement. \textbf{Assists} are subject to the same restrictions as
\textbf{augments}: they must be both credible and interesting.

The mechanics for a \textbf{assist} differ for each \textbf{sequence}
type, reflecting the \emph{tally} or \textbf{consequences}.

\hypertarget{outcome-1}{%
\subsubsection{5.1.11 Outcome}\label{outcome-1}}

Your GM may wish to determine the overall outcome, beyond the prize and
individual outcomes.

If your group has the ``last contestant'', your GM should use the
\textbf{outcome} for the contest in which your group \textbf{defeated}
the opponent with the highest \textbf{ability rating}.

If the opposition has the ``last contestant'', your GM should use the
\textbf{outcome} for the contest in which the PC with the highest
\textbf{ability rating} was \textbf{defeated}.

In the event of a tie for \textbf{ability rating} pick the
\textbf{outcome} that has the second highest \textbf{degree} of
\textbf{victory} or \textbf{defeat}.

\hypertarget{scored-sequence}{%
\subsection{5.2 Scored Sequence}\label{scored-sequence}}

A \textbf{scored sequence} consists of one or more \textbf{rounds}; each
\textbf{round} is a \textbf{contest}.

In a \textbf{scored sequence}, we \emph{tally} the position of the
contestants in a \textbf{scored sequence} via \textbf{resolution
points}. Once \emph{five} or more resolution points have been
\emph{tallied} against a contestant, they lose that \textbf{scored
sequence}.

In a \textbf{scored sequence} each \textbf{round} represents attempts by
both parties to overcome their opponent and so a is a \emph{single}
\textbf{contest}.

Your GM should determine who has the initiative to choose their
\textbf{tactics} for any \textbf{round}, based on their interpretation
of the flow of events. Their opponent has to react to that
\textbf{tactic}. If in doubt your GM should defer to you over your
opponent to describe what you do in the \textbf{round}, and describe the
NPC reacting to that.

\hypertarget{resolution-points}{%
\subsubsection{5.2.1 Resolution Points}\label{resolution-points}}

You score \textbf{resolution points} equal to \emph{one} more than the
\textbf{degree} of the \textbf{outcome}. So a \emph{zero}
\textbf{degree} \textbf{outcome} produces \emph{one} \textbf{resolution
point}, a \emph{one} \textbf{degree} \textbf{outcome} produces
\emph{two} \textbf{resolution points} and so on. (You can take the
shortcut of adding \emph{one} to the difference in \textbf{successes} of
the two rolls, if you prefer).

\hypertarget{resolution-point-table}{%
\paragraph{5.2.1.1 RESOLUTION POINT
TABLE}\label{resolution-point-table}}

\begin{longtable}[]{@{}cc@{}}
\toprule
Degree & Value \\
\midrule
\endhead
0 & 1 \\
1 & 2 \\
2 & 3 \\
3 & 4 \\
4 & 5 \\
\bottomrule
\end{longtable}

\hypertarget{resource-point-knowledge}{%
\paragraph{5.2.1.2 Resource Point
Knowledge}\label{resource-point-knowledge}}

Your GM should make the \textbf{resource point} totals for each side
public.

\hypertarget{followers}{%
\paragraph{5.2.1.3 Followers}\label{followers}}

You may choose to have your \textbf{followers} take part in
\textbf{scored sequences} in one of three ways: as full contestants, as
secondary contestants, or as supporters.

\textbf{Contestant}: The \textbf{follower} takes part in the
\textbf{contest} as any other PC would. You roll for your
\textbf{followers} as you would their main characters. However, your
\textbf{followers} are removed from the \textbf{contest} whenever 3
\textbf{resolution points} are scored against them in a given
\textbf{round}.

\textbf{Secondary contestant}: To act as a secondary contestant, your
\textbf{follower} must have an \textbf{ability} relevant to the
\textbf{contest}. The \textbf{follower} sticks by your side,
contributing directly to the effort: fighting in a battle, tossing in
arguments in a legal dispute, acting as the ship's navigator, or
whatever. Although you describe this, you do not roll for the
\textbf{follower}. Instead, you may, at any point, shift any number of
\textbf{resolution points} to a \textbf{follower} acting as a secondary
contestant. Followers with 3 or more \textbf{resource points} lodged
against them are removed from the scene.

\textbf{Supporter}: Your \textbf{follower} is present in the scene, but
does not directly engage your opponents. Instead they may perform
\textbf{assists} and other \textbf{unrelated actions}.

\textbf{Followers} acting in any of these three capacities may be
removed from the \textbf{contest} by otherwise unengaged opponents. To
remove a \textbf{follower} from a scene, an opponent engages your
\textbf{follower} in a \textbf{contest}. Your GM sets the
\textbf{resistance}, or if it is another PC's \textbf{follower} they
determine the relevant \textbf{ability} of the \textbf{follower}
engaging yours. On any failure, your \textbf{follower} is taken out of
the \textbf{contest}.

Your GM determines any long-term implications for the follower being
removed from the \textbf{contest}. Whilst your GM should not end your
character's story without consent, such as via death, they may choose to
end the story of a follower in such circumstances, viscerally
demonstrating the threat that the PCs face.

\hypertarget{what-the-score-means}{%
\paragraph{5.2.1.4 What the Score Means}\label{what-the-score-means}}

Your \textbf{resolution point} score tells you how well you're doing,
relative to your opponent, in the ebb and flow of a fluid, suspenseful
conflict. If you're leading your opponent by 0--4, you're giving them a
thorough pasting. If you're behind 4--0, you're on your last legs, while
your opponent has had an easy time of it. If you're tied, you've each
been getting in some good licks.

\begin{itemize}
\tightlist
\item
  In a fight, scoring \emph{one} \textbf{RP} might mean that you hit
  your opponent with a grazing blow, or knocked him into an awkward
  position.
\item
  Scoring \emph{two} \textbf{RPs} might mean a palpable hit, most likely
  with bone-crunching sound effects.
\item
  A \emph{three} \textbf{RP} hit sends them reeling, and, depending on
  the realism level of the genre, may be accompanied by a spray of
  blood.
\end{itemize}

However, the exact physical harm you've dished out to them remains
unclear until the \textbf{contest's} end. When that happens, the real
effects of your various \textbf{victories} become suddenly apparent.
Perhaps they stagger, merely dazed, up against a wall. Maybe they fall
over dead.

\begin{itemize}
\tightlist
\item
  In a debate, \emph{one} \textbf{RP} might occasion mild head nodding
  from spectators, or a frown on your opponent's face.
\item
  \emph{Two} \textbf{RPs} would occasion mild applause from onlookers,
  or send a flush to your opponent's face.
\item
  On \emph{three} \textbf{RPs}, your opponent might be thrown completely
  off-track, as audience members wince at the force of your devastating
  verbal jab.
\end{itemize}

If the opposition represents multiple opponents, then an \textbf{RP}
loss may represent their ranks thinning.

In interpreting the individual \textbf{contest} \textbf{rounds} within a
\textbf{scored sequence}, your GM is guided by two principles:

\begin{enumerate}
\def\labelenumi{\arabic{enumi}.}
\tightlist
\item
  No effect is certain until the entire \textbf{scored sequence} is
  over.
\item
  When a character scores points, it can reflect any positive change in
  fortunes, not just the most obvious one.
\end{enumerate}

\hypertarget{scored-sequence-outcomes}{%
\subsubsection{5.2.2 Scored Sequence
Outcomes}\label{scored-sequence-outcomes}}

Your GM uses the difference in \textbf{resource points} between the
contestants to determine the \textbf{degree} of your \textbf{victory} or
\textbf{defeat}.

\hypertarget{scored-sequence-outcome-table}{%
\paragraph{5.2.2.1 SCORED SEQUENCE OUTCOME
TABLE}\label{scored-sequence-outcome-table}}

\begin{longtable}[]{@{}cc@{}}
\toprule
Difference & Degree \\
\midrule
\endhead
1-2 & 0 \\
3-4 & 1 \\
5-6 & 2 \\
7-8 & 3 \\
9+ & 4 \\
\bottomrule
\end{longtable}

Your GM applies results as described in §2.3.7.2, including assigning
\textbf{benefits} and \textbf{consequences}.

\hypertarget{parting-shot-1}{%
\subsubsection{5.2.3 Parting Shot}\label{parting-shot-1}}

If you succeed in your \textbf{parting shot} roll, you score additional
\textbf{resource points} against your opponent, worsening their defeat.

However, if your opponent succeeds, they take the number of
\textbf{resolution points} they would, in a standard \textbf{round},
score against you, and instead subtracts them from the number of
\textbf{resolution points} scored against them in the \textbf{round}
that removed them from the \textbf{sequence}. If the revised total is
now less than 5 \textbf{RPs}, they return to the \textbf{sequence}, and
may re-engage you. Your GM describes this as a dramatic turnaround, in
which your overreaching has somehow granted them an advantage allowing
them to recover from their previous misfortune.

Where it makes sense, unengaged PCs may attempt \textbf{parting shots}
against opponents taken out of the \textbf{sequence} by someone else.
You may not revive your teammates by using your lamest abilities to make
\textbf{parting shots} on them; this, by definition, does not pass a
\textbf{credibility test}.

\hypertarget{risky-gambits}{%
\subsubsection{5.2.4 Risky Gambits}\label{risky-gambits}}

During a \textbf{scored sequence}, you can attempt to force a conflict
to an early resolution by making a \textbf{risky gambit}. If you win the
\textbf{round}, you lodge an additional 1 \textbf{resolution point}
against your opponent. However, if you lose the \textbf{round}, your
opponent lodges an additional 2 \textbf{resolution points} against you.

If both contestants engage in a \textbf{risky gambit}, the winner lodges
an additional 2 \textbf{resolution points} against the loser.

\hypertarget{defensive-responses}{%
\subsubsection{5.2.5 Defensive Responses}\label{defensive-responses}}

In a \textbf{scored sequence}, you can make a \textbf{defensive
response}, lowering the number of \textbf{resolution points} lodged
against you in a \textbf{round}. If you win the \textbf{round}, the
number of \textbf{resolution points} you lodge against your opponent
decreases by 1. If you lose, your opponent lodges 2 fewer
\textbf{resolution points} against you. The total number of
\textbf{resolution points} assigned by a \textbf{round} is never less
than 0; there is no such thing as a negative \textbf{resolution point}.

\hypertarget{assists-1}{%
\subsubsection{5.2.6 Assists}\label{assists-1}}

The \textbf{assist} alters the score against your teammate according to
the \textbf{outcome} of a \textbf{contest}.

Your first \textbf{assist} faces the \textbf{base resistance}. Each
subsequent \textbf{assist} attempt to the same beneficiary increases the
\textbf{resistance} by +5. The \textbf{resistance} escalation occurs
even when another PC steps in to make a subsequent \textbf{assist}. This
escalation allows the occasional dramatic rescue but makes it difficult
for players to prolong losing battles to excruciating length. Your GM
should make it seem credible by justifying the increasing
\textbf{resistances} with descriptions of ever-escalating
countermeasures on the part of the opposition.

Your GM may adjust the starting \textbf{resistance} up or down to
account for campaign credibility or other dramatic factors. If an
\textbf{assist} as proposed seems too improbable or insufficiently
useful, your GM should collaborate with you to propose alternate
suggestions.

On a victory, you reduce the number of \textbf{resolution points} by
\emph{one} more than the \textbf{degree} of the \textbf{victory}. On a
defeat, you increase the number of \textbf{resolution points} by
\emph{one} more than the \textbf{degree} of the \textbf{resistance's}
\textbf{victory}. See table §5.2.2.1.

Scores can never be reduced below 0.

\hypertarget{joining-an-in-progress-contest-1}{%
\subsubsection{5.2.7 Joining an In-Progress
Contest}\label{joining-an-in-progress-contest-1}}

If you join in an existing \textbf{scored sequence} use the same
\emph{tally} of \textbf{resolution points}; your fate is now bound to
your comrade's fate. In this case, two PCs may face off against the same
opposition, both sharing the same \emph{tally}.

It is usually a better tactic to reduce a comrades \emph{tally} via an
\textbf{assist} if they have \textbf{resolution points} against them,
before joining them, so that you don't risk being brought down by them.

\hypertarget{wagered-sequence}{%
\subsection{5.3 Wagered sequence}\label{wagered-sequence}}

A \textbf{wagered sequence} consists of one or more \textbf{rounds}. In
a \textbf{round} \emph{both} you and your opponent take actions in turn;
a pair of \textbf{exchanges}. Each \textbf{exchange} is a
\textbf{contest}.

In a \textbf{wagered sequence}, we \emph{tally} the position of the
contestants in \textbf{advantage points (AP)}. Once a contestant reach
\emph{zero} or fewer \textbf{APs} they lose that a \textbf{wagered
sequence}.

Your GM should determine who has the initiative to describe what they
are doing for any \textbf{exchange}. They describe the \textbf{tactic}
for their \textbf{exchange}, and wagers a matching number of
\textbf{APs}. Their opponent then describes their reaction to that
\textbf{tactic} to determine the ability used in the contest. Next,
their opponent then describes their \textbf{tactic} and wagers a
matching number of \textbf{APs}. The initiative holder then describes
their \textbf{tactic} in response.

These \textbf{exchanges} are then conducted in order of the highest
\textbf{AP} \textbf{wager} first.

If a contestant falls below \emph{zero} \textbf{APs} as a result of
their opponent's \textbf{exchange}, they immediately lose, and do not
get to conduct their own \textbf{exchange}.

If on the second \textbf{exchange} of the \textbf{round}, a contestant
no longer has the \textbf{AP} that they \textbf{bid} remaining, they
should change their action to reflect a lower \textbf{bid}.

If, on the second \textbf{exchange} the contestant's original intent no
longer makes sense in the unfolding narrative, they can change their
\textbf{tactic} but they do not change their \textbf{wager}.

\hypertarget{advantage-points}{%
\subsubsection{5.3.1 Advantage Points}\label{advantage-points}}

\hypertarget{starting-ap-totals}{%
\paragraph{5.3.1.1 Starting AP Totals}\label{starting-ap-totals}}

You describe your action towards the desired \textbf{prize} and what
\textbf{ability} you use. The \textbf{ability} used in the contest can
be varied, but \textbf{APs} are always calculated on the first
\textbf{ability} that you use in a contest. That \textbf{ability} must
be used in the first \textbf{round}.

Figure your starting \textbf{advantage point (AP)} total using the
\textbf{TN}. So a \textbf{TN} of 15 is 15\textbf{APs}. The \textbf{AP}
include +20 for each level of \textbf{mastery}, so a \textbf{TN} of 5M
is 25\textbf{APs}. Your starting \textbf{APs} and can also be increased
by \textbf{followers} (see below).

The GM figures starting \textbf{APs} for the \textbf{resistance} from
the \textbf{resistance} \textbf{TN}.

\hypertarget{wagering-advantage-points}{%
\paragraph{5.3.1.2 Wagering Advantage
Points}\label{wagering-advantage-points}}

You gamble a number of your \textbf{APs} in an attempt to reduce your
opponent's \textbf{AP}, but if you fail the attempt you lose the
\textbf{AP} or may even transfer them.

You describe your action towards the desired \textbf{prize}, what
\textbf{ability} you use, and how much risk you take. ``I want to climb
straight up to that outcrop, taking chances if needed.'' You should
specify your \textbf{wager}; if you do not, your GM determines this
based on the amount of risk you are taking.

The size of the \textbf{wager} mirrors how bold and risky your
character's action is. Extreme or aggressive actions mean a high
\textbf{wager}, and cautious actions require less.

If you describe an all-out offensive with your sword cutting vicious
arcs, you need to wager a lot of \textbf{APs}; if you say that you are
circling your foe cautiously, a low \textbf{wager} is in order.

Your GM will look at the level of risk you are taking, and may suggest
that you change your \textbf{wager} to better match your actions.

If you do not declare a \textbf{wager} before rolling the die, your GM
will decide how many points are \textbf{wager} (using 3 as a default),
with riskier actions calling for higher \textbf{wagers}.

\hypertarget{losing-advantage-points}{%
\paragraph{5.3.1.3 Losing Advantage
Points}\label{losing-advantage-points}}

The number of advantage points lost by a contestant is a multiplier of
their \textbf{wager} depending on the \textbf{degree} of the
\textbf{victory}. Determine the multiplier used as follows:

\hypertarget{wagered-sequence-exchange-table}{%
\paragraph{5.3.1.4 WAGERED SEQUENCE EXCHANGE
TABLE}\label{wagered-sequence-exchange-table}}

\begin{longtable}[]{@{}cc@{}}
\toprule
Degree & AP loss by loser \\
\midrule
\endhead
Tie & Both lose 1/2 wager, round up \\
0 & ½ x wager, round up \\
1 & 1 x wager \\
2 & 2 x wager \\
3 & 3 x wager \\
4 & 4 x wager \\
\emph{n} & \emph{n} x wager \\
\bottomrule
\end{longtable}

If the victor rolled a \textbf{big success}, the \textbf{APs} lost by
the loser are gained by the winner - a \textbf{transfer}.

\hypertarget{followers-and-advantage-points}{%
\paragraph{5.3.1.5 Followers and Advantage
Points}\label{followers-and-advantage-points}}

\textbf{Followers} can act in \emph{one} of the following ways during a
\textbf{contest}:

\begin{itemize}
\tightlist
\item
  A follower can \textbf{augment} you with their \textbf{abilities}.
\item
  You can use one of their \textbf{abilities} as if it were your own.
\item
  For a \textbf{follower} with a relevant \textbf{ability} or
  \textbf{keyword}, you can simply add their \textbf{APs} to yours at
  the beginning of the \textbf{contest}.
\end{itemize}

Neither you nor the GM makes rolls for \textbf{followers}. Instead,
their actions are subsumed into yours.

You can assign your \textbf{followers} to someone else, although you may
have to succeed at a contest to persuade a reluctant follower to go
along.

\hypertarget{advantage-point-knowledge}{%
\paragraph{5.3.1.6 Advantage Point
Knowledge}\label{advantage-point-knowledge}}

Once your opponent has won or lost \textbf{APs} during the current
contest, you can ask the GM what the opposition's \textbf{AP} total is.
This is where the element of skill comes in. When choosing how many
\textbf{APs} to stake, you must weigh the effect they want to gain if
you succeed versus the risk you face if the action fails.

\hypertarget{advantage-point-recalculation}{%
\paragraph{5.3.1.7 Advantage Point
Recalculation}\label{advantage-point-recalculation}}

\textbf{Advantage points} are only relevant for the length of a
particular \textbf{wagered sequence}. Your PC does not have any until
the next \textbf{wagered sequence} begins, when you calculate them all
over again.

\hypertarget{what-the-ap-total-means}{%
\paragraph{5.3.1.8 What the AP Total
Means}\label{what-the-ap-total-means}}

\textbf{Advantage points} represent who has the advantage in a
\textbf{sequence}. They represent who is enters the fray in the best
starting position because of skill, support, or conviction.

Losses represent:

\begin{itemize}
\tightlist
\item
  Losing \emph{momentum}.
\item
  Becoming overcome by \emph{fatigue}.
\item
  \emph{Morale} is failing.
\item
  Becoming close to losing \emph{consciousness} from repeated blows.
\item
  Losing the support of the \emph{audience}.
\end{itemize}

If the opposition represents multiple opponents, then an \textbf{AP}
loss may represent their ranks thinning.

When your GM narrates \textbf{AP} losses they need to be guided by two
principals:

\begin{enumerate}
\def\labelenumi{\arabic{enumi}.}
\tightlist
\item
  No effect is certain until the entire \textbf{wagered sequence} is
  over.
\item
  When a character loses points, it can reflect any negative change in
  fortunes, not just the most obvious one
\end{enumerate}

\hypertarget{wagered-sequence-outcomes}{%
\subsubsection{5.3.2 Wagered Sequence
Outcomes}\label{wagered-sequence-outcomes}}

Your GM uses the final \textbf{AP} total of the loser to determine the
\textbf{degree} of the \textbf{victory} or \textbf{defeat} for the PC.

\hypertarget{wagered-sequence-outcome-table}{%
\paragraph{5.3.2.1 WAGERED SEQUENCE OUTCOME
TABLE}\label{wagered-sequence-outcome-table}}

\begin{longtable}[]{@{}cc@{}}
\toprule
Final AP Total & Degree \\
\midrule
\endhead
0 to --10 AP & 0 \\
--11 to --20 AP & 1 \\
--21 to --30 AP & 2 \\
--31 or -40 AP & 3 \\
--41 or fewer AP & 4 \\
\bottomrule
\end{longtable}

Your GM may apply \textbf{consequences} and \textbf{benefits} for your
PC as they see fit, based on this outcome.

\hypertarget{parting-shot-2}{%
\subsubsection{5.3.3 Parting Shot}\label{parting-shot-2}}

A \textbf{parting shot} allows another round, in which only contestants
with positive \textbf{APs} can act. You once again \textbf{wager}
\textbf{APs} and use an appropriate \textbf{ability} against your
opponent. Your \textbf{wager} must reflect what you are doing to drive
them to greater defeat.

If you succeed, their \textbf{AP} will decrease; their \textbf{outcome}
may or may not change.

\textbf{Parting shots} are risky; if you fail, an \textbf{AP} transfer
might bring your opponent back into positive \textbf{APs} in which case
they get an \textbf{exchange} in this \textbf{round} (and subsequent
rounds as normal). Your stumble gave them an opening that they exploited
in an effort to snatch \textbf{victory} from the jaws of
\textbf{defeat}.

\hypertarget{desperation-stake}{%
\subsubsection{5.3.4 Desperation Stake}\label{desperation-stake}}

You can stake more \textbf{advantage points} than you currently have, to
a maximum of your starting \textbf{AP} total. This allows you to attempt
a \textbf{desperation stake} even when you are within a single
\textbf{AP} of \textbf{defeat}. Your GM can never stake more
\textbf{advantage points} than they have.

\hypertarget{second-chance}{%
\subsubsection{5.3.5 Second Chance}\label{second-chance}}

If your PC falls to 0 or fewer \textbf{advantage points} in a standard
\textbf{wagered sequence}, you are \textbf{defeated}. In a \textbf{group
wagered sequence}, however, you can try a \textbf{second chance} to stay
in the \textbf{contest}. A \textbf{second chance} represents the knack
to come back when your opponent turns away to gloat or deal with the
other player characters. A character may only attempt one \textbf{second
chance} in any \textbf{wagered sequence}.

To attempt a \textbf{second chance}, you must be free from attention by
the opposition. You must spend a \textbf{story point}. You can use a
relevant \textbf{ability} in a \textbf{contest} against the number of
\textbf{APs} your PC is below 0. Even if you succeed, a
\textbf{consequence} applies: take a --6 to further actions in this
contest.

If you win the \textbf{contest}, you rejoin the contest with a positive
\textbf{AP} total. Your new total is a 1/4 of your original \textbf{AP}
total at the outset of the \textbf{contest}, round up.

Your GM should not use a \textbf{second chance} for the
\textbf{resistance}.

Your GM may decide to impost a \textbf{consequence} on you, even if you
are later victorious in a contest, or your team wins the prize, that
represents the adversity you suffered that brought you initially to
defeat.

\hypertarget{assists-2}{%
\subsubsection{5.3.6 Assists}\label{assists-2}}

You can transfer some or all of your \textbf{advantage points} to
another contestant engaged in a \textbf{wagered sequence} on your side.
With more \textbf{advantage points}, they can stay in the
\textbf{sequence} for longer, or make larger \textbf{wagers} without
driving themselves to \textbf{defeat}.

State the number of \textbf{AP} you are trying to transfer. (The GM may
suggest a higher or lower \textbf{wager} based on the action you
describe.) The number of \textbf{APs} you are attempting to transfer is
the \textbf{resistance} you face in a \textbf{contest}. You lost the
\textbf{APs} whether or not you \textbf{succeed} in the contest.

You cannot transfer \textbf{advantage points} to yourself.

If a \textbf{follower's AP} are already included in your \textbf{AP}
total, the \textbf{follower} cannot transfer them to you.

\hypertarget{joining-an-in-progress-contest-2}{%
\subsubsection{5.3.7 Joining an In-Progress
Contest}\label{joining-an-in-progress-contest-2}}

Both you and your opponent use your existing \textbf{AP} totals.

\hypertarget{chained-sequence}{%
\subsection{5.4 Chained Sequence}\label{chained-sequence}}

A \textbf{chained sequence} consists of one or more \textbf{rounds};
each \textbf{round} is a \textbf{contest}.

Electing to use a \textbf{chained sequence} changes all
\textbf{contests} into \textbf{sequences}. A \textbf{contest} is always
just a \textbf{round} of a \textbf{sequence}.

In a \textbf{chained sequence} each \textbf{round} represents an attempt
by both parties to overcome their opponent. After each \textbf{round},
participants who are able to continue must decide if they wish to
continue the \textbf{sequence} or if they wish to \textbf{disengage} and
yield the \textbf{prize} to their opponent.

In a \textbf{chained sequence} we track \textbf{resolve}. Once your
\textbf{resolve} reaches \emph{zero} you cannot continue without
recovering \textbf{resolve} (see §5.4.9).

When electing to use \textbf{chained sequences} be aware of the impact
of this choice.

Your GM should award \textbf{consequences} and \textbf{benefits} to a
victorious PC after the \textbf{sequence} ends (see §2.6).

Your GM should determine who has the initiative to describe what they
are doing for any \textbf{round}, based on their interpretation of the
flow of events. If in doubt your GM should defer to you over your
opponent to describe what you do in the \textbf{round}, and describe the
NPC reacting to that.

\hypertarget{resolve}{%
\subsubsection{5.4.1 Resolve}\label{resolve}}

\textbf{Resolve} represents your resistance to the exhaustion of your
mental, physical, emotional or social reserves. Once your
\textbf{resolve} is gone, you quit.

The loser of a \textbf{round} in a \textbf{sequence}, loses
\textbf{resolve} equal to their opponent's \textbf{degrees of victory}
plus one, see §5.4.1.1.

When your resolve hits \emph{zero} it initiates a crisis, your desire to
struggle on is gone: you collapse into exhaustion; you give in to
despair; you succumb to your wounds; you flee in fear\ldots{} depending
on what harmed you. You cannot continue with the \textbf{sequence} and
lose the \textbf{prize}. Until you recover positive \textbf{resolve},
you must rest and may not initiate a \textbf{sequence}.

Your \textbf{resolve} may become negative due to losses from a
\textbf{round}. You cannot return to play until your \textbf{resolve}
becomes positive.

\hypertarget{resolve-loss-table}{%
\subsubsection{5.4.1.1 RESOLVE LOSS TABLE}\label{resolve-loss-table}}

\begin{longtable}[]{@{}cc@{}}
\toprule
Degree of Victory & Lost Resolve \\
\midrule
\endhead
0 & 1 \\
1 & 2 \\
2 & 3 \\
3 & 4 \\
4 & 5 \\
\bottomrule
\end{longtable}

\hypertarget{chained-sequence-outcomes}{%
\subsubsection{5.4.2 Chained Sequence
Outcomes}\label{chained-sequence-outcomes}}

Your GM uses the number of \textbf{resolve points} scored against the
losing contestant to determine the \textbf{degree} of your
\textbf{victory} or \textbf{defeat}. Do not count \textbf{resolve
points} traded for \textbf{consequences}, but do count any losses
absorbed by followers.

\hypertarget{chained-sequence-outcome-table}{%
\paragraph{5.4.2.1 CHAINED SEQUENCE OUTCOME
TABLE}\label{chained-sequence-outcome-table}}

\begin{longtable}[]{@{}cc@{}}
\toprule
Loss & Degree \\
\midrule
\endhead
1 & 0 \\
2-3 & 1 \\
4-5 & 2 \\
6-7 & 3 \\
8+ & 4 \\
\bottomrule
\end{longtable}

Your GM applies results as described in §2.3.7.2, including assigning
\textbf{benefits} and \textbf{consequences}.

\hypertarget{pc-resolve}{%
\subsubsection{5.4.3 PC Resolve}\label{pc-resolve}}

A PC begins with a \emph{starting} \textbf{resolve} of \emph{five}. On
your PC's sheet record your current \textbf{resolve}. You can use check
boxes to visually represent \textbf{resolve}.

\hypertarget{exchange-resolve-for-consequences}{%
\paragraph{5.4.3.1 Exchange Resolve for
Consequences}\label{exchange-resolve-for-consequences}}

As an alternative to losing \textbf{resolve} in a \textbf{round}, you
can choose to take a \textbf{consequence}.

Your GM should create a \textbf{consequence} which should be appropriate
to the source of harm. A \textbf{consequence} only negates
\textbf{resolve} losses from the most recent \textbf{round}; a
consequence cannot negate \textbf{resolve} losses from earlier
\textbf{rounds}. If you accept that \textbf{consequence} you do not lose
\textbf{resolve} for this \textbf{round} but instead take the
\textbf{consequence}.

\begin{itemize}
\tightlist
\item
  Instead of marking 1 point of \textbf{resolve}, the GM offers a
  penalty -5.
\item
  Instead of marking 2 points of \textbf{resolve}, the GM offers a
  penalty of -10.
\item
  Instead of marking 3 points of \textbf{resolve}, the GM offers a
  penalty of -15.
\item
  Instead of marking 4 points of \textbf{resolve}, the GM offers a
  penalty of -20.
\end{itemize}

You can't buy off 5 points of \textbf{resolve}, instead your resolve is
immediately reduced to \emph{zero} and you yield the \textbf{sequence}.

\hypertarget{resolve-and-incredible-powers}{%
\paragraph{5.4.3.2 Resolve and Incredible
Powers}\label{resolve-and-incredible-powers}}

In some settings your use of incredible powers may be exhausting and may
cost \textbf{resolve}.

A setting may use one of several approaches to \textbf{resolve} losses
for incredible powers.

\begin{itemize}
\tightlist
\item
  Usage of the power costs \textbf{resolve} on any \textbf{degree of
  defeat}.
\item
  Usage of the power costs resolve on a partial victory (\emph{zero}
  \textbf{degrees of success}).
\end{itemize}

\hypertarget{followers-1}{%
\subsubsection{5.4.4 Followers}\label{followers-1}}

You may choose to have your \textbf{followers} take part in
\textbf{chained sequence} in one of three ways: as full contestants, as
secondary contestants, or as supporters. You track resolve for
\textbf{followers} as you do for a PC, but a \textbf{follower} begins
the game with \emph{three} resolve. \textbf{Retainers} are treated as a
group for tracking \textbf{resolve} but \textbf{sidekicks} are handled
individually.

\textbf{Contestant}: The \textbf{follower} takes part in the
\textbf{contest} as any other PC would. You roll for your
\textbf{followers} as you would their main characters.

\textbf{Secondary contestant}: To act as a secondary contestant, your
\textbf{follower} must have an \textbf{ability} relevant to the
\textbf{contest}. The follower provides an \textbf{augment} to your
character in a \textbf{chained sequence}. In addition, if you suffer a
\textbf{defeat} in a \textbf{round} of a \textbf{chained sequence} you
may transfer that \textbf{resolve} loss to a \textbf{follower}. Doing so
takes that \textbf{follower} out of further \textbf{rounds} of the
\textbf{chained sequence}, even if they still have remaining
\textbf{resolve}. Any \textbf{augment} your \textbf{follower} provided
is lost.

\textbf{Supporter}: Your \textbf{follower} is present in the scene, but
does not directly engage your opponents. Instead they may perform
\textbf{assists} and other \textbf{unrelated actions}.

If a \textbf{follower} checks off \emph{three or more} \textbf{resolve}
they leave your service immediately - they may be dead, exhausted, or in
despair - and must be replaced.

\hypertarget{npc-resolve}{%
\subsubsection{5.4.5 NPC Resolve}\label{npc-resolve}}

We divide NPCs into two categories for \textbf{resolve}: mooks and named
NPCs.

Your GM should track \textbf{resolve} losses for an NPC. An NPC cannot
trade \textbf{resolve} losses for a \textbf{consequence}.

\hypertarget{mooks}{%
\paragraph{5.4.5.1 Mooks}\label{mooks}}

A mook is a faceless, nameless NPC who exists to allow your PC to
impress us with their competency. A mook yields a \textbf{sequence}
after suffering \emph{one} point of \textbf{resolve} loss.

\hypertarget{named-npcs}{%
\paragraph{5.4.5.2 Named NPCs}\label{named-npcs}}

A named NPC starts with between \emph{three} \textbf{resolve}.

Your GM should decide if an NPC has recovered any \textbf{resolve}
losses between encounters. An NPC who is reduced to \emph{zero} resolve
should not re-appear in the story - your PC has overcome them.

Some rare encounters, NPCs who your GM intends as an individual threat
to a group of NPCs, may have greater \textbf{resolve}. This should be
used sparingly.

\hypertarget{resolve-for-impersonal-opposition}{%
\subsubsection{5.4.6 Resolve for Impersonal
Opposition}\label{resolve-for-impersonal-opposition}}

Impersonal opposition - a security system, a science or engineering
problem, a mountain to climb or wilderness to cross - can have resolve
too, representing how resistant it is to resolution. Impersonal
opposition may include people, where are large number can be treated in
an abstract fashion - the prison guards, the ship's crew - as such the
\textbf{rating} reflects a collective \textbf{resistance}.

To justify your loss of \textbf{resolve} in a conflict, your GM should
only call for a \textbf{sequence} when it can be explained how you might
become worn out attempting to overcome the obstacle or run out of time,
otherwise your GM should let you succeed if you have a relevant
\textbf{ability}. The \textbf{resolve} of impersonal opposition
represents your progress - as you complete more steps towards overcoming
the opposition - defeating security systems, winning hearts and minds -
then the \textbf{resolve} of the impersonal opposition ebbs away.

For a lot of impersonal opposition, where it has not taken on the role
of a ``character'' or become a key obstacle in the evolving story, your
GM will just set the \textbf{resolve} at \emph{one}.

For more challenging impersonal opposition, your GM may consider setting
\textbf{resolve} at \emph{three} for an individual conflict, or higher
if overcoming it is a group effort. Where \textbf{resolve} represents
overcoming a group being treated as a single opponent, then a
\textbf{resolve} of \emph{four} of \emph{five} may be appropriate.

\hypertarget{group-chained-sequence-outcomes}{%
\subsubsection{5.4.7 Group Chained Sequence
Outcomes}\label{group-chained-sequence-outcomes}}

In a \textbf{group chained sequence} the side that has the last
contestant that has not disengaged or exhausted their \textbf{resolve}
gains the \textbf{prize}.

\hypertarget{parting-shot-3}{%
\subsubsection{5.4.8 Parting Shot}\label{parting-shot-3}}

If you succeed in your \textbf{parting shot} roll, you inflict
additional \textbf{resolve} losses on your opponent, worsening their
\textbf{defeat}.

However, if your opponent succeeds, they take the number of
\textbf{resolve} they would, in a standard \textbf{round}, score against
you, and instead subtracts them from the \textbf{resolve} scored against
them in the \textbf{round} that removed them from the \textbf{contest}.
If your opponent now has positive \textbf{resolve}, they return to the
\textbf{sequence}, and may re-engage you. Your GM describes this as a
dramatic turnaround, in which your overreaching has somehow granted them
an advantage allowing them to recover from their previous misfortune.

Where it makes sense, unengaged PCs may attempt \textbf{parting shots}
against opponents taken out of the \textbf{sequence} by someone else.
You may not revive your teammates by using your lamest abilities to make
\textbf{parting shots} on them; this, by definition, does not pass a
\textbf{credibility test}.

\hypertarget{assists-3}{%
\subsubsection{5.4.9 Assists}\label{assists-3}}

An unengaged PC or follower may attempt to help another PC recover
\textbf{resolve}.

\hypertarget{joining-an-in-progress-contest-3}{%
\subsubsection{5.4.10 Joining an In-Progress
Contest}\label{joining-an-in-progress-contest-3}}

On a \textbf{round} of a \textbf{group chained sequence}, if you are
otherwise unengaged you can engage with any opponent who remains. As
\textbf{resolve} only changes with \textbf{recovery} both participants
use their existing \textbf{resolve}.

\hypertarget{recovery-1}{%
\subsubsection{5.4.11 Recovery}\label{recovery-1}}

Without intervention, your PC will recover, given time. Your GM should
decide at what point your character recovers their resolve and can
return to play. As a guideline: one or two points of resolve heal with a
short rest, good meal, or time spent relaxing with friends; three or
four points of resolve requires a longer period of recuperation, therapy
or medical assistance; five points of resolve requires a long period of
peaceful rest, away from conflict, to heal.

You may decide that you cannot wait for time to restore your PC, and
instead want to use \textbf{abilities} to accelerate your
\textbf{recovery}.

\hypertarget{recovery-abilities-1}{%
\paragraph{5.4.11.1 Recovery Abilities}\label{recovery-abilities-1}}

When deciding what \textbf{tactic} to use for \textbf{recovery},
credible choices depend on the nature of the conflict.

\begin{itemize}
\tightlist
\item
  Medical \textbf{abilities} both conventional and incredible, such as
  first aid, trauma surgery, regenerative powers or AI doctors, can be
  used to recover from physical injuries.
\item
  Psychological \textbf{abilities} both conventional and incredible,
  such as therapy, psychiatry, meditation or telepathy, can be used to
  recover from mental injuries.
\item
  Social \textbf{abilities} both conventional and incredible, such as
  leadership, charisma, magical glamours or familiarity with social
  networks can help recovery from social injuries.
\item
  Engineering \textbf{abilities} such as mechanical repair, fusion
  engineering or blacksmithing can be used to repair damage to
  equipment.
\item
  \textbf{Abilities} that represent beliefs, convictions, or ties to a
  community can be used in multiple situations to recover, representing
  the emotional ties that keep your PC going despite adversity. Your GM
  is encouraged to be liberal when allowing the usage of emotional ties
  to recover, provided their usage is `fresh' within an episode.
\end{itemize}

\hypertarget{recovery-sequences}{%
\paragraph{5.4.11.1 Recovery Sequences}\label{recovery-sequences}}

Any attempt to recover is also a \textbf{chained sequence}. Your GM must
decide how long each round takes in game time - the limitation on
attempts to restore \textbf{resolve} is usually time.

On a \textbf{victory} you restore \textbf{recovery} equal to the
\textbf{degrees of victory} plus \emph{one} (see 5.4.11.2).

On a \textbf{defeat} your GM has the following options:

\begin{itemize}
\tightlist
\item
  Lost time is the only penalty - use this when there is little risk to
  helping others.
\item
  The helping PC loses \textbf{resolve} - use this when there is a risk
  of the helping PC becoming exhausted or otherwise drained by their
  efforts.
\item
  The PC being helped loses further \textbf{resolve} - use this when any
  intervention carries a risk of worsening the position such as when
  attempting to fix losses to relationships or mental health.
\end{itemize}

As with all \textbf{resolve} losses the PC losing \textbf{resolve} may
opt to take a \textbf{consequence} instead, and the recovery
\textbf{sequence} ends immediately. If it is credible, then
\textbf{resolve} losses during recovery may also be transferred to
\textbf{followers}.

The resistance for a recovery roll is always the \textbf{default}
resistance.

When nested within another sequence the first \textbf{recovery} roll
faces the \textbf{base resistance}. Each subsequent \textbf{recovery}
attempt to the same beneficiary increases the \textbf{resistance} by +5.
The \textbf{resistance} escalation occurs even when another PC steps in
to make a subsequent \textbf{recovery}. This escalation allows the
occasional dramatic rescue but makes it difficult for players to prolong
losing battles to excruciating length. Increasing attempts to bring
someone back from exhaustion become harder.

\hypertarget{recovery-outcome-table}{%
\paragraph{5.4.11.2 RECOVERY OUTCOME
TABLE}\label{recovery-outcome-table}}

\begin{longtable}[]{@{}cccc@{}}
\toprule
Degrees of Victory & Gain & Degrees of Defeat & Loss \\
\midrule
\endhead
0 & 1 & 0 & 1 \\
1 & 2 & 1 & 2 \\
2 & 3 & 2 & 3 \\
3 & 4 & 3 & 4 \\
4 & 5 & 4 & 5 \\
\bottomrule
\end{longtable}

\hypertarget{exhaustion}{%
\paragraph{5.4.11.3 Exhaustion}\label{exhaustion}}

If your resolve becomes \textbf{exhausted} (reaches \emph{zero}
resolve), you return to play with your starting \textbf{resolve}
permanently reduced by \emph{one}. This is the impact of trauma on your
reserves to cope with further stresses.

When your starting \textbf{resolve} becomes \emph{zero} your PC succumbs
to the strains of an adventurous life and must retire from play, perhaps
dead, perhaps in exile, perhaps incapacitated. It is time to create a
new character.

You might decide to retire your PC earlier - as your starting
\textbf{resolve} lowers your PC's ability to cope with a life of
adventure diminishes.

\hypertarget{wagered-sequences-vs-scored-sequences-vs-chained-sequences}{%
\subsection{5.5 Wagered Sequences vs Scored Sequences vs Chained
Sequences}\label{wagered-sequences-vs-scored-sequences-vs-chained-sequences}}

Your GM chooses ONE form of \textbf{sequence} for their game, and sticks
to it. The \textbf{sequence} rules presentation here is modular: your
GMs should choose the one that matches their genre.

\textbf{Scored sequences} are fast and simple, and we consider them the
default. \textbf{Wagered sequences} have the drama of wagering and tend
to encourage high-octane stunts, but can take longer to resolve.
\textbf{Chained sequences} make all \textbf{contests} a sequence, where
resolve is slowly worn down.

You can think of this as a continuum. At one end are gritty genres where
you want to use a \textbf{chained sequence} to reflect how punishing
conflict is. At the other end are gonzo, larger-than-life genres where
you want to use a \textbf{wagered sequence} to encourage crazy stunts
and outrageous action. In the middle is the \textbf{scored sequence}
which lets you focus in on tasks to add suspense and drama, without
being too grim or too over-the-top.

If in doubt, use a \textbf{scored sequence} by default.

\hypertarget{lengthy-sequences}{%
\subsection{5.7 Lengthy Sequences}\label{lengthy-sequences}}

There's no particular time scale associated with \textbf{sequences}. But
some \textbf{sequences} may by their very nature be a drama that can't
be resolved at one point in the narrative. Examples include political
campaigns, construction projects, or seductions. These can be resolved
by \textbf{sequences} where each \textbf{round} is conducted at an
appropriate moment, rather than in close succession.

Your GM will need to keep track of the \emph{tally}. They need to keep
the running total for \textbf{advantage points} or \textbf{resolution
points}, and perhaps track running totals for \textbf{consequences} or
\textbf{benefits} in a \textbf{chained sequence} if they only impact
future \textbf{rounds} in the \textbf{sequence}. They will also have to
track the \textbf{resistance}, though this might change as the context
changes (a civil war started by the players could impede their
castle-building plans).

The challenges of each round will vary, and you may use a different
\textbf{ability} or \textbf{augment} in the each round.

\hypertarget{relationships}{%
\section{6.0 Relationships}\label{relationships}}

Abilities may represent your relationship to NPCs.

\hypertarget{supporting-characters}{%
\subsection{6.1 Supporting Characters}\label{supporting-characters}}

Many relationships connect you to NPCs controlled by the GM.

When you try to use one of these relationships to solve a problem, your
\textbf{tactic} is your relationship \textbf{ability}. You can't simply
go to the \textbf{supporting character} you have a relationship with,
stick them with the problem, and expect to see it solved.

If you succeed, the \textbf{supporting character} helps you solve the
problem. If you fail, they don't. As with any \textbf{ability}, you must
still specify how the NPC goes about overcoming the \textbf{story
obstacle} or answering the \textbf{story question}. Calls on
relationships are almost always \textbf{contests}.

In crucial situations, it may seem dramatically inappropriate for you to
solve a problem indirectly, by working through others.

You may expose the \textbf{supporting character} to serious risk. When
\textbf{supporting characters} undertake significant risk, the
\textbf{supporting character} may suffer a \textbf{consequences}
commensurate with the \textbf{degree} of the \textbf{defeat} in the
\textbf{contest}. Or it may simply be your relationship that is damaged
or destroyed. Your GM should feel more at liberty to frame a contest
with \textbf{supporting character} death, exile, or breakdown as an
outcome than with a PC. If the character dies or otherwise suffers a
change of status that renders them useless to you, you lose use of the
relationship \textbf{ability} until your next \textbf{advance} (see
§8.2) where you can replace them. Your GM should work with you to
introduce a replacement at an appropriate moment in the fiction. Before
putting \textbf{supporting characters} at serious risk, your GM should
make sure the players understand the magnitude of the possible
consequences.

\hypertarget{allies}{%
\subsection{6.2 Allies}\label{allies}}

An \textbf{ally} is a character of roughly the same level of
accomplishment as you, often in the same or a similar line of work. For
every favor you ask of them they'll ask one of you. These reciprocal
favors will be roughly equivalent in terms of risk, time commitment,
resistance class, and inconvenience.

\hypertarget{patrons}{%
\subsection{6.3 Patrons}\label{patrons}}

\textbf{Patrons} enjoy greater access to assets than you, either through
personal ownership (as in a wealthy entrepreneur or rich aristocrat) or
authority (as in the governor of a state or province or the head of an
organization). They may lend you advice or provide you with assets but
are too busy and important to personally perform tasks for you. They may
hire you to do jobs, or issue orders within a command structure to which
you both belong.

When you roll your \textbf{patron} relationship, your GM adjusts the
\textbf{resistance} depending on what you have done for them lately.

\hypertarget{contacts}{%
\subsection{6.4 Contacts}\label{contacts}}

A \textbf{contact} is a specialist in an \textbf{occupation}, skill, or
area of expertise. \textbf{Contacts} provide you information and perform
minor favors, but will expect information or small favors from you in
return.

You can describe a \textbf{contact} as being a particular individual, or
as a group of similar individuals.

\hypertarget{occupational-contacts}{%
\subsubsection{6.4.1 Occupational
Contacts}\label{occupational-contacts}}

Any \textbf{occupational keyword} can be treated as a source of
\textbf{contacts}. However, using an \textbf{occupational keyword} as a
source of \textbf{contacts} will always be a \textbf{stretch} (see
§2.5). To more reliably draw on particular \textbf{contacts} associated
with your occupation, you should take a \textbf{breakout ability} under
the \textbf{occupational keyword}.

\hypertarget{followers-2}{%
\subsection{6.5 Followers}\label{followers-2}}

A \textbf{follower} is a \textbf{supporting character} that travels with
you and contributes on a regular basis to your success.

There are two types of followers: \textbf{sidekicks} and
\textbf{retainers}.

\textbf{Followers} need not be people, or even sentient beings: you can
write up a spirit guardian, trusty robot, or companion animal as a
\textbf{follower}.

\hypertarget{sidekick}{%
\subsubsection{6.5.1 Sidekick}\label{sidekick}}

A \textbf{sidekick} is a \textbf{supporting character} under your
control. Most of the time they stay at your side to render assistance,
but they can also go off and perform errands or missions on their own.

You should give your \textbf{sidekick} a name. You should, when asked,
explain how the \textbf{sidekick} came to be your \textbf{follower}, and
why they continue in that role.

\textbf{Sidekicks} start with three \textbf{abilities}, one rated at 15
and the others at 10. Any of these \textbf{abilities} may be a
\textbf{keyword}. At least one of them should indicate a
\textbf{distinguishing characteristic}.

If the sidekick is nonhuman or a member of an unusual culture, one of
its three starting \textbf{abilities} must be its species or culture
\textbf{keyword}.

Once you have determined the \textbf{sidekick's} base
\textbf{abilities}, you can allocate 10 \textbf{improvement points}
amongst them, as described in Assigning Ability Ratings (see \$3.4).

You can improve these \textbf{abilities} through the expenditure of
\textbf{experience points}.

You may use any of your \textbf{sidekick's abilities} as your own. The
\textbf{sidekick} can go off and do things without you.

\hypertarget{retainers}{%
\subsubsection{6.5.2 Retainers}\label{retainers}}

A \textbf{retainer} is a more or less anonymous servant or helper. You
may specify a single \textbf{retainer}, or, where appropriate to your
character concept, an entire staff of them.

Like any other \textbf{ability}, a \textbf{retainer} \textbf{ability}
allows you to overcome relevant \textbf{story obstacles} by engaging in
a \textbf{contest}. To model the contribution of \textbf{retainers},
when you are acting, you can use them to \textbf{augment} your
\textbf{ability}. Your GM can rule that \textbf{consequences} apply to
\textbf{retainers}.

\textbf{Retainers} generally regard you with all the affection and
loyalty due to an employer or master. If you treat them more poorly than
is expected for their culture, your GM should increase the
\textbf{resistance} of attempts to make use of their talents.

\hypertarget{relationships-as-flaws}{%
\subsection{6.6 Relationships as Flaws}\label{relationships-as-flaws}}

Certain relationships with \textbf{supporting characters} act as
\textbf{flaws}. They impose obligations on you, prompting your GM to
present you with \textbf{story obstacles} you have no choice but to
overcome or \textbf{story questions} you need to answer. Your GM should
award you an \textbf{experience point} at the conclusion of any session
of play where you or your GM created dramatic complications for you via
a \textbf{relationship} that is a flaw.

\hypertarget{dependents}{%
\subsubsection{6.6.1 Dependents}\label{dependents}}

A \textbf{dependent} is a person, usually a family member or loved one,
who requires your aid and protection. Your GM should periodically create
storylines in which your \textbf{dependent} is endangered.

Rather than taking a \textbf{dependent} as a \textbf{flaw}, you may find
it more fruitful to specify the nature of your relationship as an
\textbf{ability}, such as \emph{Love for Wife} or \emph{Love for Son}.

\hypertarget{adversaries}{%
\subsubsection{6.6.2 Adversaries}\label{adversaries}}

An \textbf{adversary} is a rival, enemy or other individual who can be
relied upon to periodically disrupt your plans.

The \textbf{adversary's} goals are probably the opposite of yours,
although they could be a bitter rival within the same community,
organization, or movement.

To treat an \textbf{adversary} as an \textbf{ability}, rather than a
\textbf{flaw}, describe your emotional response to them. Examples:
\emph{Hates Leonard Crisp}, \emph{Fears the Electronaut}, \emph{Sworn
Vengeance Against Heimdall}. That way, you still inspire your GM to add
the plot elements you desire, but can use your antipathy toward the
enemy to \textbf{augment} your \textbf{target number}s against them.

\hypertarget{story-points}{%
\section{7.0 Story Points}\label{story-points}}

\emph{QuestWorlds'} design favors pulp stories and cinematic action.
\textbf{Story points} mirror the ability of heroes in these genres to
``cheat death'', or ``escape with one bound''.

Normally, your GM should ensure that \textbf{defeat} takes the story for
your PC in an interesting new direction. Unlike some games, where your
goal is to win against challenges set by the GM, in a storytelling game
your goal is to tell a good story together. Just as in fiction the
protagonist can suffer all sorts of reversals, so in a storytelling
game, your PC should suffer all sorts of adversities before they triumph
(or meet their tragic end). As a result, we recommend against the
tendency to `buy off \textbf{defeat}' with \textbf{story points} in the
middle of the story. Instead, use \textbf{story points} when
\textbf{defeat} would damage the conception of the character that you
have been building during the story, or lead to an unsatisfactory climax
to the story.

Your GM should push the story in an interesting new direction on
\textbf{defeat} not send it to a dead end. If there is no interesting
branch from \textbf{defeat} they should consider an \textbf{assured
contest} instead.

In other genres, it may feel less appropriate that you can `cheat
certain death.' For those genres you can simply drop \textbf{story
points} without impacting the game.

In games with a strong player vs.~player element, your GM should
dispense with \textbf{story points} as they become disruptive if used
against each other.

\hypertarget{story-point-pool}{%
\subsection{7.1 Story Point Pool}\label{story-point-pool}}

At the beginning of play, your GM will create a \textbf{story point
pool} for your group. The \textbf{story point pool} has one
\textbf{story point} per PC.

During play you can \textbf{burn} one or more points from this
\textbf{pool}, after which it is lost. You can decide to spend
\textbf{story points} at any time. You do not need agreement from the
other players to do so.

\hypertarget{refreshing-story-points}{%
\subsubsection{7.1.1 Refreshing Story
Points}\label{refreshing-story-points}}

Because you burn a \textbf{story point} to use it, your \textbf{story
point pool} may become exhausted. The GM has three choices for
\textbf{refreshing} your \textbf{story point pool:}

\begin{itemize}
\tightlist
\item
  The \textbf{story point pool refreshes} at the beginning of every
  session of play.
\item
  The \textbf{story point pool refreshes} whenever your PCs engage in
  genre-appropriate downtime. Usually the GM plays this out as a
  montage, asking your character to describe genre appropriate
  activities in this time period. For example: in a police procedural
  series, the PCs might gather at a cop bar to drink and talk about
  their personal problems; in a series about high-school paranormal
  investigators they might gather in the school library to chill with
  their mentor, the librarian, and talk about teenage problems.
\item
  The \textbf{story point pool refreshes} whenever the GM deems it
  necessary, based on their desire to allow you to edit the upcoming
  story.
\end{itemize}

Ultimately your GM is always the arbiter of when and how the
\textbf{story point pool refreshes}. On a refresh your \textbf{story
points} pool resets to one \textbf{story point} per PC.

\hypertarget{story-point-pool-summary}{%
\subsubsection{7.1.2 Story Point Pool
Summary}\label{story-point-pool-summary}}

To summarize:

\begin{itemize}
\tightlist
\item
  At the beginning of a session you have 1 \textbf{story points} per PC
  in the pool.
\item
  During the session you may \textbf{burn} \textbf{story points}.
\item
  \textbf{Story points} that are burned are lost from the \textbf{story
  point pool}.
\item
  The GM decides on the conditions to a refresh a \textbf{story point
  pool}.
\item
  The \textbf{story point pool} refreshes to 1 \textbf{story point} per
  PC in the pool.
\end{itemize}

\hypertarget{success-with-a-story-point}{%
\subsection{7.2 Success with a Story
Point}\label{success-with-a-story-point}}

You can burn a \textbf{story point} to gain an additional
\textbf{success} (see §2.3.6)

\hypertarget{plot-edits}{%
\subsection{7.3 Plot Edits}\label{plot-edits}}

\emph{QuestWorlds} is a co-operative game, and you may create details
about the setting as the normal part of narration. Your GM should allow
this, as long as they do not break credibility. So, you may describe
your PC walking over to the pot of soup bubbling on the fire, swiping a
drink from the tray the waiter is carrying at the governor's ball, or
taking the monorail to the next city to continue your investigation.
Your GM should allow these additions without interruption, providing it
does not confer significant advantage to your PC. Mostly this will be
using elements that have already been established as part of the
setting.

A \textbf{plot edit} is a more significant moment of good fortune that
you wish to narrate, that provides advantage to your PC. You are not
just describing something that is plausible in the environment, but
something whose existence aids you in overcoming \textbf{story
obstacles} or revealing the answer to a \textbf{story question} .

A \textbf{plot edit} might be thought of as `fate' or `luck.'

Burning \textbf{story points} for a \textbf{plot edit} allows you to
modify the setting or environment in your PC's favor. The chance
encounter in the street with an NPC, favorable weather, car keys in the
sun visor, the forthcoming eclipse, the wind that fills the sails.

Your GM is the arbitrator of whether a \textbf{plot edit} is allowed. It
should not suspend the disbelief of the other players in the game or
setting or hamper their enjoyment. It should not derail or short-circuit
the game's entertainment. The \textbf{plot edit} should, by contrast, be
something that enhances the story for all the players.

The cost, in \textbf{story points}, of a \textbf{plot edit}, is given by
the following table.

\hypertarget{plot-edit-table}{%
\paragraph{7.3.1.1 PLOT EDIT TABLE}\label{plot-edit-table}}

\begin{longtable}[]{@{}
  >{\centering\arraybackslash}p{(\columnwidth - 6\tabcolsep) * \real{0.3125}}
  >{\centering\arraybackslash}p{(\columnwidth - 6\tabcolsep) * \real{0.0625}}
  >{\centering\arraybackslash}p{(\columnwidth - 6\tabcolsep) * \real{0.3125}}
  >{\centering\arraybackslash}p{(\columnwidth - 6\tabcolsep) * \real{0.3125}}@{}}
\toprule
\begin{minipage}[b]{\linewidth}\centering
Level
\end{minipage} & \begin{minipage}[b]{\linewidth}\centering
Cost
\end{minipage} & \begin{minipage}[b]{\linewidth}\centering
Impact
\end{minipage} & \begin{minipage}[b]{\linewidth}\centering
Example
\end{minipage} \\
\midrule
\endhead
Marginal & 1 & A substantive change that does not alter the situation
but offers an alternate avenue for resolution & The gate guard at the
secret government facility tonight is an old war buddy established by
the PC in a prior scene and cemented as a relationship \\
Minor & 2 & A substantive change that does not flow from previously
established facts in the story. A \emph{deus ex machina} change & The XO
of the Patrol ship is an old drinking buddy of your PC, a fact not
previously established in play \\
Major & 3 & A stroke of good fortune that is unrelated to prior events
and resolves a conflict or reveals a secret & The vampire has failed to
notice the approaching sun rise, which disintegrates them just as they
are about to drain the incapacitated PC \\
\bottomrule
\end{longtable}

\hypertarget{experience}{%
\section{8.0 Experience}\label{experience}}

During a session of play your character will have the chance to learn
from experience or overcoming personal obstacles. When your character
learns, they gain a \textbf{experience points}. \textbf{Experience
points} can be used to improve your character.

\hypertarget{earning-experience-points}{%
\subsection{8.1 Earning Experience
Points}\label{earning-experience-points}}

You gain one \textbf{experience points} for any of the following:

\begin{itemize}
\tightlist
\item
  When your \textbf{outcome} for a \textbf{contest} is a
  \textbf{defeat}.
\item
  Your GM uses a \textbf{flaw} or other \textbf{ability} against you in
  a contest with you (see §2.6). This happens either when the story
  forced you to confront a \textbf{flaw}, or the GM gave you a
  \textbf{hindrance} (see §3.4), if the \textbf{hindrance} results in a
  \textbf{penalty}.
\end{itemize}

Note the following restrictions:

\begin{itemize}
\tightlist
\item
  You only gain an \textbf{experience point} for each of your
  \textbf{abilities} or \textbf{flaws} once in a session of game play.
\item
  You do not get \textbf{experience points} for an \textbf{augment},
  \textbf{AP gifting} or \textbf{assist}.
\item
  You do not gain an \textbf{experience point} from an \textbf{assured
  contest}, even if you roll to determine \textbf{benefits} or
  \textbf{consequences}.
\end{itemize}

You can gain a maximum of five \textbf{experience points} in any one
session. Once you have earned five \textbf{experience points}, you
cannot gain further \textbf{experience points} in that session.

\hypertarget{experience-on-defeat}{%
\subsubsection{8.1.1 Experience on Defeat}\label{experience-on-defeat}}

Awarding \textbf{experience points} on \textbf{defeat} is a
self-correction mechanism.

\begin{itemize}
\tightlist
\item
  It slows your advance if your PC regularly outclass the
  \textbf{resistance}. This pushes your GM to introduce threats that
  \textbf{credibly} present a greater threat to your PC.
\item
  If you regularly buy off \textbf{defeat} with \textbf{story points}
  you will find it harder to advance. In \emph{QuestWorlds} your GM
  should provide an entertaining story branch on defeat; you should not
  need to buy \textbf{defeat} off, unless it damages your character
  conception or is the climax.
\end{itemize}

If the GM finds that the PCs are no longer regularly earning
\textbf{experience points} they can consider using \textbf{resistance
progression} (see §2.13) to increase the \textbf{base resistance} so
that more \textbf{contests} will feature a high enough resistance to
earn \textbf{experience points}.

\hypertarget{improving-your-character}{%
\subsection{8.2 Improving Your
Character}\label{improving-your-character}}

When you accumulate 10 \textbf{experience points}, you can buy an
\textbf{advance}. An \textbf{advance} allows you to select two of the
following. You cannot choose an option more than once, unless it is
repeated.

\begin{itemize}
\tightlist
\item
  10 \textbf{improvement points} across standalone \textbf{abilities}.
\item
  10 \textbf{improvement points} across standalone \textbf{abilities}.
\item
  5 \textbf{improvement points} across \textbf{keywords}.
\item
  5 \textbf{improvement points} across \textbf{keywords}.
\item
  increase an existing breakout \textbf{ability} by +5.
\item
  a new breakout \textbf{ability} at + 5.
\item
  a new standalone \textbf{ability} at 10.
\item
  Turn a stand-alone \textbf{ability} into a \textbf{keyword} by adding
  a new +5 breakout \textbf{ability} to it.
\item
  Drop a \textbf{flaw}, or turn it into an \textbf{ability} if story
  appropriate and agreed with the GM.
\item
  Replace a \textbf{supporting character} who has been lost (see §6.1).
\end{itemize}

In addition, if you have less than three \textbf{flaws}, you may add
another, provided it fits the story, when you take an \textbf{advance}.

You may spend \textbf{improvement points} immediately, or in play, even
after a roll. You may not spend \textbf{improvement} points to increase
\textbf{breakouts} only \textbf{keywords} or standalone
\textbf{abilities}.

In some genres you may wish to maintain a tally of the total
\textbf{experience points} earned as a measure of your reputation.

\hypertarget{rate-of-advancement}{%
\subsubsection{8.2.1 Rate of Advancement}\label{rate-of-advancement}}

We assume an average earning rate of two \textbf{experience points} per
session. This would lead to you gaining an advance every five sessions.
If your rate is lower than one \textbf{experience point} a session, your
GM should choose one of these options:

\begin{itemize}
\tightlist
\item
  Provide more credible threats
\item
  Use \textbf{resistance progression}
\item
  Reduce the cost of an \textbf{advance} to five \textbf{experience
  points}.
\end{itemize}

\hypertarget{directed-improvements}{%
\subsubsection{8.2.2 Directed
Improvements}\label{directed-improvements}}

On occasion your GM may increase one of your \textbf{abilities}, by +5,
+10 or +15, or give you a new \textbf{ability}, usually rated at 10.
These are called \textbf{directed improvements}.

\textbf{Directed improvements} are usually rewards for overcoming
particularly important or dramatic \textbf{story obstacles} or answering
a dramatically important \textbf{story question}.

Your GM will tend to use them to raise \textbf{abilities} that would
otherwise fall behind, but should increase due to story logic, or
introduce new \textbf{abilities} for the same reason.

Your GM might give you a new \textbf{flaw} to represent a story outcome
from a contest, that leads you with a hindrance to future action. If you
have three or more \textbf{flaws} you can ask your GM to drop one in
favor of the new \textbf{flaw}, if you it seems story appropriate.

\hypertarget{timing-of-improvements}{%
\subsubsection{8.2.3 Timing of
Improvements}\label{timing-of-improvements}}

Your improvements happen immediately, when you cross the threshold to
buy an \textbf{advance}, or a GM awards you a \textbf{directed
improvement}.

\hypertarget{milestone-improvements}{%
\subsection{8.3 Milestone Improvements}\label{milestone-improvements}}

Your GM may decide that they do not want to track \textbf{experience
points} earned during a game. In this case they may switch to
\textbf{milestone improvement}.

Under \textbf{milestone improvements} the GM simply declares that your
PCs have reached a point in the story where we should see them improve
their \textbf{abilities} and award you an \textbf{advance} (see §8.2).

Your GM should not use both \textbf{experience points} and
\textbf{milestone improvements} but choose one. If in doubt, choose
\textbf{experience points} as the default. \textbf{Milestone
improvements} do not naturally balance against the \textbf{resistance}
and the GM may need to use \textbf{resistance progression} to continue
to up the threat level against your PCs (see §2.8).

\hypertarget{community-resources-and-support}{%
\section{9.0 Community Resources and
Support}\label{community-resources-and-support}}

Some series revolve around the relationship between a band of
influential figures and the community they protect. In defense of the
community, they can \textbf{bolster}, expend, and juggle its various
\textbf{resources}.

These rules allow your GM to track the rise and fall of the fortunes of
your community, and your impact on them.

If your GM intends to play a game centered around a community, you
should have a \textbf{keyword} for that community.

It is possible that you have relationships with other communities that
are not the focus of play. Treat these \textbf{abilities} as
relationships that you can call on, but your GM should not track these
communities with those rules. Your GM should pick the level of community
that provides the greatest dramatic potential from its competition for
\textbf{resources}, friendly or otherwise, with its rivals.

Some campaigns do not center on a community, with the adventurers being
footloose wanderers. In that case, even if you have community
\textbf{abilities}, your GM will not track any community. Before your GM
decides this, they should consider where your PCs might turn for help,
succor, or aid. Is there somewhere in the campaign defined as a place of
refuge and safety for you? It may well be that is a community. For
example, the bar where other footloose adventurers all meet, who will
help each other out in a tight spot.

\hypertarget{community-design}{%
\subsection{9.1 Community Design}\label{community-design}}

\hypertarget{defining-resources}{%
\subsubsection{9.1.1 Defining Resources}\label{defining-resources}}

Communities have \textbf{resources} that your GM defines. Your PC can
try to draw on their community's \textbf{resources} to use them as
\textbf{bonus}. If your community is in difficulty, a strained
\textbf{resource} might act as a \textbf{penalty}. Your GM should focus
on no more than five or so broadly-labeled \textbf{resource} types, so
that the PCs can care about (and have a chance of successfully managing)
all of them.

Most communities have variants of the following \textbf{resources},
perhaps with more colorful names:

\begin{itemize}
\tightlist
\item
  Wealth --- the capacity of the community to provide financial help,
  whether counted primarily in dollars, credits, or cattle
\item
  Diplomacy --- the relationships with other groups through which a
  community can obtain favors, while minimizing the cost of its
  reciprocal obligations
\item
  Morale --- the community's resolve to achieve its goals, and
  willingness to follow the directives of its leaders
\end{itemize}

The following abilities might appear, depending on setting:

\begin{itemize}
\tightlist
\item
  Military --- its capacity to defend itself from outside threats, and
  to aggressively achieve its own aims through force of arms (for
  settings where communities of the size you're tracking field their own
  armed units)
\item
  Magic --- the capability of a community to perform supernatural acts
  (for fantasy worlds)
\item
  Technology --- its access to specialized, rare or secret devices or
  scientific knowledge not shared by its rivals (for post- apocalyptic
  or SF worlds)
\end{itemize}

Similar communities in the genre, should have the same set of
\textbf{resources}.

\hypertarget{rating-resources}{%
\subsubsection{9.1.2 Rating Resources}\label{rating-resources}}

Your GM distributes each of bonuses of: 5, 10, 15, and 20 between the
four of the five \textbf{resources}. The last resource has no
\textbf{bonus}. Note that the size of the group doesn't affect the
\textbf{bonuses}.

\hypertarget{community-questionnaires}{%
\subsubsection{9.1.3 Community
Questionnaires}\label{community-questionnaires}}

Your GM may create a questionnaire that asks the players to make choices
about the priorities of their community. The responses to each question
should be multiple-choice. Each choice you make adds points to a score
for each \textbf{resource} type. Points are awarded according to what
the answer reveals about the community's relative priorities. An answer
may give points to more than one \textbf{resource}.

You can choose your answers by consensus, majority vote, or take turns.

When you're done, rank the \textbf{resources} in the order of the
scores. Assign the high \textbf{bonuses} to the highest scores and the
lowest to the low.

A questionnaire also introduces your setting in a punchy, interactive
format, and tailors the community to the players' desires, increasing
their investment in it.

\hypertarget{drawing-on-resources}{%
\subsection{9.2 Drawing on Resources}\label{drawing-on-resources}}

You can use community \textbf{resources} as a \textbf{bonus} to your
\textbf{abilities} after convincing the community to let you expend
precious assets. This requires a preliminary \textbf{contest} using a
social \textbf{ability}, most likely your community \textbf{keyword}.
Your GM will use a \textbf{moderate resistance} as the baseline, with
higher \textbf{resistance}s when your proposals seem selfish or likely
to fail, and lower ones when everyone but the dullest dolt would readily
see their collective benefits. Your GM may increase \textbf{resistance}s
if your group draws constantly on community \textbf{resources} without
replenishing them.

The lobbying effort and the actual resource use require framing, a clear
description of what you are doing, and other details to bring them to
fictional life. You cannot use \textbf{resource abilities} directly, but
as an \textbf{bonus} to your own \textbf{abilities}.

Use of community \textbf{resources} should pass the threshold for being
\emph{memorable} and \emph{entertaining}. Normally there should be a
clear benefit to the community, or risk to the community. The PC's
actions should be in support of the community, not themselves. Community
involvement becomes part of the story. A certain amount of routine
support for your character is assumed; a \textbf{bonus} implies that the
community is expending abnormal effort on your behalf, that will cost
the community itself.

\hypertarget{resource-depletion}{%
\subsubsection{9.2.1 Resource Depletion}\label{resource-depletion}}

Unlike character abilities, each use of community \textbf{resources}
temporarily \textbf{depletes} it. Regardless of \textbf{outcome} a
\textbf{resource} takes a \textbf{penalty} of -5 when used. Effectively,
this reduces the \textbf{bonus} by 5.

Your GM decides when a \textbf{resource} is restored to its original
value. Your GM should decide what the credible interval is for the
community to recover from the expenditure of effort. At that point, your
GM restores the \textbf{bonus} for the \textbf{resource}.

You might chose to use a \textbf{resource} when it is already depleted,
in which case you use it at its reduced value. Your GM may use this to
represent attrition to your community from a continued struggle. A
\textbf{resource} that is depleted enough to fall below \emph{zero}
becomes a \textbf{penalty}.

Threats to community \textbf{resources} act as a spur to PC action. Your
GM may rule that the \textbf{outcome} from a \textbf{contest} where you
did not use the \textbf{resource} may still deplete a community
\textbf{resource}.

\hypertarget{required-resource-use}{%
\subsubsection{9.2.2 Required Resource
Use}\label{required-resource-use}}

As part of your GM's setting design, they may specify that certain
actions in a setting always require the use of a community
\textbf{resource}. Because the \textbf{resource} use is obligatory, it
need not meet the usual criteria for entertainment value.

\hypertarget{resource-as-a-penalty}{%
\subsubsection{9.2.4 Resource as a
Penalty}\label{resource-as-a-penalty}}

A \textbf{resource's} \textbf{bonus} may fall below 0. If you require
use of a community's \textbf{resources} (see §9.2.2) your actions will
be subject to a \textbf{penalty}.

\hypertarget{bolstering-resources}{%
\subsubsection{9.2.4 Bolstering Resources}\label{bolstering-resources}}

Your GM may offer you the opportunity to \textbf{bolster} a community
\textbf{resources} ahead of need by seeking out and overcoming relevant
\textbf{story obstacles}. If you succeed, the community resource
improves by +5. Your GM will set the \textbf{resistance} for the
\textbf{bolster}. The community's higher ranked resources should have
higher \textbf{resistances} to \textbf{bolstering}. As a default, use
the current \textbf{bonus} as the \textbf{modifier} to \textbf{base
resistance} as the \textbf{TN} for the \textbf{bolstering}.

For clarity, a \textbf{resource} rated at +M can be bolstered to +M2.

\textbf{Bolstering} lasts until the \textbf{resource} is used. When your
GM depletes a \textbf{bolstered resource} following usage, they remove
only the additional +5 from \textbf{bolstering}.

If a \textbf{resource} is already suffering from a \textbf{penalty},
bolstering removes that \textbf{penalty} instead of improving it by +5.

\hypertarget{background-events}{%
\subsubsection{9.2.5 Background Events}\label{background-events}}

In the background all sorts of other events periodically alter the
community's prosperity. These include the actions of other community
members, who are \textbf{depleting and bolstering resources} all the
time, as well as the unexpected intrusion of outside forces.

Your GM may decide that the community's \textbf{bonus} in a
\textbf{resource} is temporarily at a higher or lower due to these
outside events. Your GM decides when the \textbf{resource} returns to
normal. This may require you to overcome a \textbf{story obstacle}.

\hypertarget{appendix}{%
\section{10.0 Appendix}\label{appendix}}

\hypertarget{glossary-of-terms}{%
\subsection{10.1 Glossary of Terms}\label{glossary-of-terms}}

\begin{description}
\tightlist
\item[\textbf{Ability}]
Anything you can apply to solve a problem or overcome an obstacle.
\item[\textbf{Advance}]
A package of improvements to your \textbf{abilities} and
\textbf{keywords} earned through \textbf{experience points} or
\textbf{milestone advancement}.
\item[\textbf{Advantage Point (AP)}]
A measure of advantage in a \textbf{wagered sequence}.
\item[\textbf{Ally}]
A \textbf{supporting character} of roughly equal ability to your own.
\item[\textbf{AP}]
Abbreviation for Advantage Point.
\item[\textbf{AP Gifting}]
When you help another character, whilst uninvolved in a
\textbf{contest}, by giving them \textbf{advantage points} in a
\textbf{wagered sequence}.
\item[\textbf{AP Lending}]
When you help another character, whilst engaged in a \textbf{contest},
by lending them \textbf{advantage points}, in a \textbf{wagered
sequence}.
\item[\textbf{Asymmetrical Exchange}]
In a \textbf{wagered sequence}, where you are pressed by an opponent,
but want to do something other than contend directly for the
\textbf{prize}.
\item[\textbf{Asymmetrical Round}]
In a \textbf{scored sequence}, where you are pressed by an opponent, but
want to do something other than contend directly for the \textbf{prize}.
\item[\textbf{Assist}]
In a \textbf{scored sequence}, if you are unengaged you may use an
\textbf{assist} to reduce the \textbf{resolution points} scored against
another character.
\item[\textbf{Augment}]
Using one \textbf{ability} to help another \textbf{ability}.
\item[\textbf{Assured Contest}]
You have an appropriate \textbf{ability} and the GM feels
\textbf{failure} is not interesting, or makes the PC looks un-heroic.
\item[\textbf{Background Event}]
An off-stage \textbf{bonus} or \textbf{penalty} applied to a
\textbf{resource}.
\item[\textbf{Base resistance}]
The \textbf{TN} for a \textbf{moderate resistance class}, from which all
other \textbf{resistance classes} are figured as a \textbf{bonus} or
\textbf{penalty}.
\item[\textbf{Benefit of Victory}]
Long term positive \textbf{bonus}, because you won a \textbf{contest},
against a challenging opponent (not -6 or less than your
\textbf{ability}). Usually a \textbf{state of fortune}.
\item[\textbf{Bolster}]
A \textbf{story obstacle} to apply a bonus to a community
\textbf{resource}
\item[\textbf{Bonus}]
A positive modifier.
\item[\textbf{Burn}]
Using a \textbf{story point} as a bump. The \textbf{story point} is lost
after burning.
\item[\textbf{Catch-Up}]
When you cross a \textbf{mastery} threshold you can increase lesser used
\textbf{abilities} to ensure they keep pace.
\item[\textbf{Contact}]
A \textbf{supporting character} who shares an \textbf{occupation} or
interest with your character.
\item[\textbf{Contest}]
Where there is uncertainty as to whether a PC can overcome a
\textbf{story obstacle} or discover a secret, then your GM can call for
a contest to determine if the PC succeeds or fails.
\item[\textbf{Consequences}]
Long term negative modifier, because you lost a contest. Usually a
\textbf{state of adversity}.
\item[\textbf{Contest Framing}]
Setting the stakes of the \textbf{contest}, what is this conflict about.
Often not the immediate aftermath of victory.
\item[\textbf{Credibility Test}]
Is it possible to perform the action without an \textbf{ability}, with
an ordinary \textbf{ability}, or only with a \textbf{incredible
ability}?
\item[\textbf{Crisis Test}]
Used to determine if a \textbf{resource} that has a \textbf{penalty}
creates a crisis.
\item[\textbf{Defeat}]
Your \textbf{result} is worse than the \textbf{resistance's} result.
\item[\textbf{Defensive Response}]
In a \textbf{scored sequence} you can choose a defensive \textbf{tactic}
which reduces the \textbf{resolution points} you lose on a negative
\textbf{result}.
\item[\textbf{Degree}]
The difference between the \textbf{successes} of the victor and the
loser in a \textbf{contest}.
\item[\textbf{Dependent}]
A \textbf{supporting character} who depends on your PC.
\item[\textbf{Depletion}]
Use of a community \textbf{resource} leads to its depletion.
\item[\textbf{Directed Improvement}]
When your GM grants you a new \textbf{ability}, or an increase to an
existing one, to recognize a story event.
\item[\textbf{Distinguishing Characteristic}]
The dominant personality \textbf{ability} that others recognize in a
character.
\item[\textbf{Exchange}]
In a \textbf{wagered sequence} a round is divided into two
\textbf{exchanges} where both aggressor and defender act. In a
\textbf{group wagered sequence} a round consists of a sequence of
\textbf{exchanges} where everyone acts in turn. The GM determines the
order of action.
\item[\textbf{Experience Points (XP)}]
When you experience \textbf{defeat}, or a \textbf{flaw} you may gain an
\textbf{experience point}, which accumulate between sessions.
\item[\textbf{Incredible ability}]
Certain genres allow player characters to have \textbf{abilities} that
exceed human norms, these are \textbf{incredible abilities}. A genre
pack normally outlines what is possible as part of its incredible powers
framework.
\item[\textbf{Failure}]
Rolling over your \textbf{target number}.
\item[\textbf{Final Action}]
A last action by a PC on 0 AP in a \textbf{wagered sequence}
\item[\textbf{Flaw}]
An \textbf{ability} that penalizes you instead of helping you.
\item[\textbf{Follower}]
A \textbf{supporting character} under your control. Either a
\textbf{sidekick} or \textbf{retainer}
\item[\textbf{Framing the contest}]
You and your GM agree on the \textbf{prize} for the victor, and your
tactic in trying to win it.
\item[\textbf{Group chained sequence}]
A \textbf{chained sequence} in which more than a pair of opponents
contend for the \textbf{prize}
\item[\textbf{Group scored sequence}]
A \textbf{scored sequence} in which more than a pair of opponents
contend for the \textbf{prize}
\item[\textbf{Group wagered sequence}]
A \textbf{wagered sequence} in which more than a pair of opponents
contend for the \textbf{prize}
\item[\textbf{Group Contest}]
A \textbf{contest} where one side has multiple participants.
\item[\textbf{Graduated Goals}]
When a contestant has a \textbf{primary} and \textbf{secondary} goal,
and may have to choose between them.
\item[\textbf{Story Point}]
Allows you to alter fate for a player character, either by a
\textbf{bump} to their \textbf{result} or a \textbf{plot edit}.
\item[\textbf{Keyword}]
A single \textbf{ability} that encompasses a range of abilities within
it, such as an \textbf{occupation} or culture. An \textbf{ability}
within an \textbf{umbrella keyword} is a \textbf{break-out ability}, an
\textbf{ability} within a \textbf{package keyword} is a
\textbf{stand-alone ability}.
\item[\textbf{Milestone Advancement}]
A method for improving a character where the GM declares when you
receive an \textbf{advance}.
\item[\textbf{Mastery}]
An \textbf{ability} \textbf{score} that rises above 20 is said to have a
\textbf{mastery}.
\item[\textbf{Mismatched Goals}]
When the opposing sides in a \textbf{contest} want different
\textbf{prizes}.
\item[\textbf{Modifier}]
A change to the PC's \textbf{target number} from the situation; the
strength of the \textbf{resistance} to the PC's actions.
\item[\textbf{Occupation}]
An \textbf{ability} that indicates the profession, or primary area of
expertise, of your character.
\item[\textbf{Outcome}]
A \textbf{contest} has an \textbf{outcome}, described as a
\textbf{victory} or \textbf{defeat} in obtaining the \textbf{prize} that
was agreed in \textbf{contest framing} for any PCs involved.
\item[\textbf{Parting Shot}]
An attempt to make your opponent's \textbf{defeat} worse in a
\textbf{sequence} (\textbf{scored} or \textbf{extended}), by `finishing
them off'.
\item[\textbf{Patron}]
A \textbf{supporting character} with superior assets.
\item[\textbf{Penalty}]
A negative modifier.
\item[\textbf{Prize}]
What is at stake in the \textbf{contest}, decided during
\textbf{framing}.
\item[\textbf{Rating}]
An ability has a \textbf{rating}, between 1 and 20, indicating how
likely a character is to succeed at using it.
\item[\textbf{Resistance}]
The forces opposing the PC in a conflict, or concealing a secret that
must be overcome by using an \textbf{ability} in a \textbf{contest}. A
\textbf{resistance} is measured as a \textbf{base resistance} modified
by a \textbf{degree} depending on how easy or hard the obstacle is to
overcome in genre.
\item[\textbf{Resolution Point (RP)}]
In a \textbf{scored sequence} an \textbf{RP} tracks the advantage one
contestant has over the other.
\item[\textbf{Resource}]
A community \textbf{ability} that your PC may draw on.
\item[\textbf{Result}]
The \textbf{outcome} of a die roll against a \textbf{TN}. One of
\textbf{big success}, \textbf{success}, and \textbf{failure}
\item[\textbf{Retainer}]
A \textbf{follower} of your PC who is not `fleshed out' and cannot act
independently.
\item[\textbf{Risky Gambit}]
In a \textbf{sequence} you can take an action that puts you at more risk
on defeat, but enhances victory.
\item[\textbf{Round}]
A \textbf{sequence} is broken into a series of rounds, each of which is
an attempt to obtain the \textbf{prize}. In a \textbf{wagered sequence}
a round is further broken into a number of \textbf{exchanges} in which
all participants have the chance to act. Resolution is via a
\textbf{contest} that affects the \emph{tally} used for that
\textbf{sequence} type, or applies immediate \textbf{consequences}.
\item[\textbf{Scored Sequence}]
A \textbf{sequence} where we track the relative advantage one contestant
has over another using \textbf{resolution points}
\item[\textbf{Second Chance}]
An attempt by \textbf{defeated}, but unengaged, PCs to re-enter a
\textbf{wagered sequence}.
\item[\textbf{Sequence}]
A \textbf{contest} where we drill-down to the individual exchanges that
resolve the conflict. We support \textbf{scored}, \textbf{extended}, and
\textbf{chained sequences}
\item[\textbf{Sidekick}]
A fleshed out \textbf{follower} of your PC who can act independently.
\item[\textbf{Situational Modifier}]
A \textbf{bonus} or \textbf{penalty} modifying a \textbf{target number}
due to notably clever or foolish \textbf{tactics}.
\item[\textbf{Supporting Characters}]
Additional characters under the player's control that play a supporting
role to their PC.
\item[\textbf{Contest}]
A one roll resolution method, the default \textbf{contest} type, used
when learning the \textbf{outcome} matters more than the breakdown of
how you achieved it.
\item[\textbf{Stand Alone Ability}]
An \textbf{ability} raised separately to a \textbf{keyword}. It may have
been added to the character as part of a \textbf{package keyword}, or on
its own.
\item[\textbf{Story Obstacle}]
Something that prevents you from getting what you want, the
\textbf{prize}. A \textbf{story obstacle} is the trigger for a
\textbf{contest}.
\item[\textbf{Story Question}]
Something that you need to understand before you can move forward in a
story, the \textbf{prize}. A \textbf{story question} is the trigger for
a question.
\item[\textbf{Stretch}]
A \textbf{penalty} applied to an \textbf{ability} because it stretches
credibility that it is a reasonable \textbf{tactic}.
\item[\textbf{Success}]
Rolling under your \textbf{target number}. It can be a \textbf{big
success} or just a plain \textbf{success}.
\item[\textbf{Tactic}]
How you intend to use one of your \textbf{abilities} to overcome a
\textbf{story obstacle}
\item[\textbf{Target Number (TN)}]
The number, either an \textbf{ability} \textbf{rating}, or a
\textbf{resistance}, to roll under or equal to in order to
\textbf{succeed}.
\item[\textbf{TN}]
Abbreviation for \textbf{Target Number}
\item[\textbf{Unrelated Action}]
An action when you are disengaged in a \textbf{sequence} that does not
relate to your attempt to win the \textbf{prize}.
\item[\textbf{Victory}]
Your \textbf{result} is a better roll than the \textbf{resistance}.
\item[\textbf{Wager}]
Also an \textbf{AP Wager} or \textbf{advantage point wager} is your
wager in a \textbf{wagered sequence}.
\item[\textbf{Wagered sequence}]
A type of \textbf{sequence} in which you track the relative advantage
one opponent has over another using \textbf{advantage points}.
\end{description}

\hypertarget{version-changes}{%
\subsection{10.2 Version Changes}\label{version-changes}}

\hypertarget{version-3.0}{%
\subsubsection{Version 3.0}\label{version-3.0}}

These are the major changes in this version of the rules

\begin{itemize}
\tightlist
\item
  Moved to measuring a result by a number of successes and comparing
  them, simplifying masteries.
\item
  Split hero points into story points (bumps) and experience points
  (character improvement). Flaws generate experience points as do
  failures.
\item
  Removed the Degree of Victory. Now just calculate outcome degrees from
  success counts.
\item
  Changed degree of success and failure, to degree of victory and
  defeat, as success and failure are for individual rolls, victory and
  defeat once compared.
\item
  Made degrees codify the +5, +10, +15, \ldots{} progression used
  throughout, for example degrees of resistance.
\item
  For outcomes clarified that contest results are only reciprocal
  between PCs. When the contest is against a resistance set by the GM,
  the results indicate whether the PC gains the prize, and the GM
  narrates the result for the resistance based on this.
\item
  Changed outcomes to emphasize degrees. This change is designed to
  dissuade GMs from misunderstanding that the prize is obtained on a
  marginal victory, one of the most common result types, and instead
  encourage GMs to allow PCs to fail forward on a zero degree victory by
  introducing downstream complications or consequences.
\item
  Provided clarity that consequences of defeat and benefit of victory
  are optional and the GM should focus on using the prize to narrate the
  outcome of a contest, only applying mechanical benefits or penalties
  if they make sense.
\item
  Specific Ability Bonuses are dropped. They were hard for the GM to
  adjudicate and the same intent is better served by using a stretch on
  a broad ability when contesting against a PC with a more specific
  ability.
\item
  Made it clear that only a PC should use a parting shot, not the
  resistance.
\item
  Sequences replace all `long' contest types. Between version 1 and
  version 2 extended contests switched to scored contests, this approach
  restores both variants, but changes the name to a sequence
  generically, factoring out commonality, and to scored and wagerding
  respectively. Goal is to show contest as the atomic unit within a
  sequence.
\item
  Dropped edges and handicaps - we use a resistance not stats, so makes
  no sense to have edges and handicaps
\item
  Added Mythic Russia's Plot Edits
\item
  Simplified how multiple opponents are handled.
\item
  Clarified contest outcomes for sequences, and how to determine the
  overall winner in a sequence.
\item
  Do not allow transfers in a wagered sequence where the abilities
  differ by 6 or more. Consistent with benefits of victory and prevents
  `loading up on mooks' as a strategy.
\end{itemize}

\end{document}
